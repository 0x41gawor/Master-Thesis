%-----------------------------------------------
%  Engineer's & Master's Thesis Template
%  Copyleft by Artur M. Brodzki & Piotr Woźniak
%  Warsaw University of Technology, 2019-2022
%-----------------------------------------------

\documentclass[
    bindingoffset=5mm,  % Binding offset
    footnoteindent=3mm, % Footnote indent
    hyphenation=true    % Hyphenation turn on/off
]{src/wut-thesis}

\graphicspath{{tex/img/}} % Katalog z obrazkami.
\addbibresource{bibliografia.bib} % Plik .bib z bibliografią

\facultyeiti    
\MasterThesis 
\langpol 

\begin{document}

%------------------
% Strona tytułowa
%------------------
\instytut{Telekomunikacji}
\kierunek{Telekomunikacja}
\specjalnosc{Teleinformatyka i Zarządzanie w Telekomunikacji}
\title{
    Propozycja architektury platformy do modelowania i uruchamiania zamkniętych pętli sterowania w Kubernetes
}
\engtitle{
    Creation of a platform for designing and running closed control loops in Kubernetes
}
\author{Andrzej Gawor}
\album{300528}
\promotor{dr inż. Dariusz Bursztynowski}
\date{\the\year}
\maketitle

%-------------------------------------
% Streszczenie po polsku 
%-------------------------------------
\cleardoublepage
\abstract
Projekt opisany w niniejszej pracy skupia się na zaproponowaniu oraz zaimplementowaniu reużywalnej architektury (zwanej dalej "Platformą"), która pozwala na modelowanie oraz uruchamianie zamkniętych pętli sterowania w Kubernetes. Genezą projektu jest praca jednego z komitetów ETSI o nazwie "ENI - Experiential Networked Intelligence", która skupia się na ułatwieniu pracy operatora sieci telekomunikacyjnych wykorzystując mechanizmy sztucznej inteligencji w zamkniętych pętlach sterowania opartych na politykach, które są sterowanie metadanymi oraz świadome kontekstu. ENI w jednym ze swoich dokumentów dokonuje przeglądu zamkniętych pętli sterowania znanych ludzkości z innych dziedzin. Naturalnym następnym krokiem jest zapropowanie platformy, na której operator mógłby takowe pętle projektować oraz uruchamiać. W tym celu zdefiniowanio zestaw wymagań oraz założeń dla takiego systemu. Jako środowisko uruchomieniowe wybrano  Kubernetes z racji, że jest to system dobrze znany w społeczności oraz sam natywnie używa zamkniętych pętli sterowania. Nastpęnie przeprowadzono obszerną analizę jak za pomocą mechanizmów rozszerzania Kubernetes takich jak "Custom Resources" oraz "Operator" pattern można stworzyć framework umożliwiający modelowanie zamkniętych pętli sterowania. Praca opisuje powstałą platformę, jej architekturę, semantykę składni w definiowanych obiektach, zasady działania, integracje z zewnętrznymi systemami wraz z instrukcją jej użycia. Omówiona została również implemtacja platformy, technologie za nią stojące oraz decyzje podjęte podczas jej powstawania. Finalnie przedstawiono również test działania platformy w praktyce wykorzystując do tego emulator systemu 5G jakim jest Open5GS w połączeniu z UERANSIM. Pracę podsumuję lista wniosków oraz potencjalnych rozszerzeń lub poprawek platformy. 
\keywords Zamknięte pętle sterowania, Kubernetes, Zarządzanie sieciami telekomunikacyjnymi, Automatyzacja, Go, Open5GS, Mikroserwisy

%----------------------------------------
% Streszczenie po angielsku
%----------------------------------------
\clearpage
\secondabstract
The project described in this work focuses on proposing and presenting the implementation of a platform architecture that enables the modeling and execution of closed control loops in Kubernetes. The genesis of the project stems from the work of standardization committees such as ETSI ZSM and ETSI ENI, which focus on specifying autonomous and cognitive management systems for telecommunication networks. The architectures of these specified systems are based on closed control loops.

A natural next step is to propose a platform where an operator could design and execute such loops. To this end, a set of requirements and assumptions for such a system has been defined. Kubernetes was chosen as the runtime environment. Subsequently, an extensive analysis was conducted on how Kubernetes extension mechanisms, such as "Custom Resources" and the "Operator Pattern," can be leveraged to create a framework that enables the modeling of closed control loops. Based on this analysis, a highly flexible architecture was developed and then implemented using the Kubebuilder framework. Finally, the platform was tested using a 5G network emulator, specifically Open5GS combined with UERANSIM.

This work describes a review of related literature, the developed platform, its architecture, and a user guide. It also discusses an exemplary implementation of the platform, the technologies behind it, and the decisions made during its development. Finally, a practical test of the platform's functionality is presented. The work concludes with a summary of findings and potential future development paths for the platform.
\secondkeywords Closed Control Loops, Kubernetes, Managing Telco-Networks, Automation, Go, Open5GS, Microservices

\pagestyle{plain}

%--------------
% Spis treści
%--------------
\cleardoublepage
\tableofcontents

%------------
% Rozdziały
%------------
\cleardoublepage 
\pagestyle{headings}
\newpage
\section{Wstęp}
\subsection{Przedmowa}

Wraz z rozwojem telekomunikacji stopień jej skomplikowania oraz mnogość podłączonych urządzeń stale rośnie. Sieci 5G zwiastują obsługę miliardów urządzeń, co sprawia, że tradycyjne podejście do zarządzania sieciami staje się niewystarczające. W pewnym momencie manualne operowanie sieciami (ang. \textit{human-driven networks}) stanie się wręcz niemożliwe. Dlatego obserwujemy obecnie zwrot w stronę wirtualizacji oraz automatyzacji sieci. Równocześnie dynamiczny rozwój sztucznej inteligencji otwiera nowe możliwości. Te dwa czynniki stanowią wspólnie solidny fundament do tego, aby branża sieci telekomunikacyjnych postawiła sobie za cel budowę "inteligentnych" sieci - takich, które są w pełni autonomiczne, samowystarczalne oraz nie wymagają nadzoru ludzkiego.

W tym celu ETSI (European Telecommunications Standards Institute) powołało dwa komitety: ENI (Experiential Networked Intelligence) oraz ZSM (Zero touch network \& Service Management). Oba komitety mają na celu wypracowanie specyfikacji Kognitywnych Systemów Zarządzania Siecią (Cognitive Network Management system). Kognitywny system oznacza taki, który jest w stanie uczyć się i podejmować decyzje, bazując na zebranej wiedzy, w sposób przypominający ludzki umysł. Obie architektury opierają swoje działanie na sztucznej inteligencji oraz zamkniętych pętlach sterowania. 

Pętlą sterowania ETSI nazywa mechanizm, który monitoruje wydajność systemu lub procesu poddawanego kontroli w celu osiągnięcia pożądanego zachowania. Innymi słowy, pętla sterowania reguluje działanie zarządzanego obiektu. Pętle sterowania można podzielić na zamknięte lub otwarte, w zależności od tego, czy działanie sterujące zależy od sprzężenia zwrotnego z kontrolowanego obiektu. Jeśli tak, pętle nazywamy zamkniętą, jeśli nie - otwartą. 

W przypadku architektur ETSI, zamknięta pętla sterowania służy jako model organizacji przepływu pracy (ang. \hyperlink{def:workflow}{\textit{workflow}}) elementów odpowiedzialnych za sztuczną inteligencję. Obie architektury referencyjne opisują modelowanie, uruchamianie oraz zarządzanie zamkniętymi pętlami sterowania w celu osiągania celów zarządzania. Jednak wciąż brakuje ogólnodostępnej platformy, która pozwalałaby operatorom w praktyce projektować, wdrażać i zarządzać takimi pętlami sterowania zgodnie z założeniami ETSI ZSM i ENI, co stanowi istotną lukę technologiczną. Naturalnym następnym krokiem jest zaproponowanie architektury platformy, która umożliwia takie działania oraz wpisuje się w aktualne trendy systemów teleinformatycznych, takich jak podejście cloud-native, wykorzystanie mikrousług oraz integracja z ekosystemem Kubernetes.

\input{tex/1-wstep/2-cele-pracy.tex}
\subsection{Struktura pracy}

Praca została podzielona na 6 rozdziałów. Drugi rozdział przedstawia bardziej szczegółowo niż we wstępie badania podjęte przez ENI. Stanowi on wprowadzenie teoretyczne oraz pojęciowe, aby ułatwić przekaz w dalszej części pracy. Rozdział trzeci zawiera opis proponowanej architektury. Rozdział czwarty opisuje implementację PoC platformy oraz napotkane wyzwania podczas formułowania architektury, które celowo zostały zebrane w jedno miejsce i umieszczone oddzielnie w celu łatwiejszej lektury. Rozdział numer pięć przedstawia użycie platformy w praktyce przy okazji stanowiąc jej test. Na koniec, w rozdziale szóstym przeprowadzono analizę zaproponowanej architektury, jej możliwości oraz ograniczenia oraz wskazano kierunki potencjalnego rozwoju.
\section{Stan wiedzy}
\subsection{Wstęp}

Rozdział ten stanowi krótki wstęp historyczny problemu jakiego podejmuje się praca oraz przytacza kluczowe znaleziska z powiązanej literatury w celu identyfikacji luki badawczej. 

\subsection{Historia}

Na początku XX wieku, gdy sieci telekomunikacyjne dopiero się rozwijały, wszystkie połączenia zestawiane były ręcznie. W momencie, gdy abonent podnosił słuchawkę telefonu, jego aparat wysyłał sygnał do lokalnej centrali telefonicznej. Na tablicy świetlnej zapalała się lampka informująca telefonistkę o próbie połączenia. Telefonistka odbierała połączenie, pytając, z kim abonent chce się połączyć. Nastpnie wprowadzała odpowiednią wtyczkę do odpowiedniego gniazda na tablicy rozdzielczej, zestawiając fizyczne połączenie między dwoma liniami. Jeśli rozmowa miała się odbyć na większą odległość (np. między miastami) połączenie przekazywane było przez kolejne centrale. Każda centrala po drodze wymagała ręcznej obsługi przez pracujące w nich telefonistki. 

\begin{figure}[!h]
    \centering \includegraphics[width=1\linewidth]{telefonistki.jpg}
    \caption{Pracowniczki warszawskiej centrali telefonicznej na zdjęciu z końca lat 20. (domena publiczna).}\label{fig:telefonistki}
\end{figure}

Dziś taki scenariusz wydaje się wręcz absurdalny a oferty pracy dla telefonistek dawno już zniknęły z tablic ogłoszeń. Praca wykonywana przez telefonistki (czyli komutacja łączy) nadal jest potrzebna do prawidłowego funkcjonowania sieci telekomunikacyjnych, lecz wykonywana jest przez programy komputerowe w sposób w pełni zautomatyzowany. Ręczna komutacja była pierwszym krokiem w kierunku rozwoju globalnych sieci telekomunikacyjnych. Choć z dzisiejszej perspektywy wydaje się być bardzo pracochłonna i ograniczająca, bez niej nie powstałyby fundamenty, na których oparto późniejsze systemy automatyczne. Jest to przykład tego, jak technologia stopniowo uwalniała człowieka od bezpośredniej obsługi różnych systemów (dając mu przestrzeń na rózwój w innych obszarach). 

Na zasadzie indukcji możemy przyjąć, iż dziś znajdujemy się w podobnym położeniu - sieci telekomunikacyjne wciąż wymagają bezpośredniego zarządzania przez człowieka. Po prostu granica tego styku system-człowiek jest mocno przesunięta. Dziś człowiek spotyka się z systemami dużo bardziej złożonymi, a zarazem dającymi dużo więcej możliwości. Możliwe, że nie istnieje ostateczny punkt styku i systemy telekomunikacyjne zawsze będą wymagały nadzoru ludzkiego. Niezależnie od tej kwestii przesunięcie tej granicy w czasach telefonistek, a dziś wymaga automatyzacji innego rodzaju. 

//TODO no i tu juz musze czytac ENI i odnieść to do pojęć z ENI.
\section{Architektura}

\subsection{Wstęp}

Niniejszy rozdział przedstawia zaproponowaną w pracy architekturę  platformy, na której możliwe jest modelowanie oraz uruchamianie workflow zamkniętych pętli sterowania. Platformie nadano nazwę w celu ułatwienia jej opisu. Nazwa brzmi "Lupus". Powstała od przekształcenia angielskiego słowa "loops" oznaczającego pętle, oraz od zakotwiczenia o wyraz mający znaczenie nadające się na "maskotkę" projektu. "Lupus" po łacinie oznacza wilka, stąd w logo projektu wilk. 

Architektura w rozumieniu ENI jest to "zbiór reguł i metod opisujących funkcjonalność, organizację oraz implementację systemu". Niniejszy rozdział pomija aspekt implementacji, która jest omówiona w następnym rozdziale.


\subsection{Wymagania i założenia}

Po pierwsze Lupus ma przyjąć rolę \textit{regulatora} (ang. \textit{Control System}) znanego z teorii sterowania. Regulowanymi systemami mają być w tym przypadku systemy telekomunikacyjne. Aby odegrać rolę ENI System, platforma musi być w stanie zamodelować oraz uruchomić \underbar{dowolną} zamkniętą pętle sterowania, zwłaszcza te zawarte w \cite{enioverview}. 

Z racji ogólności systemu oraz zwiększenia szansy na pozytywne przyjęcie w społeczności wybrano Kubernetes jako infrastrukturę dla platformy. Kubernetes natywnie używa zamkniętych pętli sterowania w swojej warstwie sterowania (ang. \textit{control plane}), z których użytkownik jest w stanie skorzystać za pomocą mechanizmów rozszerzeń Kubernetes takich jak \textit{Definicje Zasobów Własnych} (ang. \textit{Custom Resource Definitions (CRD)} oraz \textit{Wzorzec Operatora} (ang. )\textit{Operator Pattern}). Są to mechanizmy dobrze znane w branży. 

Wybór Kubernetes narzuca jednakże pewne ograniczenie. Regulowane procesy nie mogą odbywać się w czasie rzeczywistym. Warstwa sterowania Kubernetes działa nieco wolniej. 

Lupus musi być "data-driven" co na polski można przetłumaczyć jako "sterowany danymi". Oznacza to, że platforma nie może narzucać żadnej postaci logiki pętli. Warstwa sterowania Kubernetes musi być w stanie interpretować zamiary użytkownika platformy, który może wyrazić dowolną pętle. Zamiary te wyrażone są właśnie w danych. 

Logikę pętli możemy podzielić na dwie części: workflow pętli oraz części obliczeniowa. Workflow jest to zdefiniowane elementów oraz relacji między nimi. Częścią obliczeniową za to nazywamy procesowanie wykonywane przez konkretne elementy. Z racji podejścia "data-driven" nie możemy zaszyć części obliczeniowej w warstwie sterowania Kubernetes. Dlatego elementy odpowiedzialne za części obliczeniową są "na zewnątrz" pętli Lupus, przykładowo są to serwery HTTP, do których Lupus wykonując pętle może się odwołać.

Z tego opisu powstaje garść wymagań oraz założeń, które nie są zbiorami rozłącznymi i wielu miejscach się zacierają, jednakże warto wyszczególnić je w sposób wylistowany poniżej, aby móc łatwiej się do nich odwoływać w dalszej części pracy:

\begin{enumerate}
    \item \label{req:1} Lupus jest skierowany do branży telekomunikacyjnej.
    \item \label{req:2} Lupus umożliwia modelowanie i uruchamianie dowolnych architektur zamkniętych pętli sterowania, w szczególności tych zaproponowanych w \cite{enioverview}.
    \item \label{req:3} Lupus zarządza procesami, które nie wymagają regulacji w czasie rzeczywistym (są "non-realtime").
    \item \label{req:4} Lupus jest zaimplementowany na bazie Kubernetes, wykorzystując jego \textit{Controller Pattern}.
    \item \label{req:5} Lupus jest oparty na danych (\textit{data-driven}), co oznacza, że nie narzuca i nie ma wbudowanej żadnej domyślnej logiki pętli.
    \item \label{req:6} Faktyczne komponenty przetwarzające w pętli (część obliczeniowa) Lupus są zewnętrzne względem niego (np. serwery HTTP, szczególnie Open Policy Agent).
    \item \label{req:7} Lupus powinien być w stanie regulować pracą dowolnego systemu teleinformatycznego bez żadnych jego modyfikacji.
    \item \label{req:8} Zamodelowanie pętli oraz wyrażenie jej \hyperlink{def:workflow-petli}{\textbf{Workflow}} w Lupus nie powinno wymagać umiejętności technicznych.
\end{enumerate}

Powyższa lista nazwana jest Wymaganiami i referowana w dalszej części dokumentu. 

\subsection{Pojęcia i zasady}

\subsubsection{Wstęp}
Ten rozdział dokumentuje \hyperlink{def:lupus}{\textbf{Lupus}} w formie wyjaśniania zdefiniowanych na jego potrzeby pojęć, konceptów i zasad. Są one punktem wyjścia do dokładniejszych specyfikacji i zarazem zrozumienia architektury systemu. Rozdział ten zawiera wiele odnośników do definicji znajdujących się w załącznikach i nie wszystkie pojęcia tłumaczone są w tym rozdziale.

\subsubsection{Problem zarządzania}

W rzeczywistym świecie często napotkamy sytuacje, w których byłoby dobrze, gdyby praca jakiegoś systemu mogła być stale regulowana. Na przykład:

\begin{itemize}
    \item chcielibyśmy, aby samochody miały funkcję regulującą pracę silnika w celu utrzymania stałej prędkości,
    \item przydałoby się, gdyby lodówka mogła utrzymywać chłodną, ale nie ujemną temperaturę, niezależnie od tego, jak często otwierane są drzwi lub jaka jest temperatura na zewnątrz,
    \item byłoby korzystne, gdyby serwer w chmurze mógł zagwarantować, że aplikacja z wystarczającymi zasobami do pokrycia potrzeb użytkowników będzie uruchomiona i działała.
\end{itemize}

Problemy wymienione powyżej można traktować jako \hyperlink{def:problem-zarzadzania}{\textbf{problemy zarządzania}} (ang. \textit{management problems}). Nazwa bierze się stąd, że nie ma żadnych technicznych ograniczeń uniemożliwiających osiągnięcie tych celów. Wszystkie wymienione systemy są odpowiednio wyposażone; np. możemy dodać więcej paliwa do silnika lub dostarczyć więcej mocy do sprężarki w lodówce. Problem leży w faktycznym wykonaniu tych czynności w odpowiednich momentach np. dodanie więcej paliwa, gdy auto zwalnia lub dostarczenie większej mocy, gdy temperatura w lodówce wzrasta. Dlatego jest to problem czystego zarządzania.

Systemem, którym chcemy zarządzać, nazywamy \hyperlink{def:system-zarzadzany}{\textbf{System Zarządzany}} (ang. \textit{managed system}).

\subsubsection{System Sterowania}

\hyperlink{def:system-sterowania}{\textbf{System Sterowania}} (ang. \textit{Control System\footnote{innym tłumaczeniem na polski jest "regulator"}}) to system, który reguluje pracą \hyperlink{def:system-zarzadzany}{\textbf{Systemu Zarządzanego}}. Przykładowo jest to:
\begin{itemize}
    \item tempomat, który reguluje pracą silnika w celu utrzymania stałej prędkości,
    \item lodówka, która reguluje pracą sprężarki w celu utrzymania stałej, chłodnej temperatury,
    \item Kubernetes, który reguluje liczbę działających Podów, aby utrzymać pożądaną dostępność aplikacji.
\end{itemize}

Innymi słowy mówiąc, \hyperlink{def:system-sterowania}{\textbf{System Sterowania}} rozwiązuje \hyperlink{def:problem-zarzadzania}{\textbf{Problem Zarządzania}}.

\subsubsection{Pętla Sterowania}

Ogólną architekturą \hyperlink{def:system-sterowania}{\textbf{Systemów Sterowania}} używaną do rozwiązywania \hyperlink{def:problem-zarzadzania}{\textbf{Problemów Zarządzania}} jest \hyperlink{def:petla-sterowania}{\textbf{Pętla Sterowania}}.

Pętle sterowania są klasyfikowane w zależności od tego, czy wykorzystują mechanizmy sprzężenia zwrotnego (ang. \textit{feedback mechanism}):
\begin{itemize}
    \item \hyperlink{def:petla-sterowania}{\textbf{Otwarte Pętle Sterowania}}: \hyperlink{def:akcja-sterujaca}{\textbf{Akcja Sterująca}} (ang. \textit{Control Action}) (czyli wejście do \hyperlink{def:system-zarzadzany}{\textbf{Systemu Zarządzanego}}) jest niezależne od wyjścia \hyperlink{def:system-zarzadzany}{\textbf{Systemu Zarządzanego}}.
    \item \hyperlink{def:zamknieta-petla-sterowania}{\textbf{Zamknięte Pętle Sterowania}}: Wyjście \hyperlink{def:system-zarzadzany}{\textbf{Systemu Zarządzanego}} jest sprzężane do wejścia \hyperlink{def:system-sterowania}{\textbf{Systemu Sterowania}} i wpływa na \hyperlink{def:akcja-sterujaca}{\textbf{Akcję Sterującą}}. 
\end{itemize}

W \hyperlink{def:lupus}{\textbf{Lupus}} bierzemy pod uwagę wyłącznie \hyperlink{def:zamknieta-petla-sterowania}{\textbf{Zamknięte Pętle Sterowania}}.

\subsubsection{Zamknięta Pętla Sterowania}

Jest to punkt wyjściowy dla naszej architektury referencyjnej (rys. \ref{fig:33-arch}).

\begin{figure}[!h]
    \centering \includegraphics[width=1\linewidth]{33-arch.png}
    \caption{Architektura referencyjna dla Lupus. Źródło: Opracowanie własne.}\label{fig:33-arch}
\end{figure}

Architekturę (rys. \ref{fig:33-arch}) należy czytać, mając na uwadze następujące definicje:

\begin{itemize}
    \item \hyperlink{def:system-sterowania}{\textbf{System Sterowania}} (ang. \textit{Control System}) - system, który rozwiązuje \hyperlink{def:problem-zarzadzania}{\textbf{Problem Zarządzania}} występujący w \hyperlink{def:system-zarzadzany}{\textbf{Systemie Zarządzanym}} za pomocą \hyperlink{def:zamknieta-petla-sterowania}{\textbf{Zamkniętej Pętli Sterowania}}. W każdej iteracji \hyperlink{def:system-sterowania}{\textbf{System Sterowania}} analizuje \hyperlink{def:sprzezenie-zwrotne}{\textbf{Sprzężenie Zwrotne}} (ang. \textit{Control Feedback}) i \textit{wnioskuje} \hyperlink{def:akcja-sterujaca}{\textbf{Akcję Sterującą}}.
    \item \hyperlink{def:zamknieta-petla-sterowania}{\textbf{Zamknięta Pętla Sterowania}} - nieskończona pętla, która reguluje stan \hyperlink{def:system-zarzadzany}{\textbf{Systemu Zarządzanego}}, iteracyjnie zbliżając jego \hyperlink{def:stan-aktualny}{\textbf{Stan Aktualny}} do \hyperlink{def:stan-pozadany}{\textbf{Stanu Pożądanego}}. 
    \item \hyperlink{def:akcja-sterujaca}{\textbf{Akcja Sterująca}} (ang. \textit{Control Action}) - akcja wykonywana na \hyperlink{def:system-zarzadzany}{\textbf{Systemie Zarządzanym}}, która ma na celu przybliżenie go do \hyperlink{def:stan-pozadany}{\textbf{Stanu Pożądanego}}.
    \item \hyperlink{def:sprzezenie-zwrotne}{\textbf{Sprzężenie Zwrotne}} (ang. \textit{Control Feedback}) - reprezentacja \hyperlink{def:stan-aktualny}{\textbf{Stanu Aktualnego}} wysyłana z (odbierana od) \hyperlink{def:system-zarzadzany}{\textbf{Systemu Zarządzanego}}.
\end{itemize}

W powyższej architekturze \hyperlink{def:lupus}{\textbf{Lupus}} pełni rolę \hyperlink{def:system-sterowania}{\textbf{Systemu Sterowania}}.


\subsubsection{Agenci Translacyjni}
Z racji, że każdy system, bez żadnych modyfikacji (\hyperref[req:15]{Wymaganie 15}) może wejść w rolę \hyperlink{def:system-zarzadzany}{\textbf{Systemu Zarządzanego}}, potrzebujemy warstwy integracji między \hyperlink{def:system-zarzadzany}{\textbf{Systemem Zarządzanym}} a \hyperlink{def:lupus}{\textbf{Lupus}}, podobnej do koncepcji "API Broker" w architekturze \hyperlink{def:eni}{ENI} (rys. \ref{fig:23-eni-arch}). Tak narodziła się koncepcja \hyperlink{def:agent-translacji}{\textbf{Agentów Translacji}} (ang. \textit{Translation Agents}). 

W każdym \hyperlink{def:wdrozenie-lupus}{\textbf{wdrożeniu Lupus}}, \hyperlink{def:uzytkownik}{\textbf{Użytkownik}} musi stworzyć \hyperlink{def:agent-translacji}{\textbf{Agentów Translacyjnych}}. 

Z punktu widzenia komunikacji, każdego \hyperlink{def:agent-translacji}{\textbf{Agenta Translacyjnego}} możemy podzielić na dwie części:
\begin{itemize}
    \item Część komunikująca się z \hyperlink{def:system-zarzadzany}{\textbf{Systemem Zarządzanym}}. Jest zewnętrzna względem \hyperlink{def:lupus}{\textbf{Lupus}} i nie podlega żadnej specyfikacji.
    \item Część komunikująca się z \hyperlink{def:lupus}{\textbf{Lupus}}. Musi być zgodna z jednym z \hyperlink{def:interfejsy-lupus}{\textbf{Interfejsów Lupus}}: \hyperlink{def:interfejs-lupin}{\textbf{Lupin}} lub \hyperlink{def:interfejs-lupout}{\textbf{Lupout}}.
\end{itemize}

Mamy dwóch \hyperlink{def:agent-translacji}{\textbf{Agentów Translacji}}, jednego do komunikacji przychodzącej (ang. \textit{Ingress}) i jednego do wychodzącej (ang. \textit{Egress}). 

W każdej iteracji pętli zadaniem \hyperlink{def:agent-ingress}{\textbf{Agenta Ingress}} jest odbieranie/zbieranie \hyperlink{def:sprzezenie-zwrotne}{\textbf{Sprzężenia Zwrotnego}} z \hyperlink{def:system-zarzadzany}{\textbf{Systemu Zarządzanego}} i translacja go na format zrozumiały przez \hyperlink{def:lupus}{\textbf{Lupus}} za pomocą \hyperlink{def:interfejs-lupin}{\textbf{Interfejsu Lupin}}. Z kolei zadaniem \hyperlink{def:agent-egress}{\textbf{Agenta Egress}} jest odbieranie \hyperlink{def:finalne-dane}{\textbf{Finalnych Danych}} i translacja ich na \hyperlink{def:akcja-sterujaca}{\textbf{Akcję Sterującą}}, która później zostaje wysłana do (lub przeprowadzona na) \hyperlink{def:system-zarzadzany}{\textbf{Systemie Zarządzanym}}.

Architektura wraz z wprowadzeniem \hyperlink{def:agent-translacji}{\textbf{Agentów Translacji}} ukazana jest na rysunku \ref{fig:33-arch2}.

\begin{figure}[!h]
    \centering \includegraphics[width=1\linewidth]{33-arch2.png}
    \caption{Architektura referencyjna z agentami translacji. Źródło: Opracowanie własne.}\label{fig:33-arch2}
\end{figure}

Specyfikacja interfejsów \hyperlink{def:interfejs-lupin}{\textbf{Lupin}} oraz \hyperlink{def:interfejs-lupout}{\textbf{Lupout}} zawarta jest w \hyperref[appendix:4]{Załączniku 4}.

\subsubsection{Workflow pętli}

Kiedy \hyperlink{def:lupus}{\textbf{Lupus}} otrzyma \hyperlink{def:stan-aktualny}{\textbf{Stan Aktualny}} poprzez interfejs \hyperlink{def:interfejs-lupin}{\textbf{Lupin}}, rozpoczyna się \hyperlink{def:workflow-petli}{\textbf{Workflow Pętli}}, które ma na celu dostarczać \hyperlink{def:logika-petli}{\textbf{Logikę Pętli}} (ang. \textit{Loop Logic}) i składa się z \hyperlink{def:element-petli}{\textbf{Elementów Pętli}} (ang. \textit{Loop Elements}). 

Elementem pętli może być zarówno:
\begin{itemize}
    \item \hyperlink{def:element-lupus}{\textbf{Element Lupus}}, który działa w warstwie sterowania Kubernetes, a jego misją jest wykonywać \hyperlink{def:workflow-petli}{\textbf{Workflow Pętli}}.
    \item referencja do \hyperlink{def:element-zewnetrzny}{\textbf{Elementu Zewnętrznego}}, który działa poza warstwą sterowania Kubernetes, a jego misją jest wykonywać \hyperlink{def:czesc-obliczeniowa}{\textbf{Część Obliczeniową}} \hyperlink{def:logika-petli}{\textbf{Logiki Pętli}}.
\end{itemize}

\hyperlink{def:workflow-petli}{\textbf{Workflow Pętli}} jest wyrażany w \hyperlink{def:lupn}{\textbf{LupN}}, specjalnej notacji do opisywania workflow pętli, której dokładna specyfikacja znajduje się w \hyperref[appendix:3]{Załączniku 3}.

Przykładowe \hyperlink{def:workflow-petli}{\textbf{Workflow Pętli}} pokazano na rysunku \ref{fig:33-workflow}.

\begin{figure}[!h]
    \centering \includegraphics[width=1\linewidth]{33-workflow.png}
    \caption{Przykładowe workflow pętli. Źródło: Opracowanie własne.}\label{fig:33-workflow}
\end{figure}

\begin{itemize}
    \item Niebieskie, zaokrąglone prostokąty reprezentują \hyperlink{def:element-lupus}{\textbf{Elementy Lupus}}.
    \item Czerwone prostokąty oznaczają \hyperlink{def:element-zewnetrzny}{\textbf{Elementy Zewnętrzne}}.
    \item Niebieski obszar wyznaczony linią przerywaną wskazuje elementy działające w warstwie sterowania Kubernetes. 
    \item Wyróżniono \hyperlink{def:element-lupus}{\textbf{Elementy Lupus}} odpowiedzialne za \hyperlink{def:interfejs-lupin}{\textbf{Ingress}} i \hyperlink{def:interfejs-lupout}{\textbf{Egress}}.
\end{itemize}

\hyperlink{def:element-zewnetrzny}{\textbf{Elementy Zewnętrzne}} to zazwyczaj serwery HTTP (w szczególności serwery \hyperlink{def:opa}{\textbf{Open Policy Agent}}).

Jeden \hyperlink{def:element-lupus}{\textbf{Element Lupus}} może komunikować się z żadnym, jednym lub wieloma \hyperlink{def:element-zewnetrzny}{\textbf{Elementami Zewnętrznymi}}, a liczba ta może się różnić w każdej iteracji pętli.

\subsubsection{Dane i Akcje}

\hyperlink{def:dane}{\textbf{Dane}} (ang. \textit{data}) to nośnik informacji w ramach jednej iteracji pętli w formacie JSON. Na wejściu \hyperlink{def:element-ingres}{\textbf{Elementu Ingress}} reprezentują one \hyperlink{def:stan-aktualny}{\textbf{Stan Aktualny}} \hyperlink{def:system-zarzadzany}{\textbf{Systemu Zarządzanego}}. Następnie, w trakcie iteracji pętli, \hyperlink{def:projektant}{\textbf{Projektant}} decyduje, jakie informacje będą przenosić. Zazwyczaj są to informacje związane z \hyperlink{def:logika-petli}{\textbf{Logiką Pętli}}, takie jak wejścia (ang. \textit{inputs}) do \hyperlink{def:element-zewnetrzny}{\textbf{Elementów Zewnętrznych}} oraz ich odpowiedzi. Na końcu iteracji, gdy \hyperlink{def:dane}{\textbf{Dane}} trafiają do \hyperlink{def:agent-egress}{\textbf{Agenta Egress}}, muszą reprezentować \hyperlink{def:akcja-sterujaca}{\textbf{Akcję Sterowania}}. 

Działanie pojedynczego \hyperlink{def:element-lupus}{\textbf{Elementu Lupus}} jest opisane poprzez \hyperlink{def:workflow-petli}{\textbf{Workflow Akcji}}. \hyperlink{def:akcja}{\textbf{Akcje}} wykonują różne operacje na \hyperlink{def:dane}{\textbf{Danych}}. Istnieje wiele rodzajów akcji, a kluczowym typem akcji jest "Send", która pozwala komunikować się \hyperlink{def:element-lupus}{\textbf{Elementowi Lupus}} z \hyperlink{def:element-zewnetrzny}{\textbf{Elementem Zewnętrznym}}. Inne typy akcji służą organizacji danych.

\begin{figure}[!h]
    \centering \includegraphics[width=1\linewidth]{33-workflow-akcji.png}
    \caption{Przykładowe workflow akcji. Źródło: Opracowanie własne.}\label{fig:33-workflow-akcji}
\end{figure}

Pełna specyfikacja \hyperlink{def:dane}{\textbf{Danych}} znajduje się w \hyperref[appendix:5]{Załączniku 5}, zaś pełna specyfikacja \hyperlink{def:akcja}{\textbf{Akcji}} w \hyperref[appendix:6]{Załączniku 6}.

\subsubsection{Open Policy Agent}

\hyperlink{def:opa}{\textbf{Open Policy Agent}} (OPA) to otwartoźródłowe narzędzie przeznaczone do definiowania, egzekwowania i zarządzania politykami w systemach oprogramowania. Systemy informatyczne odpytują serwery \hyperlink{def:opa}{\textbf{OPA}} podczas wykonywania operacji, które wymagają decyzji (np. decyzji dostępu, konfiguracji czy zachowania aplikacji). Dzięki takiemu podejściu można oddzielić logikę podejmowania decyzji od kodu aplikacji, co zwiększa modularność, wprowadza centralny punkt zarządzania politykami oraz ułatwia utrzymanie systemów.

Polityki w \hyperlink{def:opa}{\textbf{OPA}} definiowane są w języku Rego. Jest to deklaratywny język stworzony specjalnie na potrzeby \hyperlink{def:opa}{\textbf{OPA}}. Pozwala na tworzenie złożonych reguł i logiki decyzji. Oficjalna strona projektu dostępna jest pod linkiem: \url{https://www.openpolicyagent.org}.

\begin{lstlisting}[language=sh, caption={\emph{Przykładowy kod rego}}\label{lst:1}]
allow {
    input.user.role == "admin"
}

allow {
    input.user.role == "user"
    input.action == "read"
}
\end{lstlisting}

\hyperlink{def:opa}{\textbf{Open Policy Agent}} jest rekomendowanym \hyperlink{def:element-zewnetrzny}{\textbf{Elementem Zewnętrznym}} dla \hyperlink{def:lupus}{\textbf{Lupus}}.

\clearpage
\section{Implementacja}

\subsection{Wstęp}
Rozdział ten opisuje wydzieloną z implementację Lupus. Pierwsza część rozdziału opisuje kluczowe mechanizmy Kubernetes, wykorzystane przy implementacji Lupus. Druga część przedstawia decyzje podjęte podczas implementacji platformy.
\subsection{Mechanizmy Kubernetes stojące za Lupus}
Niniejsza praca zakłada znajomość czytelnika platformy Kubernetes na podstawowym poziomie.

\subsubsection{Kontroler}
Działanie Kubernetes opiera się na zamkniętej pętli sterowania \footnote{https://kubernetes.io/docs/concepts/architecture/controller/}. \textbf{Aktualny Stan} systemu zapisany jest w bazie etcd. \textbf{Stan Pożądany} wyrażony jest poprzez pliki manifestacyjne. Każdy \textit{Obiekt API} posiada swoją sekcję \texttt{spec}, która określa jego pożądany stan. Każdy obiekt ma swój kontroler, który rekoncyliuję aktualny stan do stan pożądanego. Kontroler jest to proces działający w warstwie sterowania Kubernetes. Każdy typ (ang. \textit{Kind}) wbudowanych zasobów (ang. \textit{buil-in resources}) posiada kontroler opracowany przez zespół Kubernetes. Kontrolery każdego typu zasobu działają w podzie \texttt{kube-controller-manager}. 

\begin{figure}[!h]
    \centering \includegraphics[width=1\linewidth]{42-k8s-arch.png}
    \caption{Architektura Kubernetes. Źródło: \url{https://kubernetes.io/docs/concepts/architecture/}}\label{fig:42-k8s-arch}
\end{figure}

Przepływ pracy kontrolera pokazano na rysunku \ref{fig:42-flow}. Gdy \texttt{kube-api-server} otrzymuje żądanie zmiany danego obiektu API, zanim zleci jego utrwalenie w bazie etcd, wykonuje tzw. webhooki do kontrolera danego obiektu. Webhooki nie są jednakże istotne z punktu widzenia niniejszej pracy dyplomowej. Po dokonaniu zmian w obiekcie kontroler zostaje o nich poinformowany. Jego głównym zadaniem jest rekoncyliacja, czyli porównanie aktualnego stanu obiektu ze stanem pożądanym. Kontroler zawiera logikę rekoncyliacyjną, która dąży do doprowadzenia stanu aktualnego do stanu pożądanego. Realizacja logiki rekoncyliacji często wiąże się z wykonywaniem różnych akcji w innych częściach klastra, takich jak skalowanie zasobów, tworzenie nowych instancji lub modyfikacja konfiguracji.

\begin{figure}[!h]
    \centering \includegraphics[width=1\linewidth]{42-flow.png}
    \caption{Flow pracy kontrolera. Źródło: Opracowanie własne.}\label{fig:42-flow}
\end{figure}

\subsubsection{Zasoby własne} 

\hyperlink{def:zasoby-wlasne}{\textit{Zasoby własne} (ang. \textit{Custom Resources})} rozszerzają zasoby wbudowane (ang. \textit{buit-in resources}) o niestandardowe typy, zdefiniowane przez użytkownika. Najczęściej tworzone są w celu zarządzania konfiguracją skomplikowanych lub stanowych aplikacji, tam gdzie wbudowane typy zasobów, takie jak Pody, Wdrożenia (Deployments) czy Serwisy (Services), nie są wystarczające. Aby utworzyć nowy typ zasobu, należy:
\begin{enumerate}
    \item Zdefiniować plik manifestacyjny YAML typu Custom Resource Definition (CRD).
    \item Zaaplikować CRD w klastrze Kubernetes.
\end{enumerate}

Po pomyślnym dodaniu definicji możliwe jest tworzenie obiektów nowego typu, które będą zarządzane przez Kubernetes w taki sam sposób jak zasoby wbudowane.

\subsubsection{Operatory}

Operator to kolejny mechanizm rozszerzania Kubernetes. Kiedy tworzymy Zasoby Własne (ang. \textit{Custom Resources}), możemy również implementować dla nich kontrolery. Takie podejście nazywane jest "wzorem operatora" ("operator-pattern"). Nazwa ta wynika z faktu, że operator pełni funkcję automatycznego zarządcy, zastępując człowieka, który w przeciwnym razie musiałby ręcznie zarządzać wdrożeniem i konfiguracją aplikacji wymagającej Zasobów Własnych. Lupus wykorzystuje zasoby własne w celu reprezentacji zewnętrznego bytu, nie skomplikowanej aplikacji.
\subsubsection{Kubebuilder}

Kubebuilder jest frameworkiem programistycznym do tworzenia \textit{Zasobów Własnych} oraz ich \textit{Operatorów}. Pozwala zaprogramować sekcje przepływu pracy kontrolera zaznaczone na rysunku \ref{fig:42-flow2} jasno czerwonym kolorem.


\begin{figure}[!h]
    \centering \includegraphics[width=1\linewidth]{42-flow2.png}
    \caption{Zaznaczenie możliwych do zaprogramowania elementów flow pracy kontrolera. Źródło: Opracowanie własne.}\label{fig:42-flow2}
\end{figure}
\subsection{Decyzje podjęte podczas developmentu}

\subsubsection{Wstęp}

W tym rozdziale objawia się badawcza natura pracy. Podczas realizacji projektu zbadano wiele podejść oraz rozwiązano wiele problemów implementacyjnych, aby dojść do postawionej w sekcji 3 architektury. Każda podsekcja omawia po jednym z aspektów – od wymagań do implementacji.

\subsubsection{Komunikacja pomiędzy Elementami Lupus}

Pierwszym krokiem w implementacji było wymyślenie sposobu na komunikację pomiędzy \hyperlink{def:element-lupus}{\textbf{Elementami Lupus}}. Z racji, że \hyperlink{def:element-lupus}{\textbf{Elementy Lupus}} są \textit{Zasobami Własnymi}, wykorzystano tu natywne mechanizmy Kubernetes opisane w podsekcji 4.2 i 4.3. Ideą było to, że \textit{Operator} jednego zasobu dokonuje zmian w obiekcie API innego elementu, co z kolei wywoła ponownie Operator z innym obiektem wejściowym. \hyperlink{def:element-lupus}{\textbf{Elementy Lupus}} modyfikują nawzajem swój \texttt{Status}, a dokładniej jego pole \texttt{input}, przekazując tam swoją \hyperlink{def:finalne-dane}{\textbf{Finalną Postać Danych}}.

\begin{figure}[!h]
    \centering \includegraphics[width=1\linewidth]{43-komunikacja.png}
    \caption{Komunikacja pomiędzy Elementami Lupus}\label{fig:43-komunikacja}
\end{figure}

Działa to w sposób ukazany na rysunku \ref{fig:43-komunikacja}. Początkowo \hyperlink{def:element-lupus}{\textbf{Elementy Lupus}} mogły przyjąć postać jednego z 4 typów: Observe, Learn, Decide lub Execute\footnote{W myśl pętli ODA}. \hyperlink{def:agent-ingress}{\textbf{Agent Ingress}} modyfikował \texttt{Status} obiektu Observe, natomiast Operator Observe modyfikował \texttt{Statusy} obiektów Learn oraz Decide. Na koniec modyfikacje otrzymywał obiekt Execute, którego Operator przekazywał swoje \hyperlink{def:dane}{\textbf{Dane}} do \hyperlink{def:agent-egress}{\textbf{Agenta Egress}}. 

Z czasem jednak, aby nie narzucać konkretnej struktury pętli, zrezygnowano z 4 typów i stworzono jeden uniwersalny typ, którego Operator jest w stanie wykonać \hyperlink{def:workflow-petli}{\textbf{Workflow Akcji}}, co pozwalało na wyrażenie logiki dowolnego z 4 poprzednich typów. W ten sposób spełniono \hyperref[req:5]{Wymaganie 5}.

\subsubsection{Dane}

\hyperref[req:5]{Wymaganie 5} nakazuje, aby \hyperlink{def:element-lupus}{\textbf{Elementy Lupus}} były sterowane danymi (\textit{data-driven}). \hyperlink{def:dane}{\textbf{Dane}} oraz \hyperlink{def:workflow-petli}{\textbf{Workflow Akcji}} spełniają to wymaganie. Dzięki nim \hyperlink{def:element-lupus}{\textbf{Elementy Lupus}} mogą wykonać dowolny zestaw \hyperlink{def:akcja}{\textbf{Akcji}} na \hyperlink{def:dane}{\textbf{Danych}}, dając \hyperlink{def:projektant}{\textbf{Projektantowi}} pewną elastyczność. To właśnie \hyperlink{def:dane}{\textbf{Dane}} przekazywane są jako pole \texttt{input} w statusie obiektu API \hyperlink{def:element-lupus}{\textbf{Elementu Lupus}}. 

Z racji uniwersalności \hyperlink{def:dane}{\textbf{Danych}}, należało wybrać format, który pozwoli na ich możliwie dużą dowolność. Wybór padł na JSON, ze względu na jego ogólnie przyjęty standard i możliwości reprezentacji dowolnej struktury danych. 

Kubernetes z kolei wymaga, aby pola umieszczane w \texttt{Statusie} Zasobu miały konkretny typ. Dokonano więc analizy, jaki typ nadaje się do reprezentacji obiektu JSON. Wybór padł na \texttt{RawExtension}. Jest to typ zdefiniowany przez zespół Kubernetes, używany do obsługi dowolnych surowych danych w formacie JSON lub YAML. Należy do pakietu \texttt{k8s.io/apimachinery/pkg/runtime} i jest często stosowany, gdy zasób musi osadzić lub pracować ze strukturą danych, która jest elastyczna, ale jednocześnie uporządkowana. \texttt{RawExtension} spełnia te wymagania.

\begin{lstlisting}[language=go, caption={\emph{Definicja struktury Go reprezentującej status Elementu Lupus}}\label{lst:431}]
// ElementStatus defines the observed state of Element
type ElementStatus struct {
	// Input contains operational data
	Input runtime.RawExtension `json:"input"`
	// Timestamp of the last update
	LastUpdated metav1.Time `json:"lastUpdated"`
}
\end{lstlisting}

\texttt{RawExtension} to typ, którego użyjemy do przenoszenia \hyperlink{def:dane}{\textbf{Danych}} między \hyperlink{def:element-lupus}{\textbf{Elementami Lupus}}. Pozostaje kwestia reprezentacji tych \hyperlink{def:dane}{\textbf{Danych}} w Operatorach \hyperlink{def:lupus}{\textbf{Lupus}}. 

\begin{lstlisting}[language=go, caption={\emph{Definicja struktury Go RawExtension w paczce k8s.io/apimachinery/pkg/runtime}}\label{lst:432}]
type RawExtension struct {
    Raw []byte `json:"-"` // Serialized JSON or YAML data
    Object Object         // A runtime.Object representation
}
\end{lstlisting}

Zamierzonym celem \texttt{RawExtension} według deweloperów Kubernetes jest umożliwienie deserializacji do określonej, znanej struktury. Jednak ze względu na \hyperref[req:5]{Wymaganie 5} taka struktura nie istnieje. Potrzebujemy więc natywnej struktury Go, która jest w stanie reprezentować dowolny obiekt JSON. Pierwszym pomysłem, który się nasuwa, jest użycie typu Go \texttt{interface{}}, ponieważ może on reprezentować dowolne dane. Problem z \texttt{interface{}} polega jednak na tym, że nie można na nim operować – nie udostępnia żadnego interfejsu do interakcji. Jest to typ podstawowy.

Drugim pomysłem na reprezentację JSON-a było użycie \texttt{map[string]interface{}}, ponieważ większość instancji JSON-a to faktycznie obiekty klucz-wartość. Klucze w tym przypadku są typu \texttt{string}, a wartości mogą być dowolne (stąd użycie \texttt{interface{}}) w Go. W większości przypadków obiekty JSON zawierają kilka głównych pól (ang. \textit{top-level fields}), co idealnie pasuje do reprezentacji \texttt{map[string]interface{}}.

Tak właśnie narodziło się pojęcie \hyperlink{def:dane}{\textbf{Danych}}. \hyperlink{def:dane}{\textbf{Dane}} to w rzeczywistości struktura opakowująca i dająca odpowiedni interfejs (ang. \textit{Wrapper}) dla wspomnianej wcześniej mapy.

\begin{lstlisting}[language=go, caption={\emph{Definicja struktury Go dla Danych}}\label{lst:433}]
type Data struct {
	Body map[string]interface{}
}
\end{lstlisting}

Ta struktura posiada bogaty zestaw funkcji (metod), które pełnią rolę interfejsu do pracy z \hyperlink{def:dane}{\textbf{Danymi}}. Metody te są wywoływane podczas wykonywania \hyperlink{def:akcja}{\textbf{Akcji}} i zazwyczaj – z wyjątkiem metod \texttt{Get} i \texttt{Set} – każda metoda odpowiada dokładnie jednej akcji. Kluczowym konceptem \hyperlink{def:dane}{\textbf{Danych}} jest \hyperlink{def:pole-danych}{\textbf{Pole Danych}} (ang. \textit{Data-field}). Jest ono odpowiednikiem pola w JSON-ie. Każde pole jest identyfikowane przez swój \hyperlink{def:klucz}{\textbf{Klucz}} i przechowuje wartość. Za pomocą metody \texttt{Get} możemy pobrać wartość znajdującą się pod danym kluczem, a przy użyciu \texttt{Set} możemy ustawić nową wartość dla pola wskazanego określonym kluczem. 

Nie obejmuje to jednak wszystkich obiektów JSON, jakie istnieją. JSON pozwala, aby element na najwyższym poziomie był tablicą. Nakłada to pewne ograniczenia na projektowanie pętli. Szczególnie JSON reprezentujący aktualny stan \hyperlink{def:system-zarzadzany}{\textbf{Systemu Zarządzanego}}, wysyłany przez \hyperlink{def:interfejs-lupin}{\textbf{Interfejs Lupin}}, musi być serializowalny do \texttt{map[string]interface{}}. Oznacza to, że nie może być:
\begin{itemize}
    \item typem prymitywnym,
    \item tablicą,
    \item obiektem JSON z kluczami innymi niż \texttt{string}.
\end{itemize}


\subsubsection{Polimorfizm w Go}

Notacja \textbf{LupN} pozwala, aby wiele jej obiektów posiadało swój typ. Przykładowo akcje są różnorakiego typu. Mają pewną część wspólną, ale też pola szczególne dla każdego typu. Go jest statycznie typowanym językiem, który nie posiada dziedziczenia ani tradycyjnego obiektowego polimorfizmu. Dlatego dokonano analizy jak w Go osiągnąć tę wielopostaciowość.

\textbf{Polimorfizm poprzez interfejsy}

Natywnym sposobem na polimorfizm w Go jest ten osiągany poprzez interfejsy. To zagadnienie najlepiej tłumaczy poniższy kod.


\begin{lstlisting}[language=go, caption={\emph{Przykład polimorfizmu poprzez interfejsy}}\label{lst:434}]
package main

import (
	"fmt"
)

type Forwarder interface {
	Forward() string
}

type NextElement struct {
	Name string
}

func (e *NextElement) Forward() string {
	return fmt.Sprintf("Forwarding to element: %s", e.Name)
}

type Destination struct {
	URL string
}

func (d *Destination) Forward() string {
	return fmt.Sprintf("Forwarding to destination: %s", d.URL)
}

func ProcessForwarder(f Forwarder) {
	fmt.Println(f.Forward())
}

func main() {
	element := &NextElement{Name: "Element1"}
	destination := &Destination{URL: "https://example.com"}

	ProcessForwarder(element)
	ProcessForwarder(destination)
}
\end{lstlisting}

\textbf{Jak to działa}

\begin{itemize}
    \item \textbf{Definicja interfejsu:}  
    \texttt{Forwarder} definiuje metodę \texttt{Forward()}, którą muszą zaimplementować określone typy.
    
    \item \textbf{Konkretne implementacje:}  
    \texttt{NextElement} oraz \texttt{Destination} implementują metodę \texttt{Forward()}.
    
    \item \textbf{Zastosowanie:}  
    Każdy typ spełniający interfejs \texttt{Forwarder} może być przekazywany do funkcji oczekujących obiektu \texttt{Forwarder}.
\end{itemize}

\textbf{Polimorfizm poprzez wskaźniki oraz pole dyskryminatora}

Kolejnym potężnym i idiomatycznym wzorcem w Go jest \textbf{polimorfizm w stylu Go z użyciem wskaźników}, gdzie struktura posiada opcjonalne pola wskaźnikowe, a pole \texttt{type} (znacznik) określa, które z tych pól jest istotne w czasie działania programu.

\begin{lstlisting}[language=go, caption={\emph{Przykład polimorfizmu poprzez wskaźniki oraz pole dyskryminatora}}\label{lst:435}]
type Next struct {
	// Type specifies the type of next loop-element
	Type string `json:"type"`
	// List of input keys (Data fields) that have to be forwarded
	Keys []string `json:"keys"`
	// One of the fields below is not null
	Element     *NextElement `json:"element,omitempty"`
	Destination *Destination `json:"destination,omitempty"`
}

type NextElement struct {
	Name string `json:"name"`
}

type Destination struct {
	URL string `json:"url"`
}

func (n *Next) Validate() error {
	if n.Type == "element" && n.Element == nil {
		return fmt.Errorf("Element must be set 
        for type 'element'")
	}
	if n.Type == "destination" && n.Destination == nil {
		return fmt.Errorf("Destination must be set 
        for type 'destination'")
	}
	if n.Element != nil && n.Destination != nil {
		return fmt.Errorf("Only one of Element 
        or Destination can be set")
	}
	return nil
}
\end{lstlisting}

\textbf{Jak to działa}

\textbf{Unia tagowana} to wzorzec, w którym pole \texttt{tag} określa, którą z kilku możliwych reprezentacji danych wykorzystuje dany obiekt. W Go jest to realizowane poprzez kombinację:

\begin{itemize}
    \item Pole dyskryminujące typ (np. \texttt{Type string}).
    \item Pola wskaźnikowe dla możliwych wariantów. Jeśli dane pole nie występuje w aktualnej reprezentacji obiektu, jego wartość jest po prostu \texttt{nil}.
    \item Podczas działania programu możemy zweryfikować, której reprezentacji danych używa obiekt, i odpowiednio na tej podstawie podjąć działanie.
\end{itemize}

Podczas działania programu możemy zweryfikować, której reprezentacji danych używa obiekt, i odpowiednio na tej podstawie podjąć działanie.

\textbf{Porównanie}

Porównanie zostało przedstawione w tabeli \ref{tab:431}

\begin{table}[h!]
\centering
\renewcommand{\arraystretch}{1.5} % Zwiększenie odstępów między wierszami
\begin{tabular}{|p{4cm}|p{5.5cm}|p{5.5cm}|}
\hline
\textbf{Funkcja}               & \textbf{Polimorfizm przez wskaźniki}                                   & \textbf{Polimorfizm przez interfejsy}                          \\ \hline
\textbf{Zachowanie w czasie działania} & Używa dyskryminatora (\texttt{Type}) i pól wskaźnikowych           & Używa implementacji metod do polimorfizmu                     \\ \hline
\textbf{Bezpieczeństwo typów}      & Wymaga jawnej walidacji                                              & Wymuszane podczas kompilacji za pomocą interfejsów            \\ \hline
\textbf{Serializacja}          & Bezproblemowa z JSON                                                 & Może wymagać niestandardowego \texttt{marshaling}             \\ \hline
\textbf{Rozszerzalność}        & Dodaj nowe pola wskaźnikowe i zaktualizuj enum \texttt{Type}          & Dodaj nowe typy implementujące interfejs                      \\ \hline
\textbf{Łatwość użycia}         & Prosta, ale wymaga ręcznej walidacji                                 & Czysta i idiomatyczna w Go                                    \\ \hline
\textbf{Wspólne zachowanie}    & Wymaga zewnętrznej logiki                                            & Enkapsulowane w metodach interfejsu                           \\ \hline
\end{tabular}
\caption{Porównanie polimorfizmu przez wskaźniki i przez interfejsy w Go.}\label{tab:431}
\end{table}

Polimorfizm przez wskaźniki oferuje przejrzystą reprezentację danych, która może być łatwo serializowana i deserializowana (np. do formatu JSON lub YAML). Dodatkowo wspiera włączenie pola \texttt{type} jako części modelu danych, dzięki czemu może być ono również przechowywane lub przesyłane. Z drugiej strony polimorfizm przez interfejsy jest natywny dla Go, bardziej przejrzysty i gwarantuje silne sprawdzanie typów podczas kompilacji.

\textbf{Podsumowanie:}

\begin{itemize}
    \item Polimorfizm przez wskaźniki jest preferowany w aplikacjach skoncentrowanych na danych.
    \item Polimorfizm przez interfejsy jest preferowany w aplikacjach skoncentrowanych na zachowaniach.
\end{itemize}

W \hyperlink{def:lupus}{\textbf{Lupus}} polimorfizm był potrzebny do reprezentacji różnych typów (odmian) niektórych \hyperlink{def:element}{\textbf{Obiektów LupN}}, takich jak \hyperlink{def:akcja}{\textbf{Actions}} lub \hyperlink{def:next}{\textbf{Next}}, dlatego wybrano polimorfizm poprzez wskaźniki oraz pole dyskryminatora.

\subsubsection{Dwa rodzaje workflow}

W \hyperlink{def:lupus}{\textbf{Lupus}} występują dwa rodzaje \textit{workflow}: \hyperlink{def:workflow-petli}{\textbf{Workflow Pętli}}, które definiuje przepływ pracy \hyperlink{def:element-lupus}{\textbf{Elementów Lupus}}, oraz \hyperlink{def:workflow-petli}{\textbf{Workflow Akcji}}, które definiuje przepływ \hyperlink{def:akcja}{\textbf{Akcji}} we wnętrzu pojedynczego elementu. Występują między nimi pewne różnice w możliwościach, wynikające ze sposobu implementacji komunikacji między ich węzłami.

\begin{itemize}
    \item Węzły w \hyperlink{def:workflow-petli}{\textbf{Workflow Pętli}} komunikują się ze sobą poprzez pobudzanie operatorów.
    \item Węzły w \hyperlink{def:workflow-petli}{\textbf{Workflow Akcji}} komunikują się poprzez pamięć RAM zaalokowaną przez pojedynczy operator \hyperlink{def:element-lupus}{\textbf{Elementu Lupus}}.
\end{itemize}

Różnice są widoczne w specyfikacji \hyperlink{def:lupn}{\textbf{LupN}} //TODO.

\subsubsection{Funkcje użytkownika}\label{sec:funkcje-uzytkownika}

Zgodnie z \hyperref[req:6]{Wymaganiem 6}, \hyperlink{def:element-lupus}{\textbf{Elementy Lupus}} nie wykonują \hyperlink{def:czesc-obliczeniowa}{\textbf{Części Obliczeniowej Logiki Pętli}}. Zamiast tego, jest ona delegowana do \hyperlink{def:element-zewnetrzny}{\textbf{Elementów Zewnętrznych}}.

Pojawiają się tutaj dwa aspekty:
\begin{itemize}
    \item Co w sytuacji, gdy ktoś potrzebuje wykonać małą i prostą operację na \hyperlink{def:dane}{\textbf{Danych}} (np. dodanie dwóch pól), a wdrażanie dedykowanego serwera HTTP jest nieekonomiczne?
    \item Każda platforma ramowa powinna być rozszerzalna. Powinniśmy zapewnić mechanizm umożliwiający rozszerzanie naszej platformy.
\end{itemize}

Z tych dwóch powodów wdrożono funkcję o nazwie „Definiowane przez użytkownika, wewnętrzne funkcje Go” (lub w skrócie – \hyperlink{def:funkcje-uzytkownika}{\textbf{Funkcje Użytkownika}}).

\hyperlink{def:uzytkownik}{\textbf{Użytkownik}} może definiować własne fragmenty kodu Go jako funkcje i wywoływać je jako jedną z \hyperlink{def:destynacja}{\textbf{Destynacji}} w akcji \hyperlink{def:akcja}{\textbf{Send}}.

W repozytorium kodu z projektem Kubebuilder znajduję plik o nazwie \texttt{user-functions.go} \footnote{\url{https://github.com/0x41gawor/lupus/blob/master/lupus/internal/controller/user-functions.go}}. To w nim \textbf{Użytkownik} może definiować swoje funkcje. Plik posiada już jedną przykładową funkcję.

\begin{lstlisting}[language=go, caption={\emph{Przykładowa funkcja użytkownika}}\label{lst:436}]
// Exemplary user-function. It just returns the input
func (UserFunctions) Echo(input interface{}) (interface{}, error) {
	return input, nil
}
\end{lstlisting}

Funkcja jako argument przyjmuje \texttt{interface{}}, który będzie reprezentował \textbf{pole danych}.\footnote{Pola lub pola, gdyż pod tym pojęciem może też się kryć użycie "*"}. Zwracanym typem jest również \texttt{interface{}}, który zostanie przez akcję send wpisany jako \textbf{pole danych}.

Nazwa \texttt{UserFunctions} jest strukturą, która gromadzi funkcje użytkownika.

\begin{lstlisting}[language=go, caption={\emph{Stukrutra składują funckje użytkowniak jako swoje metody.}}\label{lst:437}]
// UserFunctions struct for user-defined, internal functions
type UserFunctions struct{}
\end{lstlisting}

Powstaje teraz pytanie jak wywołać takową funkcję w notacji \hyperlink{def:lupn}{\textbf{LupN}}. Na szczęście funkcje w Go mogą być używane jako typy i przechowywane jako wartości. Dzięki temu mogą zostać zapisane jako wartości w mapie klucz-wartość, gdzie nazwy funkcje typu \texttt{string} pełnią rolę kluczy.


\begin{lstlisting}[language=go, caption={\emph{Mapa przechowująca funkcje użytkownika}}\label{lst:438}]
// A global map to store function references
var FunctionRegistry = map[string]func(input interface{}) (interface{}, error){}
\end{lstlisting}

Następnie, z pomocą biblioteki \texttt{reflect}\footnote{\url{https://pkg.go.dev/reflect}} można zaimplementować funkcję, która iteruje po metodach struktury \texttt{UserFunctions} i zapisuje je do powyższej mapy jako wartości, dla których kluczami są nazwy funkcji. 

\begin{lstlisting}[language=go, caption={\emph{Funkcja zapełniająca mapę funkcji użytkownika}}\label{lst:439}]
// RegisterFunctions dynamically registers user-defined functions
func RegisterFunctions(target interface{}) {
	t := reflect.TypeOf(target)
	v := reflect.ValueOf(target)

	for i := 0; i < t.NumMethod(); i++ {
		method := t.Method(i)

		// Ensure the method matches the required signature
		if method.Type.NumIn() == 2 && // Receiver + input
			method.Type.NumOut() == 2 && // Output + error
			method.Type.In(1).Kind() == reflect.Interface && // Input: interface{}
			method.Type.Out(0).Kind() == reflect.Interface && // Output: interface{}
			method.Type.Out(1).Implements(reflect.TypeOf((*error)(nil)).Elem()) { // Second output: error

			funcName := method.Name
			FunctionRegistry[funcName] = func(input interface{}) (interface{}, error) {
				// Call the user-defined function
				result := method.Func.Call([]reflect.Value{v, reflect.ValueOf(input)})

				// Handle result[1] (error) being nil
				var err error
				if !result[1].IsNil() {
					err = result[1].Interface().(error)
				}

				return result[0].Interface(), err
			}
		}
	}
}
\end{lstlisting}

Funkcja widoczna na listingu \ref{lst:439} jest wywoływana podczas inicjalizacji paczki kontrolera z strukturą \texttt{UserFunctions} jako argument.

\begin{lstlisting}[language=go, caption={\emph{Inicjaliza mapy}}\label{lst:4310}]
func init() {
    // Fill in the FunctionRegistry map with functions defined as a method of UserFunctions{}
	RegisterFunctions(UserFunctions{})
}
\end{lstlisting}


Od tego momentu funkcja może zostać wywołana po nazwie w funkcji obsługujące akcję Send w interpreterze Operatora Elementu Lupus.

\begin{lstlisting}[language=go, caption={\emph{Wywołąnie funkcji użytkownika w akcji send}}\label{lst:4311}]
func sendToGoFunc(funcName string, body interface{}) (interface{}, error) {
	if fn, exists := FunctionRegistry[funcName]; exists {
		return fn(body)
	} else {
		return nil, fmt.Errorf("no such UserFunction defined")
	}
}
\end{lstlisting}

W notacji \textbf{LupN} wystarczy zapis:

\begin{lstlisting}[language=sh, caption={\emph{Użycie funkcji użytkownika w LupN}}\label{lst:4312}]
  - name: "bounce"
    type: send
    send:
        inputKey: "field1"
        destination:
        type: gofunc
        gofunc:
            name: echo
        outputKey: "field2"
\end{lstlisting}
\clearpage
\section{Test platformy na emulatorze sieci 5G}\label{sec:5}

\subsection{Wstęp}

Niniejszy rozdział omawia przypadek użycia (ang. \textit{use case}) proponowanej architektury na emulatorze sieci 5G zbudowanym przy użyciu Open5GS oraz UERANSIM. Opisuje również sam emulator sieci 5G. Następnie, prezentuje przykładowy \hyperlink{def:problem-zarzadzania}{\textbf{Problem Zarządzania}} oraz zastosowanie \hyperlink{def:lupus}{\textbf{Lupus}} w celu jego rozwiązywania. Rozdział omawia proces pojedynczego \hyperlink{def:wdrozenie-lupus}{\textbf{Wdrożenia Lupus}}.

\subsection{Emulator sieci 5G}

\subsubsection{Open5GS}

Pierwszym komponentem emulatora jest otwartoźródłowe oprogramowanie Open5GS. Oficjalna strona projektu znajduje się tutaj: \url{https://open5gs.org/open5gs/}. Open5GS emuluje \textbf{Sieć Rdzeniową} (ang. \textit{core network}) czwartej oraz piątej generacji sieci standaryzowanych przez 3GPP. Architektura oprogramowania przedstawiona jest na rysunku \ref{fig:52-open5gs}.

\begin{figure}[!h]
    \centering \includegraphics[width=1\linewidth]{52-open5gs.png}
    \caption{Architektura Open5GS}\label{fig:52-open5gs}
\end{figure}

\subsubsection{UERANSIM}

Drugim komponentem emulatora jest otwartoźródłowe oprogramowanie UERANSIM. Oficjalna strona projektu: \url{https://github.com/aligungr/UERANSIM}. UERANSIM symuluje \textbf{Radiową Sieć Dostępową} (ang. \textit{Radio Access Network (RAN)}) wraz z \textbf{Terminalami Abonenta} (ang. \textit{User Equipment (UE)}). UERANSIM składa się z dwóch programów \texttt{nr-ue} oraz \texttt{nr-gnb}, które można uruchamiać w wielu instancjach. Ich rolę w architekturze sieci 5G przedstawiono na rysunku \ref{fig:52-ueransim} w czerwonej obwódce.

\begin{figure}[!h]
    \centering \includegraphics[width=1\linewidth]{52-ueransim.png}
    \caption{Architektura UERANSIM}\label{fig:52-ueransim}
\end{figure}

\subsubsection{Wdrożenie na Kubernetes}

Emulator sieci 5G złożony z Open5GS oraz UERANSIM wdrożono na klastrze Kubernetes według repozytorium: \url{https://github.com/niloysh/open5gs-k8s}. Przeprowadzenie kroków wskazanych w repozytorium skutkuje stanem podów w klastrze przedstawionym na rysunku \ref{fig:52-pody}.

\begin{figure}[!h]
    \centering \includegraphics[width=1\linewidth]{52-pody.png}
    \caption{Stan podów w klastrze}\label{fig:52-pody}
\end{figure}

Programy UERANSIM typu \texttt{nr-ue} będą uruchamiane poza klastrem. Tej decyzji dokonano z prostego powodu. System operacyjny hostujący klaster (Ubuntu 22.04 LTS) posiada dużo większe możliwości, jeśli chodzi o narzędzia generowania ruchu, niż okrojony system operacyjny użyty w podach UERANSIM.
\subsection{Wdrożenie Lupus}

Rozdział opisuje użycie \hyperlink{def:lupus}{\textbf{Lupus}} jako \hyperlink{def:system-sterowania}{\textbf{Systemu Sterowania}} dla \hyperlink{def:problem-zarzadzania}{\textbf{Problemu Zarządzania}} opisanego w następnym podrozdziale dla emulatora sieci 5G, który przyjmie z kolei rolę \hyperlink{def:system-zarzadzany}{\textbf{Systemu Zarządzanego}}. 

\subsubsection{Problem Zarządzania}

Celem jest utrzymanie żądanych wartości zasobów dla funkcji płaszczyzny użytkownika (ang. \textit{User Plane Function, UPF}) na optymalnym poziomie. Optymalność oznacza zapewnienie wystarczających zasobów do obsługi obciążenia, ale jednocześnie unikanie ich nadmiernego przydziału, co pozwala na ograniczenie niepotrzebnych kosztów związanych z wynajmem zasobów chmurowych.

W ramach wdrożenia Open5GS, w tym dla komponentu UPF, istnieje kilka instancji uruchomieniowych (ang. \textit{Deployments}), jak pokazano na rysunku \ref{fig:53-depy}.

\begin{figure}[!h]
    \centering \includegraphics[width=1\linewidth]{53-depy.png}
    \caption{Wdrożenia Open5GS, w tym instancje UPF}\label{fig:53-depy}
\end{figure}

Każdy \textit{pod} w klastrze Kubernetes posiada zdefiniowane wartości dotyczące przydziału zasobów:

\begin{itemize}
    \item \textbf{\texttt{requests}} – określa ilość zasobów CPU oraz pamięci operacyjnej (RAM), jaką dany pod żąda od środowiska chmurowego. Kubernetes zapewnia, że \textit{węzeł} posiada wystarczające zasoby, aby spełnić te wymagania przed przydzieleniem poda do \textit{węzła}.
    \item \textbf{\texttt{limits}} – maksymalna ilość zasobów, jaką pod może zużyć. Przekroczenie tych wartości skutkuje zakończeniem procesu (np. jego ubiciem w celu ochrony środowiska).
\end{itemize}

Dodatkowo możliwe jest monitorowanie rzeczywistego (aktualnego, chwilowego) wykorzystania procesora i pamięci operacyjnej.


\textbf{Mechanizmy dostępne w Kubernetes}

Kubernetes umożliwia:
\begin{itemize}
    \item monitorowanie bieżącego zużycia CPU oraz pamięci dla konkretnego poda,
    \item monitorowanie zdefiniowanych wartości \texttt{requests} i \texttt{limits} dla konkretnego poda,
    \item aktualizację wartości \texttt{requests} i \texttt{limits} dla konkretnego poda.
\end{itemize}

Wdrożenie komponentu UPF w repozytorium \href{https://github.com/niloysh/open5gs-k8s}{open5gs-k8s} wykorzystuje następujące wartości zasobów:

\begin{lstlisting}[language=sh, caption=Konfiguracja zasobów dla UPF w Open5GS]
resources:
  requests:
    memory: "256Mi"
    cpu: "200m"
  limits:
    memory: "512Mi"
    cpu: "500m"
\end{lstlisting}

Niniejsza konfiguracja jest traktowana jako \texttt{punkt operacyjny (punkt pracy)}, który powinien być wystarczający w normalnych warunkach (np. przez 80\% czasu). Jednak w sytuacjach, gdy pojedynczy pod UPF sporadycznie przekracza te wartości, konieczne jest zwiększenie dostępnych zasobów oraz proporcjonalne podniesienie limitów.

Zakłada się, że każdorazowe przekroczenie wartości operacyjnych CPU lub pamięci skutkuje przydzieleniem dodatkowych 20\% zasobów dla \texttt{requests} oraz ustawieniem \texttt{limits} na poziomie dwukrotności nowej wartości \texttt{requests}.

\textbf{Przykładowa adaptacja zasobów}

\begin{itemize}
    \item Gdy zużycie CPU osiąga wartość \texttt{270m}:
    \begin{itemize}
        \item \texttt{requests} zostaje zwiększone do \( 270m \times 1.2 = 324m \),
        \item \texttt{limits} zostaje zwiększone do \( 324m \times 2 = 648m \).
    \end{itemize}
    \item Analogiczne zasady dotyczą przydziału pamięci operacyjnej.
\end{itemize}

Jeśli rzeczywiste wykorzystanie zasobów jest znacząco niższe od wartości operacyjnej, nie jest podejmowana żadna akcja korygująca. Nadmierne skalowanie w dół skutkowałoby zbyt częstymi restartami podów, co w konsekwencji generowałoby większe koszty niż wynikające z niedostatecznego wykorzystania zasobów.

\subsubsection{Implementacja Agentów Translacji}

\textbf{Agent Ingress}

\hyperlink{def:agent-ingress}{\textbf{Agent Ingress}} okresowo pobiera aktualne zużycie zasobów oraz zdefiniowane wartości \texttt{requests} oraz \texttt{limits} dla podów UPF w obu wdrożeniach (ang. \textit{deployments}). Następnie, zgodnie ze specyfikacją \hyperlink{def:interfejs-lupin}{\textbf{Interfejsu Lupin}}, będzie aktualizował pole \texttt{Input} statusu obiektu API \hyperlink{def:element-ingres}{\textbf{Elementu Ingress}} obiektem JSON, jak pokazano na listingu \ref{lst:521} z przykładowymi wartościami. Jednocześnie taki obiekt JSON będzie stanowił początkową postać \hyperlink{def:dane}{\textbf{Danych}}.

\begin{lstlisting}[language=sh, caption={\emph{}}\label{lst:521}]
    {
        "open5gs-upf1: {
            "actual": {
                "cpu": "1m",
                "memory": "18Mi"
            },
            "requests": {
                "cpu": "100m",
                "memory": "128Mi"
            },
            "limits": {
                "cpu": "250m",
                "memory": "256Mi"
            }
        },
        "open5gs-upf2: {
            "actual": {
                "cpu": "1m",
                "memory": "25Mi"
            },
            "requests": {
                "cpu": "200m",
                "memory": "256Mi"
            },
            "limits": {
                "cpu": "500m",
                "memory": "512Mi"
            }
        }
    }
\end{lstlisting}

Kod agenta Ingress znajduje się w \hyperlink{appengix:7}{Załączniku 7}.

\textbf{Agent Egress}

Agent Egress udostępnia endpoint HTTP, który przyjmuje dane w postaci przedstawionej na rysunku \ref{fig:53-egress}

\begin{figure}[!h]
    \centering \includegraphics[width=1\linewidth]{53-egress.png}
    \caption{Endpoint Egress Agent przyjmujący finalne dane}\label{fig:53-egress}
\end{figure}

Następnie dla wdrożenia (ang. \textit{deployment}) zadanego polem \texttt{name} zmodyfikuje zdefiniowane wartości \texttt{requests} oraz \texttt{limits} według pola \texttt{spec}. Opcjonalne jest pole \texttt{interval}, które określa z jakim interwałem czasowym (w sekundach) raportować aktualne zużycie zasobów. Agent Egress odpowiednio wysteruje tym polem Agenta Ingress. Kod Agenta Egress znajduje się w \hyperlink{appendix:8}{Załączniku 8}.

\subsubsection{Planowanie Logiki Pętli}

Analizując dane wejściowe z listingu \ref{lst:521} można wyróżnić cztery możliwe stany pojedynczego wdrożenia UPF:
\begin{itemize}
    \item \textbf{NORMAL} – wartości \texttt{requests} i \texttt{limits} ustawione na wartości domyślne, rzeczywiste zużycie (\texttt{actual}) pozostaje poniżej wartości domyślnych.
    \item \textbf{NORMAL\_TO\_CRITICAL} – wartości \texttt{requests} i \texttt{limits} nadal są domyślne, ale rzeczywiste zużycie (\texttt{actual}) przekracza wartości domyślne.
    \item \textbf{CRITICAL} – wartości \texttt{requests} i \texttt{limits} przekraczają wartości domyślne, a rzeczywiste zużycie (\texttt{actual}) również pozostaje powyżej wartości domyślnych.
    \item \textbf{CRITICAL\_TO\_NORMAL} – wartości \texttt{requests} i \texttt{limits} nadal przekraczają wartości domyślne, ale rzeczywiste zużycie (\texttt{actual}) spadło poniżej wartości domyślnych.
\end{itemize}

W zależności od stanu systemu należy podjąć następujące działania:
\begin{itemize}
    \item \textbf{NORMAL} – brak działań, system działa poprawnie.
    \item \textbf{NORMAL\_TO\_CRITICAL} – dostosowanie wartości \texttt{requests} i \texttt{limits}, ustawienie interwału obserwacji na wysoki (\texttt{HIGH}).
    \item \textbf{CRITICAL} – dostosowanie wartości \texttt{requests} i \texttt{limits} do aktualnego zapotrzebowania.
    \item \textbf{CRITICAL\_TO\_NORMAL} – przywrócenie wartości \texttt{requests} i \texttt{limits} do wartości domyślnych, ustawienie interwału obserwacji na niski (\texttt{LOW}).
\end{itemize}

W związku z powyższym opisem pętla powinna w każdej iteracji określi punkt pracy, czyli stan w jakim znajduje się UPF. Jeśli jest to stan \texttt{NORMAL}, nie ma potrzeby aby wykonywać jakiekolwiek akcje sterujące. W przeciwnym wypadku, należy "dostroić" UPF odpowiednimi wartościami \texttt{requests} oraz \texttt{limits}. Jeśli UPF jest lub zbliża się do stanu krytycznego należy je ustawić z odpowiednim zapasem, zaś jeżeli UPF wraca już do stanu normalnego należ przywrócić im wartości domyślne. Na sam koniec, jeżeli mamy jeden ze stanów przejściowych (\texttt{NORMAL\_TO\_CRITICAL} bądź \texttt{CRITICAL\_TO\_NORMAL}) należy zmienić interwał obserwacji odpowiednio na wysoki lub niski.

Z tego opisu widzimy, że \hyperlink{def:czesc-obliczeniowa}{\textbf{Część obliczeniowa}} logiki pętli składa się z trzech części:
\begin{itemize}
    \item Ustalenie punktu pracy 
    \item Obliczenie wartości \texttt{requests} oraz \texttt{limits}
    \item Ustalenie wartości interwału na podstawie punkty pracy
\end{itemize}

Dlatego też powstaną trzy \hyperlink{def:element-zewnetrzny}{\textbf{Elementy Zewnętrzne}}:
\begin{itemize}
    \item \textbf{POINT} – akceptuje \texttt{"*"} (wszystkie \hyperlink{def:pole-danych}{\textbf{Pola Danych}}), zwraca \texttt{"point"}.
    \item \textbf{SPEC} – akceptuje \texttt{"actual"}, zwraca \texttt{"spec"} zawierające wartości \texttt{"requests"} oraz \texttt{"limits"}.
    \item \textbf{INTERVAL} – akceptuje \texttt{"point"}, zwraca \texttt{"interval"}.
\end{itemize}

A \hyperlink{def:workflow-petli}{\textbf{Workflow Akcji}} dla \hyperlink{def:element-lupus}{\textbf{Elementu Lupus}} odpowiedzialnego za rekoncyliację jednego wdrożenia będzie wyglądało jak na rysunku \ref{fig:53-workflow-akcji}.

\begin{figure}[!h]
    \centering \includegraphics[width=1\linewidth]{53-workflow-akcji.png}
    \caption{Workflow Akcji Elementu Lupus}\label{fig:53-workflow-akcji}
\end{figure}

\subsubsection{Przygotowanie Elementów Zewnętrznych}

Każdy z \hyperlink{def:element-zewnetrzny}{\textbf{Elementów Zewnętrznych}} przygotowano jako aplikację Python emulującą serwer \hyperlink{def:opa}{\textbf{Open Policy Agent}}.

Implementacje można znaleźć w \hyperlink{appendix:9}{Załączniku 9}.

\subsubsection{Wyrażenie workflow pętli w notacji LupN}

\hyperlink{def:workflow-petli}{\textbf{Workflow Pętli}} pokazano na rysunku \ref{fig:53-workflow-petli}.

\begin{figure}[!h]
    \centering \includegraphics[width=1\linewidth]{53-workflow-petli.png}
    \caption{Workflow Pętli}\label{fig:53-workflow-petli}
\end{figure}

Element \texttt{demux} służy do rozdzielenia \hyperlink{def:dane}{\textbf{Danych}} na dwie części. Każde wdrożenie UPF ma swój dedykowany element \texttt{upf1} lub \texttt{upf2}, który obsługuje je według \hyperlink{def:workflow-petli}{\textbf{Workflow Akcji}} przedstawionego w poprzednim podrozdziale. Elementy \texttt{upf1} oraz \texttt{upf2} niezależnie pobudzają endpoint \texttt{/api/data} \hyperlink{def:agent-egress}{\textbf{Agenta Egress}}.

Kod \hyperlink{def:lupn}{\textbf{LupN}} znajduje się w \hyperlink{appendix:10}{Załączniku 10}. Ostatnim krokiem w celu uruchomienia pętli jest zaaplikowanie \hyperlink{def:plik-lupn}{\textbf{Pliku LupN}}.

\subsubsection{Prezentacja działania jednej iteracji pętli}

Iterację rozpoczyna \hyperlink{def:agent-ingress}{\textbf{Agent Ingress}}, który pobiera \hyperlink{def:dane}{\textbf{Dane}} z Kubernetes (rys. \ref{fig:53-scr-1}). Następnie aktualizuje on \texttt{Status} obiektu API \texttt{lola-demux} (rys. \ref{fig:53-scr-2}), który wykonuje swoje \hyperlink{def:workflow-petli}{\textbf{Workflow Akcji}} (rys. \ref{fig:53-scr-3}), skutkujące aktualizacją \texttt{Statusu} obiektów API \texttt{upf1} i \texttt{upf2} (rys. \ref{fig:53-scr-6}). To skutkuje wywołaniem \texttt{Operatora} dla tych obiektów. Operator wykonuje dla obu \hyperlink{def:workflow-petli}{\textbf{Workflow Akcji}} (rys. \ref{fig:53-scr-4}). Na koniec oba elementy wywołują endpoint \texttt{api/data} serwowany przez \hyperlink{def:agent-egress}{\textbf{Agenta Egress}} (rys. \ref{fig:53-scr-5}).


\begin{figure}[!h]
    \centering \includegraphics[width=1\linewidth]{53-scr-1.png}
    \caption{Logi Agenta Ingress}\label{fig:53-scr-1}
\end{figure}

\begin{figure}[!h]
    \centering \includegraphics[width=0.5\linewidth]{53-scr-2.png}
    \caption{Status obiektu API "lola-demux"}\label{fig:53-scr-2}
\end{figure}

\begin{figure}[!h]
    \centering \includegraphics[width=1\linewidth]{53-scr-3.png}
    \caption{Logi operatora element rekoncylującego obiekt API "lola-demux"}\label{fig:53-scr-3}
\end{figure}

\begin{figure}[!h]
    \centering \includegraphics[width=1\linewidth]{53-scr-6.png}
    \caption{Statusy obiektów API "lola-upf1" oraz "lola-upf2"}\label{fig:53-scr-6}
\end{figure}

\begin{figure}[!h]
    \centering \includegraphics[width=1\linewidth]{53-scr-4.png}
    \caption{Logi operatora element rekoncylującego obiekt API "lola-upf1"}\label{fig:53-scr-4}
\end{figure}

\begin{figure}[!h]
    \centering \includegraphics[width=1\linewidth]{53-scr-5.png}
    \caption{Logi Agenta Egress}\label{fig:53-scr-5}
\end{figure}

Gdy pobudzimy funkcję UPF do większej pracy ruchem z UE, widzimy raportowane większe zużycie zasobów przez odpowiednie wdrożenie (rys. \ref{fig:53-scr-7}). Aktualne zużycie przekraczające zdefiniowany punkt pracy powoduje otrzymanie od elementu zewnętrznego \texttt{opa-point} punkt \texttt{NORMAL\_TO\_CRITICAL}, co z kolei wpływa na workflow akcji (rys. \ref{fig:53-scr-8}). W kolejnych iteracjach w logach Agenta Ingress widzimy błąd spowodowany tym, ze Pod UPF1 chwilowo nie istnieje, stało się tak dlatego iż Agent Egress w poprzedniej iteracji zmienił jego wartość \texttt{requests} oraz \texttt{limits}, co poskutkowało restartem Poda. Po restarcie widzimy iż Agent Ingress raportuje nowe wartości \texttt{requests} i \texttt{limits} dla Poda UPF1.

\begin{figure}[!h]
    \centering \includegraphics[width=1\linewidth]{53-scr-7.png}
    \caption{Logi Agenta Ingress przy pobudzeniu ruchem}\label{fig:53-scr-7}
\end{figure}

\begin{figure}[!h]
    \centering \includegraphics[width=1\linewidth]{53-scr-8.png}
    \caption{Logi z workflow akcji elementu "upf1" przy pobudzeniu ruchem}\label{fig:53-scr-8}
\end{figure}

Dokmuntacja elektroniczna całego procesu wraz z instrukcją uruchomienia na własnym środowisku znaduje się pod linkiem: \url{https://github.com/0x41gawor/lupus/tree/master/examples/open5gs}.


%---------------
% Literatura
%---------------
\cleardoublepage 
\printbibliography



%--------------------------------------
% Spisy
%--------------------------------------
\newpage
\pagestyle{plain}
\vspace{0.8cm}
\acronymlist
\acronym{5GC}{5G Core Network}
\acronym{5GS}{5G System}

\listoffigurestoc    
\vspace{1cm}         
\listoftablestoc     
\vspace{1cm}         
\listofappendicestoc 

%-------------
% Załączniki
%-------------
\setcounter{tocdepth}{1}
% Obrazki i tabele w załącznikach nie trafiają do spisów
\captionsetup[figure]{list=no}
\captionsetup[table]{list=no}

\appendix{Definicje Lupus}\label{appendix:1}

Specyfikacja w formie elektronicznej znajduje się pod linkiem: \url{https://github.com/0x41gawor/lupus/blob/master/docs/defs.md}.
\appendix{Instalacja Lupus}\label{appendix:2}

Specyfikacja w formie elektronicznej znajduje się pod linkiem: \url{https://github.com/0x41gawor/lupus/blob/master/docs/installation.md}.

\subsection{Przedsłowie}

Instalacja \hyperlink{def:lupus}{\textbf{Lupus}} wymaga umiejętności technicznych oraz podstawowej znajomości operacyjnej Kubernetes.

\hyperlink{def:lupus}{\textbf{Lupus}} jest zaimplementowany jako projekt Kubebuilder\footnote{\url{https://book.kubebuilder.io}}. Zalecanym sposobem instalacji \hyperlink{def:lupus}{\textbf{Lupus}} jest sklonowanie tego repozytorium i przyjęcie roli dewelopera tego projektu.

Nie istnieje coś takiego jak instalacja \hyperlink{def:lupus}{\textbf{Lupus}} (np. w systemie operacyjnym). Można zainstalować \hyperlink{def:zasoby-wlasne}{\textbf{Zasoby Własne}} dla \hyperlink{def:element-lupus}{\textbf{Elementów Lupus}} w klastrze Kubernetes i uruchomić dla nich \hyperlink{def:operator-zasobu-element}{\textbf{Kontrolery}}. Niniejszy załącznik opisuje właśnie taki proces.

\subsection{Wymagania wstępne}

\hyperlink{def:uzytkownik}{\textbf{Użytkownik}} musi posiadać działający klaster Kubernetes. Może to być Minikube\footnote{\url{https://minikube.sigs.k8s.io/docs/}}, zainstalowany silnik kontenerów (ang. \textit{container engine}) (np. Docker\footnote{\url{https://docs.docker.com}}) oraz język Go\footnote{\url{https://go.dev}}.

\subsubsection{Instalacja Kubebuilder}

Instrukcja dostępna pod adresem: \url{https://book.kubebuilder.io/quick-start}.

\subsection{Klonowanie repozytorium}

\begin{lstlisting}[language=bash, caption={Klonowanie repozytorium}]
git clone https://github.com/0x41gawor/lupus
cd lupus
\end{lstlisting}

\subsection{Instalacja CRD w klastrze}

To polecenie zastosuje \hyperlink{def:crd}{\textbf{CRD}} (pl. Definicje \hyperlink{def:zasoby-wlasne}{\textbf{Zasobów Własnych}}) dla \hyperlink{def:master}{\textbf{Master}} i \hyperlink{def:element}{\textbf{Element}}, umożliwiając ich użycie.

\begin{lstlisting}[language=bash, caption={Instalacja CRD}]
make install
\end{lstlisting}

\subsection{Uruchomienie kontrolerów }

To polecenie uruchomi \textit{kontrolery} dla \textit{zasobów własnych} \hyperlink{def:master}{\textbf{master}} i \hyperlink{def:element}{\textbf{element}}.

\begin{lstlisting}[language=bash, caption={Uruchomienie kontrolerów}]
make run
\end{lstlisting}

Istnieje możliwość uruchomienia kontrolerów jako pody w klastrze Kubernetes. W tym celu użytkownik jest zaproszony do bliższego zapoznania się z platformą Kubebuilder. Dopóki \hyperlink{def:uzytkownik}{\textbf{Użytkownik}} jest pewien, że nie będzie dopisywał \hyperlink{def:funkcje-uzytkownika}{\textbf{Funkcji Użytkownika}} nie jest to zalecane podejście. 



\appendix{Specyfikacja notacji LupN}\label{appendix:3}\hypertarget{appendix:3}{}

Specyfikacja w formie elektronicznej znajduje się pod linkiem: \url{https://github.com/0x41gawor/lupus/blob/master/docs/spec/lupn.md}.


\hyperlink{def:lupn}{\textbf{LupN}} (od ang. \textit{loop} oraz \textit{Notation}) to język/notacja służąca do wyrażania \hyperlink{def:workflow-petli}{\textbf{Workflow Pętli}}. Nie zawiera opisu \hyperlink{def:czesc-obliczeniowa}{\textbf{Części Obliczeniowej}} \hyperlink{def:logika-petli}{\textbf{Logiki Pętli}}. \hyperlink{def:czesc-obliczeniowa}{\textbf{Część Obliczeniowa}} jest określona poza \hyperlink{def:lupus}{\textbf{Lupus}}, w \hyperlink{def:element-zewnetrzny}{\textbf{Elementach Zewnętrznych}}.

\texttt{LupN} specyfikuje:
\begin{itemize}
    \item \hyperlink{def:workflow-petli}{\textbf{Workflow Pętli}}, czyli workflow \hyperlink{def:element-lupus}{\textbf{Elementów Lupus}},
    \item odniesienia do \hyperlink{def:element-zewnetrzny}{\textbf{Elementów Zewnętrznych}}, wyrażone jako \hyperlink{def:destynacja}{\textbf{Destynacje}},
    \item \hyperlink{def:workflow-petli}{\textbf{Workflow Akcji}} w ramach \hyperlink{def:element-lupus}{\textbf{Elementu Lupus}},
    \item odniesienie (lub odniesienia) do \hyperlink{def:agent-egress}{\textbf{Agenta Egress}} jako \hyperlink{def:destynacja}{\textbf{Destynacja}}.
\end{itemize}

Można zauważyć iż, \texttt{LupN} wyraża \textbf{workflow} na dwóch poziomach: globalnym (czyli \hyperlink{def:workflow-petli}{\textbf{Workflow Elementów Lupus}}) oraz wewnątrz \hyperlink{def:element-lupus}{\textbf{Elementu Lupus}} (czyli \hyperlink{def:workflow-petli}{\textbf{Workflow Akcji}}). Możliwości obu poziomów są do siebie zbliżone, ale ostatecznie różne. Ten załącznik omówi również tę kwestię.


Z punktu widzenia implementacji \hyperlink{def:plik-lupn}{\textbf{Plik LupN}} to w rzeczywistości \textit{YAML manifest file} dla \hyperlink{def:zasoby-wlasne}{\textit{Zasobu Własnego}} \hyperlink{def:master}{\textbf{Master}}. Po zaaplikowaniu (ang. \textit{apply}), \hyperlink{def:operator-zasobu-master}{\textbf{Operator Zasobu Master}} uruchamia \hyperlink{def:element-lupus}{\textbf{Elementy Lupus}}, które realizują wyrażone \hyperlink{def:workflow-petli}{\textbf{Workflow Pętli}}.

\texttt{LupN} wyraża \hyperlink{def:workflow-petli}{\textbf{Workflow Pętli}} poprzez specyfikację różnych obiektów w notacji YAML. Obiekty te nazwano \hyperlink{def:obiekt-lupn}{\textbf{Obiektami LupN}}. \hyperref[appendix:3]{Załącznik 3} specyfikuje te obiekty oraz relacje między nimi. Wskazuje również, co oznacza użycie każdego z nich w kontekście \hyperlink{def:workflow-petli}{\textbf{Workflow Pętli}} i jak \hyperlink{def:operator-zasobu-element}{\textbf{Operator Zasobu Element}} interpretuje je w czasie działania.

Obiekty YAML w \hyperlink{def:plik-lupn}{\textbf{Pliku LupN}} są pochodnymi struktur Go, dlatego \hyperlink{def:obiekt-lupn}{\textbf{Obiekty LupN}} możemy opisać na ich podstawie. 

Wymagane jest wcześniejsze zapoznanie się z formatem YAML. Załącznik nie obejmuje translacji dokonywanej przez Kubernetes między strukturami Go a reprezentacjami obiektów YAML. Serializacja jest wykonywana przez \texttt{controller-gen} i opisana w \textit{Kubebuilder Book}. Tłumaczenie to można łatwo zaobserwować i nauczyć się, analizując przykłady zamieszczone w repozytorium projektu: \url{https://github.com/0x41gawor/lupus/tree/master/examples}. Dodatkowo, w \hyperlink{appendix:11}{Załączniku 11} zamieszczono krótkie komentarze pliku LupN wykorzystywanego w teście platformy opisanym w rozdziale \ref{sec:5}.

\subsection{Możliwości LupN}

Ze względu na różnice implementacyjne węzłów omówione w podrozdziale \ref{sec:dwa-rodzaje-workflow}, możliwości \hyperlink{def:workflow-petli}{Workflow Pętli} oraz \hyperlink{def:workflow-akcji}{Workflow Akcji} różnią się od siebie. Sekwencja wykonawcza elementów jest definiowana poprzez obiekty \texttt{next}. Każdy element definiuje listę następnych elementów. Zazwyczaj jest to jeden element. Możliwe są rozgałęzienia (ang. \textit{forks}) czyli lista z większą ilością elementów, ale wtedy zostaje wywołane wiele elementów niezależnie. Nie ma możliwości na zsynchronizowane powrotne złączenie przepływu danych. W przypadku akcji stosowany jest ten sam mechanizm opierający się na definiowaniu następnej akcji, z tym, że tutaj może być ona tylko jedna. Flow akcji interpretowane jest bowiem przez kontroler elementu, który procesuje je po jednej na raz. Z tego powodu możliwe jest sterowanie przepływem (ang. \textit{flow control}), które zostało zaimplementowane jako specjalny typ akcji - \texttt{switch}. 

\subsection{Specyfikacja}

\textbf{Plik LupN} posiada 4 główne pola (ang. \textit{top-level fields}).

\begin{lstlisting}[language=bash, caption={Główne pola pliku LupN}\label{lst:a31}]
apiVersion: lupus.gawor.io/v1
kind: Master
metadata:
  labels:
    app.kubernetes.io/name: lupus
    app.kubernetes.io/managed-by: kustomize
  name: lola
spec:
	<lupn-objects>
\end{lstlisting}

Każde z nich musi być ustawione jak w \ref{lst:a31} oprócz \texttt{metadata.name}, to pole odróżnia instancje pętli między sobą w obrębie klastra Kubernetes.

Notacja LupN rozpoczyna się od pola \texttt{spec}. Każdy obiekt Lupn zostanie opisany poprzez swoją definicję w Go.

\subsubsection{Drzewo obiektów LupN}

Podczas przeglądania specyfikacji \hyperlink{def:obiekt-lupn}{\textbf{obiektów LupN}} pomocne będzie śledzenie aktualnej pozycji w drzewie zależności obiektów. 

\begin{figure}[!h]
    \centering \includegraphics[width=1\linewidth]{a3-tree.png}
    \caption{Drzewko zależności obiektów LupN. Źródło: Opracowanie własne.}\label{fig:a3-tree}
\end{figure}

\subsubsection{MasterSpec}
\begin{lstlisting}[language=go, caption={MasterSpec}\label{lst:a32}]
// MasterSpec defines the desired state of Master
type MasterSpec struct {
	// Name of the Master CR (indicating the name of the loop)
	Name string `json:"name"`
	// Elements is a list of Lupus-Elements
	Elements []*ElementSpec `json:"elements"`
}
\end{lstlisting}
Każdy element na liście \texttt{Elements} spowoduje, że \textbf{Operator Zasobu Master} stworzy obiekt API typu \textbf{Lupus Element}. 

\subsubsection{ElementSpec}
\begin{lstlisting}[language=go, caption={MasterSpec}\label{lst:a32}, basicstyle=\ttfamily\tiny]
// ElementSpec defines the desired state of Element
type ElementSpec struct {
	// Name is the name of the element, its distinct from Kubernetes API Object name, 
    // but rather serves ease of managemenet aspect for loop-designer
	Name string `json:"name"`
	// Descr is the description of the lupus-element, same as Name it serves as 
    // the ease of management aspect for loop-designer
	Descr string `json:"descr"`
	// Actions is a list of Actions that lupus-element has to perform
	Actions []Action `json:"actions,omitempty"`
	// Next is a list of next objects (can be lupus-element or external-element) 
    // to which send the final-data
	Next []Next `json:"next,omitempty"`
	// Name of master element (used as prefix for lupus-element name)
	Master string `json:"master,omitempty"`
}
\end{lstlisting}

\subsubsection{Next}
\begin{lstlisting}[language=go, caption={Next}\label{lst:next}, basicstyle=\ttfamily\tiny]
// Next specifies the next loop-element in a loop workflow, 
// it may be either lupus-element or reference to an external-element
// It allows to forward the whole final-data, but also parts of it
type Next struct {
	// Type specifies the type of next loop-element, lupus-element (element) 
    // or external-element (destination)
	Type string `json:"type" kubebuilder:"validation:Enum=element,destination"`
	// List of input keys (Data fields) that have to be forwarded
	// Pass array with single element '*' to forward the whole input
	Keys []string `json:"keys"`
	// One of the fields below is not null
	Element     *NextElement `json:"element,omitempty" kubebuilder:"validation:Optional"`
	Destination *Destination `json:"destination,omitempty" kubebuilder:"validation:Optional"`
}
\end{lstlisting}

\subsubsection{NextElement}
\begin{lstlisting}[language=go, caption={NextElement}\label{lst:nextelement}]
// NextElement indicates the next loop-element 
// in loop-workflow of type lupus-element
type NextElement struct {
	// Name is the lupus-name of lupus-element 
	// (the one specified in Element struct)
	Name string `json:"name"`
}
\end{lstlisting}

\subsubsection{Destination}
\begin{lstlisting}[language=go, caption={Destination}\label{lst:destination}, basicstyle=\ttfamily\tiny]
// Destination represents an external-element
// It holds all the info needed to make a call to an external-element
// It supports calls to HTTP server, Open Policy Agent or user-functions
type Destination struct {
	// Type specifies if the external element is: a HTTP server in general, 
    // a special kind of HTTP server like Open Policy Agent or internal, a user-function
	Type string `json:"type" kubebuilder:"validation:Enum=http;opa;gofunc"`
	// One of these fields is not null depending on a Type
	HTTP   *HTTPDestination   `json:"http,omitempty" kubebuilder:"validation:Optional"`
	Opa    *OpaDestination    `json:"opa,omitempty" kubebuilder:"validation:Optional"`
	GoFunc *GoFuncDestination `json:"gofunc,omitempty" kubebuilder:"validation:Optional"`
}
\end{lstlisting}

\subsubsection{HTTPDestination}
\begin{lstlisting}[language=go, caption={HTTPDestination}\label{lst:httpdestination}]
// HTTPDestination defines fields specific to a HTTP type
// This is information needed to make a HTTP request
type HTTPDestination struct {
	// Path specifies HTTP URI
	Path string `json:"path"`
	// Method specifies HTTP method
	Method string `json:"method"`
}
\end{lstlisting}

\subsubsection{OpaDestination}
\begin{lstlisting}[language=go, caption={OpaDestination}\label{lst:opadestination}]
// OpaDestination defines fields specific to Open Policy Agent type
// This is information needed to make an Open Policy Agent request
// Call to Opa is actually a special type of HTTP call
type OpaDestination struct {
	// Path specifies HTTP URI, since method is known
	Path string `json:"path"`
}
\end{lstlisting}

\subsubsection{GoFuncDestination}
\begin{lstlisting}[language=go, caption={GoFuncDestination}\label{lst:gofuncdestination}]
// GoFuncDestination defines fields specific to GoFunc type
// This is information needed to call an user-function
type GoFuncDestination struct {
	// Name specifies the name of the function
	Name string `json:"name"`
}
\end{lstlisting}

\subsubsection{Action}
\begin{lstlisting}[language=go, caption={Action}\label{lst:action}, basicstyle=\ttfamily\tiny]
// Action represents operation that is performed on Data
// Action is used in Element spec. Element has a list of Actions 
// and executes them in a workflow manner
// In general, each action has an input and output keys that define 
// which Data fields it has to work on
// Each action indicates the name of the next Action in Action Chain
// There is special type - Switch. Actually, it does not perform any operation on Data, 
// but rather controls the flow of Actions chain
type Action struct {
	// Name of the Action, it is for designer to ease the management of the Loop
	Name string `json:"name"`
	// Type of Action
	Type string `json:"type" kubebuilder:"validation:Enum=send,nest,remove,rename,duplicate,print,insert,switch"`
	// One of these fields is not null depending on a Type.
	Send      *SendAction      `json:"send,omitempty" kubebuilder:"validation:Optional"`
	Nest      *NestAction      `json:"nest,omitempty" kubebuilder:"validation:Optional"`
	Remove    *RemoveAction    `json:"remove,omitempty" kubebuilder:"validation:Optional"`
	Rename    *RenameAction    `json:"rename,omitempty" kubebuilder:"validation:Optional"`
	Duplicate *DuplicateAction `json:"duplicate,omitempty" kubebuilder:"validation:Optional"`
	Print     *PrintAction     `json:"print,omitempty" kubebuilder:"validation:Optional"`
	Insert    *InsertAction    `json:"insert,omitempty" kubebuilder:"validation:Optional"`
	Switch    *Switch          `json:"switch,omitempty" kubebuilder:"validation:Optional"`
	// Next is the name of the next action to execute, in the case of Switch-type action it stands as a default branch
	Next string `json:"next"`
}
\end{lstlisting}

Pole \texttt{Next}, oprócz nazw akcji, może przyjąć jedną z dwóch zdefiniowanych wartości. Wartość \texttt{final} oznacza, że postać \hyperlink{def:dane}{\textbf{Danych}} po tej akcji jest już w swojej \hyperlink{def:finalne-dane}{\textbf{Finalnej Postaci}} i musi zostać przekazana do następnego \hyperlink{def:element-lupus}{\textbf{Elementu Lupus}}. Wartość \texttt{exit} oznacza nagłe zaniechanie aktualnej iteracji \hyperlink{def:zamknieta-petla-sterowania}{\textbf{Pętli Sterowania}} (zazwyczaj wskutek błędu).

\subsubsection{SendAction}
\begin{lstlisting}[language=go, caption={SendAction}\label{lst:sendaction}]
// SendAction is used to make call to external-element
// Element's controller obtains a data field using InputKey,
// and attaches it as a json body when performing a call to destination.
// Response is saved in data under an OutputKey
type SendAction struct {
	InputKey    string      `json:"inputKey"`
	Destination Destination `json:"destination"`
	OutputKey   string      `json:"outputKey"`
}
\end{lstlisting}

\subsubsection{InsertAction}
\begin{lstlisting}[language=go, caption={InsertAction}\label{lst:insertaction}]
// InsertAction is used to make a new field and insert value to it
// Normally new fields are created as an outcome of other types of actions
// It is useful in debugging or logging, 
// e.g. can indicate the path taken by the actions workflow
type InsertAction struct {
	OutputKey string               `json:"outputKey"`
	Value     runtime.RawExtension `json:"value"`
}
\end{lstlisting}

\subsubsection{NestAction}
\begin{lstlisting}[language=go, caption={NestAction}\label{lst:nestaction}]
// NestAction is used to group a number of data-fields together.
// Element's controllers gather fields indicated by InputKeys list
// and nest them in a new field under an OutputKey.
type NestAction struct {
	InputKeys []string `json:"inputKeys"`
	OutputKey string   `json:"outputKey"`
}
\end{lstlisting}

\subsubsection{RemoveAction}
\begin{lstlisting}[language=go, caption={RemoveAction}\label{lst:removeaction}]
// RemoveAction is used to delete a data-field.
// Elements' controllers remove fields indicated by the list InputKeys
type RemoveAction struct {
	InputKeys []string `json:"inputKeys"`
}
\end{lstlisting}

\subsubsection{RenameAction}
\begin{lstlisting}[language=go, caption={RenameAction}\label{lst:renameaction}]
// RenameAction is used to change the name of a data-field.
// InputKey indicates a field to be renamed
// OutputKey is the new field name.
type RenameAction struct {
	InputKey  string `json:"inputKey"`
	OutputKey string `json:"outputKey"`
}
\end{lstlisting}

\subsubsection{DuplicateAction}
\begin{lstlisting}[language=go, caption={DuplicateAction}\label{lst:duplicateaction}]
// DuplicateAction is used to make a copy of a data-field.
// InputKey indicates the field of which value has to be copied.
// OutputKey indicates the field to which values have to be pasted in.
type DuplicateAction struct {
	InputKey  string `json:"inputKey"`
	OutputKey string `json:"outputKey"`
}
\end{lstlisting}

\subsubsection{PrintAction}
\begin{lstlisting}[language=go, caption={PrintAction}\label{lst:printaction}]
// PrintAction is used to print the value of each field 
// indicated by InputKeys in a controller's console.
// It is useful in debugging or logging.
type PrintAction struct {
	InputKeys []string `json:"inputKeys"`
}
\end{lstlisting}

\subsubsection{Switch}
\begin{lstlisting}[language=go, caption={Switch}\label{lst:switch}]
// Switch is a special type of action used for flow-control
// When Element's controller encounters switch action on the chain
// it emulates the work of a switch known in other programming languages
type Switch struct {
	Conditions []Condition `json:"conditions"`
}
\end{lstlisting}

\subsubsection{Condition}
\begin{lstlisting}[language=go, caption={Condition}\label{lst:condition}, basicstyle=\ttfamily\tiny]
// Condition represents a single condition present in Switch action
// It defines on which Data field it has to be performed, 
// the actual condition to be evaluated,
// and the next Action if evaluation returns true.
type Condition struct {
	// Key indicates the Data field that has to be retrieved
	Key string `json:"key"`
	// Operator defines the comparison operation, e.g. eq, ne, gt, lt
	Operator string `json:"operator" kubebuilder:"validation:Enum=eq,ne,gt,lt"`
	// Type specifies the type of the value: string, int, float, bool
	Type string `json:"type" kubebuilder:"validation:Enum=string,int,float,bool"`
	// One of these fields is not null depending on a Type.
	BoolCondition   *BoolCondition   `json:"bool,omitempty" kubebuilder:"validation:Optional"`
	IntCondition    *IntCondition    `json:"int,omitempty" kubebuilder:"validation:Optional"`
	StringCondition *StringCondition `json:"string,omitempty" kubebuilder:"validation:Optional"`
	// Next specifies the name of the next action to execute if evaluation returns true
	Next string `json:"next"`
}
\end{lstlisting}

\subsubsection{BoolCondition}
\begin{lstlisting}[language=go, caption={BoolCondition}\label{lst:boolcondition}]
// BoolCondition defines a boolean-specific condition
type BoolCondition struct {
	Value bool `json:"value"`
}
\end{lstlisting}

\subsubsection{IntCondition}
\begin{lstlisting}[language=go, caption={IntCondition}\label{lst:intcondition}]
// IntCondition defines an integer-specific condition
type IntCondition struct {
	Value int `json:"value"`
}
\end{lstlisting}

\subsubsection{StringCondition}
\begin{lstlisting}[language=go, caption={StringCondition}\label{lst:stringcondition}]
// StringCondition defines a string-specific condition
type StringCondition struct {
	Value string `json:"value"`
}
\end{lstlisting}

\appendix{Specyfikacja interfejsów Lupus}\label{appendix:4}

Specyfikacja w formie elektronicznej znajduje się pod linkiem: \url{https://github.com/0x41gawor/lupus/blob/master/docs/spec/lupin-lupout.md}.

\subsection{Architektura}

\begin{figure}[!h]
    \centering \includegraphics[width=1\linewidth]{a4-arch.png}
    \caption{Architektura Lupus}\label{fig:a4-arch}
\end{figure}

\subsection{Interfejs Lupin}

Projektant może zdefiniować wiele \textbf{Elementów Lupus}, połączonych na różne, skomplikowane sposoby. Musi jednak zdecydować, który z nich zostanie wywołany przez \textbf{IAgenta Ingress}. Taki element można nazwać \textbf{Element Ingress}. Lupus zaleca posiadanie tylko jednego Elementu Ingress.
Jeśli \textbf{Agent Ingress} chce zasygnalizować, że można zaobserwować nowy stan systemu zarządzanego (co oznacza, że musi zostać uruchomiona nowa iteracja pęli sterowania), musi zmodyfikować pole \texttt{Status.Input} w \textit{obiekcie API} \textbf{Elementu Ingress}. Wartość umieszczona w tym polu będzie reprezentować nowy \textbf{Aktualny Stan}.

Pole \texttt{Status.Input} w \textbf{Ingress Element CR} jest typu \texttt{RawExtension}, co oznacza, że podlega pod specyfikacje danych \\TODO załącznik daty

JSON przesłany w tym miejscu będzie stanowił \textbf{Dane} dla tego elementu.

Oprogramowanie implementuje interfejs \texttt{Lupin}, jeśli w pewnym miejscu swojego kodu wysyła żądanie HTTP do \textbf{kube-api-server}, które aktualizuje status \textbf{Elementu Ingress}, a dokładniej pole \texttt{input}. Wartość musi być obiektem JSON, który reprezentuje \textbf{Aktualny Stan} \textbf{Zarządzanego Systemu}. 

\subsection{Interfejs Lupout}

Punktem wyjścia z \textbf{Systemu Sterowania Lupus} jest ostatni \textbf{Element Lupus} czyli (\textbf{Element Egress}). Wysyła on swoje \textbf{finalne dane} (lub ich część) do \textbf{Agenta Egress}. \textbf{Egress Agent} musi przekształcić to wejście w \textbf{Akcje Sterowania}, wykonywaną bezpośrednio na \textbf{systemie zarządzanym}.

Oprogramowanie implementuje interfejs \texttt{Lupout}, jeśli implementuje serwer HTTP, który akceptuje wejściowe dane JSON i tłumaczy je na \textbf{Akcje Sterowania} wykonywaną na \textbf{Systemie Zarządzanym}. 


\appendix{Specfyfikacju obiektu danych}\label{appendix:5}

Specyfikacja w formie elektronicznej znajduje się pod linkiem: \url{https://github.com/0x41gawor/lupus/blob/master/docs/spec/data.md}.

Data jest kluczowym elementem spełnienia \hyperref[req:5]{Wymagania 5}. \textbf{Dane} jest to sposób w jaki \textbf{użytkownik}, podczas każdej iteracji, może:
\begin{itemize}
    \item uzyskać informacje o \textbf{Aktualnym Stanie}
    \item przechowywać pomocnicze informacje (takie jak odpowiedzi od \textbf{Elementów Zewnętrznych})
    \item przechowywać informacje debuggingowe
    \item zapisywać informacje potrzebne do sformułowania \textbf{Akcji Sterowania}
\end{itemize}

\textbf{Dane} są reprezentowane jako JSON dający się zapisać w strukturze Go \texttt{map[string]interface{}}. Nie może, więc być jedną z następujących form obiektu JSON:
\begin{itemize}
    \item typem prymitywnym,
    \item tablicą,
    \item obiektem JSON z kluczami innymi niż string.
\end{itemize}

To w jaki sposób można operować na danych prezentuje specyfikacja akcji //TODO link.
\appendix{Specyfikacja akcji}\label{appendix:6}

Specyfikacja w formie elektronicznej znajduje się pod linkiem: \url{https://github.com/0x41gawor/lupus/blob/master/docs/spec/actions.md}.

Specyfikacja \hyperlink{def:akcja}{\textbf{Akcji}} w \hyperlink{def:lupn}{\textbf{LupN}} znajduje się w \hyperref[appendix:3]{Załączniku 3}. Niniejszy załącznik prezentuje przykładowe działanie \hyperlink{def:akcja}{\textbf{Akcji}} na \hyperlink{def:dane}{\textbf{Danych}}.

\hyperlink{def:akcja}{\textbf{Akcje}} zostały opracowane jako najbardziej atomowe operacje, które, gdy zostaną odpowiednio połączone, stanowią narzędzie umożliwiające \hyperlink{def:projektant}{\textbf{Projektantowi Pętli}} pełne operowanie na \hyperlink{def:dane}{\textbf{Danych}}.

Czasami operacja, która na pierwszy rzut oka wydaje się atomowa, wymaga użycia dwóch połączonych \hyperlink{def:akcja}{\textbf{Akcji}}. Z drugiej strony, zdarza się, że operacja początkowo uznana za atomową okazuje się jedynie szczególnym przypadkiem bardziej ogólnej operacji. Dobrym przykładem jest nieistniejąca już \hyperlink{def:akcja}{\textbf{Akcja}} \texttt{concat}. Została ona zaprojektowana do łączenia dwóch pól w jedno, jednak okazało się, że jest to specyficzny przypadek \hyperlink{def:akcja}{\textbf{Akcji}} \texttt{nest}, w której lista \texttt{InputKey} zawiera tylko dwa elementy.

\subsection{Podział ogólny}

Mamy 8 typów akcji:

\begin{itemize}
    \item \textbf{Send}
    \item \textbf{Nest}
    \item \textbf{Remove}
    \item \textbf{Rename}
    \item \textbf{Duplicate}
    \item \textbf{Insert}
    \item \textbf{Print}
    \item \textbf{Switch}
\end{itemize}

Możemy wyróżnić następujące kategorie:

\begin{itemize}
    \item 6 akcji, które mogą być używane do modyfikacji danych: \\ 
          \texttt{\{Send, Nest, Remove, Rename, Duplicate, Insert\}}
    \item 1 akcja do komunikacji z \textbf{Elementami Zewnętrznymi}: \\ 
          \texttt{\{Send\}}
    \item 2 akcje do debugowania: \\ 
          \texttt{\{Insert, Print\}}
    \item 1 akcja do logowania: \\ 
          \texttt{\{Print\}}
    \item 1 akcja do sterowania przepływem \textbf{workflow akcji}: \\ 
          \texttt{\{Switch\}}
\end{itemize}

\subsection{Przykłady}

Załącznik przedstawi przykładowe użycie 6 akcji, które mogą modyfikować dane.

Każdy przykład zawiera:

\begin{itemize}
    \item reprezentację JSON stanu \textbf{danych} przed modyfikacją akcji,
    \item notację \textbf{LupN} zastosowanej akcji,
    \item reprezentację JSON stanu \textbf{danych} po modyfikacji akcji
\end{itemize}

\subsubsection{Send}
\begin{figure}[!h]
    \centering \includegraphics[width=1\linewidth]{a6-send.png}
    \caption{Przykład modyfikacji danych przez akcje Send}\label{fig:a6-send}
\end{figure}
\subsubsection{Nest}
\begin{figure}[!h]
    \centering \includegraphics[width=1\linewidth]{a6-nest.png}
    \caption{Przykład modyfikacji danych przez akcje Nest}\label{fig:a6-nest}
\end{figure}
\subsubsection{Remove}
\begin{figure}[!h]
    \centering \includegraphics[width=1\linewidth]{a6-remove.png}
    \caption{Przykład modyfikacji danych przez akcje Remove}\label{fig:a6-remove}
\end{figure}
\subsubsection{Rename}
\begin{figure}[!h]
    \centering \includegraphics[width=1\linewidth]{a6-rename.png}
    \caption{Przykład modyfikacji danych przez akcje Rename}\label{fig:a6-rename}
\end{figure}
\subsubsection{Duplicate}
\begin{figure}[!h]
    \centering \includegraphics[width=1\linewidth]{a6-duplicate.png}
    \caption{Przykład modyfikacji danych przez akcje Duplicate}\label{fig:a6-duplicate}
\end{figure}
\subsubsection{Insert}

\input{tex/appendix/7-test-agent-ingress.tex}
\input{tex/appendix/8-test-agent-egress.tex}
\input{tex/appendix/9-test-opa.tex}
\appendix{Kod LupN (model danych pętli sterowania)}\hypertarget{appendix:10}{}

Kod w wersji elektronicznej znajduje się pod linkiem: \url{https://github.com/0x41gawor/lupus/blob/master/examples/open5gs/sample-loop/master.yaml}

\begin{lstlisting}[language=sh, caption={\emph{Kod LupN}}, label={lst:a101}, numbers=left, stepnumber=1]
apiVersion: lupus.gawor.io/v1
kind: Master
metadata:
  labels:
    app.kubernetes.io/name: lupus
    app.kubernetes.io/managed-by: kustomize
  name: lola
spec:
  name: "lola"
  elements:
    - name: "demux"
      descr: "Demuxes Data input into separate elements for each UPF"
      actions: 
        - name: "insert1"
          type: insert
          insert:
            outputKey: "open5gs-upf1"
            value: {name: "open5gs-upf1"}
          next: "insert2"
        - name: "insert2"
          type: insert
          insert:
            outputKey: "open5gs-upf2"
            value: {name: "open5gs-upf2"}
          next: "print"
        - name: "print"
          type: print
          print:
            inputKeys: ["*"]
          next: final
      next:
        - type: element
          element:
            name: "upf1"
          keys: ["open5gs-upf1"]
        - type: element
          element:
            name: "upf2"
          keys: ["open5gs-upf2"]
    - name: "upf1"
      descr: "Reconcilation of UPF1 deployment"
      actions:
        - name: "print1"
          type: print
          print:
            inputKeys: ["*"]
          next: "opa-point"
        - name: "opa-point"
          type: send
          send: 
            inputKey: "*"
            destination: 
              type: opa
              opa: 
                path: http://192.168.56.112:9500/v1/data/policy/point
            outputKey: "point"
          next: "print2"
        - name: "print2"
          type: print
          print:
            inputKeys: ["*"]
          next: "switch1"
        - name: "switch1"
          type: switch
          switch:
            conditions:
              - key: "point"
                operator: eq
                type: string
                string: 
                  value: "NORMAL"
                next: final
          next: "opa-spec"
        - name: "opa-spec"
          type: send
          send: 
            inputKey: "actual"
            destination: 
              type: opa
              opa: 
                path: http://192.168.56.112:9500/v1/data/policy/spec
            outputKey: "spec"
          next: "print3"
        - name: "print3"
          type: print
          print:
            inputKeys: ["*"]
          next: "switch2"
        - name: "switch2"
          type: switch
          switch:
            conditions:
              - key: "point"
                operator: eq
                type: string
                string: 
                  value: "CRITICAL"
                next: final
          next: "print4"
        - name: "print4"
          type: print
          print:
            inputKeys: ["*"]
          next: "opa-interval"
        - name: "opa-interval"
          type: send
          send: 
            inputKey: "point"
            destination: 
              type: opa
              opa: 
                path: http://192.168.56.112:9500/v1/data/policy/interval
            outputKey: "interval"
          next: "print5"
        - name: "print5"
          type: print
          print:
            inputKeys: ["*"]
          next: final
      next: 
        - type: destination
          destination: 
            type: http
            http: 
              path: http://192.168.56.112:9001/api/data
              method: POST
          keys: ["*"]
    - name: "upf2"
      descr: "Reconcilation of UPF2 deployment"
      actions:
        - name: "print"
          type: print
          print:
            inputKeys: ["*"]
          next: "opa-point"
        - name: "opa-point"
          type: send
          send: 
            inputKey: "*"
            destination: 
              type: opa
              opa: 
                path: http://192.168.56.112:9500/v1/data/policy/point
            outputKey: "point"
          next: "print2"
        - name: "print2"
          type: print
          print:
            inputKeys: ["*"]
          next: "switch1"
        - name: "switch1"
          type: switch
          switch:
            conditions:
              - key: "point"
                operator: eq
                type: string
                string: 
                  value: "NORMAL"
                next: final
          next: "opa-spec"
        - name: "opa-spec"
          type: send
          send: 
            inputKey: "actual"
            destination: 
              type: opa
              opa: 
                path: http://192.168.56.112:9500/v1/data/policy/spec
            outputKey: "spec"
          next: "print3"
        - name: "print3"
          type: print
          print:
            inputKeys: ["*"]
          next: "switch2"
        - name: "switch2"
          type: switch
          switch:
            conditions:
              - key: "point"
                operator: eq
                type: string
                string: 
                  value: "CRITICAL"
                next: final
          next: "print4"
        - name: "print4"
          type: print
          print:
            inputKeys: ["*"]
          next: "opa-interval"
        - name: "opa-interval"
          type: send
          send: 
            inputKey: "point"
            destination: 
              type: opa
              opa: 
                path: http://192.168.56.112:9500/v1/data/policy/interval
            outputKey: "interval"
          next: "print5"
        - name: "print5"
          type: print
          print:
            inputKeys: ["*"]
          next: final
      next: 
        - type: destination
          destination: 
            type: http
            http: 
              path: http://192.168.56.112:9001/api/data
              method: POST
          keys: ["*"]
\end{lstlisting}


\textbf{Krótki opis powyższego pliku}

Kod LupN w rzeczywistości jest plikiem manifestowym YAML zasobu \textbf{Master}. Specyfikacje obiektu tego typu zasobu w całości opisuje modelowaną pętle sterowania, dlatego analizę należy rozpocząć od linii 8. W lini 9 widzimy nazwę pętli. W lini 10 zdefiniowany jest obiekt elementów jakie należą do pętli. Omawiana w tym przykładzie pętla ma trzy elementy a ich obiekty znajdziemy odpowiednio w liniach: 11, 40 oraz 128. Skupmy się na elemencie pierwszym o nazwie "demux", którego celem jest rozdzielenie danych na dwie części i przekazanie odpowiednich części do następnych elementów pętli. Jego opis znajduje się między liniami 11 a 39. Linia 13 rozpoczyna opis workflow akcji. Mamy zdefiniowane tu trzy akcje (linie: 14, 20 i 26). Pierwsza akcja dodaje do danych pole \texttt{\{name: "open5gs-upf1"\}} w odpowiedniej sekcji struktury danych (przedstawionej w listingu \ref{lst:521}). Analogiczna akcja jest wykonywana dla drugiej części danych. Celem tej operacji jest nazwanie obu sekcji danych, aby później Agent Egress mógł rozpoznać na rzecz, którego wdrożenia UPF wykonuje akcje sterującą. Ostatnią akcją elementu "demux" jest wyświetlenie zawartości danych w logach operatorach. 

Kiedy workflow akcji dobiegnie końca, dane wysyłane są do destynacji zdefiniowanych przez obiekt specyfikowany w linii 31. Posiada on dwa obiekty next. Oba typu element. Każdemu elementowi przekazywana jest inna sekcja danych, specyfikowana poprzez klucz wskazujący na odpowiednie pole danych.

Kolejny element - "upf1" - odpowiedzialny jest za rekoncyliację wdrożenia UPF 1. Jego workflow akcji jest następujące:
\begin{itemize}
  \item Linia 43: Wyświetlenie zawartości danych
  \item Linia 48: Wysłanie wszystkich danych (operator \texttt{*}) do elementu zewnętrznego jakim jest "POINT"
  \item Linia 58: Ponowne wyświetlenie zawartości danych
  \item Linia 63: Decyzja na podstawie otrzymanej odpowiedzi od elementu zewnętrznego "POINT". Jeśli stan jest normalny, następuje natychmiastowe przejście do końca workflow akcji.
  \item Linia 74: Wysłanie rzeczywistego zużycia zasobów do elementu zewnętrznego "SPEC".
  \item Linia 84: Ponowne wyświetlenie zawartości danych
  \item Linia 89: Decyzja na podstawie pola "point", jeśli jest ono równe "CRITIAL" następuje natychmiastowe przejście do końca workflow akcji
  \item Linia 100: Ponowne wyświetlenie zawartości danych
  \item Linia 105: Wysłanie żądania HTTP z body w postaci pola JSON "point" na adres \texttt{http://192.168.56.112:9500/v1/data/policy/interval}
\end{itemize}

Pomijając akcje logujące stan danych, workflow elementu składa się z:
\begin{enumerate}
  \item trzech akcji typu \texttt{send}, które mimo różnych opisów powyżej różnią się jedynie parametrami wywołania.
  \item dwóch akcji typu \texttt{switch}, które decydują czy następne akcje są istotne w danej iteracji pętli
\end{enumerate} 

Kiedy workflow akcji dobiegnie końca, dane wysyłane są do destynacji zdefiniowanych przez obiekt specyfikowany w linii 120. Tym razem jest ono jednoelementowy i reprezentuje Agenta Egress.

Opis elementu "upf2" zawarty w liniach 128 do 215 jest analogiczny jak w przypadku elementu "upf1".

\end{document} % Dobranoc.
