%-----------------------------------------------
%  Engineer's & Master's Thesis Template
%  Copyleft by Artur M. Brodzki & Piotr Woźniak
%  Warsaw University of Technology, 2019-2022
%-----------------------------------------------

\documentclass[
    bindingoffset=5mm,  % Binding offset
    footnoteindent=3mm, % Footnote indent
    hyphenation=true    % Hyphenation turn on/off
]{src/wut-thesis}

\graphicspath{{tex/img/}} % Katalog z obrazkami.
\addbibresource{bibliografia.bib} % Plik .bib z bibliografią

\facultyeiti    % Wydział Elektroniki i Technik Informacyjnych
\MasterThesis % Praca inżynierska
\langpol % Praca w języku polskim

\begin{document}

%------------------
% Strona tytułowa
%------------------
\instytut{Instytut telekomunikacji}
\kierunek{Telekomunikacja}
\specjalnosc{Teleinformatyka i Zarządzanie w Telekomunikacji}
\title{
    Sworzenie platformy do modelowania i uruchamiania zamkniętych pętli sterowania w Kubernetes
}
\engtitle{
    Creation of a platform for designing and running closed control loops in Kubernetes
}

\author{Andrzej Gawor}
\album{300528}
\promotor{dr inż. Dariusz Bursztynowski}
\date{\the\year}
\maketitle

%-------------------------------------
% Streszczenie po polsku 
%-------------------------------------
\cleardoublepage % Zaczynamy od nieparzystej strony
\abstract
Projekt opisany w niniejszej pracy skupia się na zaproponowaniu oraz zaimplementowaniu reużywalnej architektury (zwanej dalej "Platformą"), która pozwala na modelowanie oraz uruchamianie zamkniętych pętli sterowania w Kubernetes. Genezą projektu jest praca jednego z komitetów ETSI o nazwie "ENI - Experiential Networked Intelligence", która skupia się na ułatwieniu pracy operatora sieci telekomunikacyjnych wykorzystując mechanizmy sztucznej inteligencji w zamkniętych pętlach sterowania opartych na politykach, które są sterowanie metadanymi oraz świadome kontekstu. ENI w jednym ze swoich dokumentów dokonuje przeglądu zamkniętych pętli sterowania znanych ludzkości z innych dziedzin. Naturalnym następnym krokiem jest zapropowanie platformy, na której operator mógłby takowe pętle projektować oraz uruchamiać. W tym celu zdefiniowanio zestaw wymagań oraz założeń dla takiego systemu. Jako środowisko uruchomieniowe wybrano  Kubernetes z racji, że jest to system dobrze znany w społeczności oraz sam natywnie używa zamkniętych pętli sterowania. Nastpęnie przeprowadzono obszerną analizę jak za pomocą mechanizmów rozszerzania Kubernetes takich jak "Custom Resources" oraz "Operator" pattern można stworzyć framework umożliwiający modelowanie zamkniętych pętli sterowania. Praca opisuje powstałą platformę, jej architekturę, semantykę składni w definiowanych obiektach, zasady działania, integracje z zewnętrznymi systemami wraz z instrukcją jej użycia. Omówiona została również implemtacja platformy, technologie za nią stojące oraz decyzje podjęte podczas jej powstawania. Finalnie przedstawiono również test działania platformy w praktyce wykorzystując do tego emulator systemu 5G jakim jest Open5GS w połączeniu z UERANSIM. Pracę podsumuję lista wniosków oraz potencjalnych rozszerzeń lub poprawek platformy. 
\keywords Zamknięte pętle sterowania, Kubernetes, Zarządzanie sieciami telekomunikacyjnymi, Automatyzacja, Go, Open5GS, Mikroserwisy

%----------------------------------------
% Streszczenie po angielsku
%----------------------------------------
\clearpage
\secondabstract
The project described in this work focuses on proposing and presenting the implementation of a platform architecture that enables the modeling and execution of closed control loops in Kubernetes. The genesis of the project stems from the work of standardization committees such as ETSI ZSM and ETSI ENI, which focus on specifying autonomous and cognitive management systems for telecommunication networks. The architectures of these specified systems are based on closed control loops.

A natural next step is to propose a platform where an operator could design and execute such loops. To this end, a set of requirements and assumptions for such a system has been defined. Kubernetes was chosen as the runtime environment. Subsequently, an extensive analysis was conducted on how Kubernetes extension mechanisms, such as "Custom Resources" and the "Operator Pattern," can be leveraged to create a framework that enables the modeling of closed control loops. Based on this analysis, a highly flexible architecture was developed and then implemented using the Kubebuilder framework. Finally, the platform was tested using a 5G network emulator, specifically Open5GS combined with UERANSIM.

This work describes a review of related literature, the developed platform, its architecture, and a user guide. It also discusses an exemplary implementation of the platform, the technologies behind it, and the decisions made during its development. Finally, a practical test of the platform's functionality is presented. The work concludes with a summary of findings and potential future development paths for the platform.
\secondkeywords Closed Control Loops, Kubernetes, Managing Telco-Networks, Automation, Go, Open5GS, Microservices

\pagestyle{plain}

%--------------
% Spis treści
%--------------
\cleardoublepage % Zaczynamy od nieparzystej strony
\tableofcontents

%------------
% Rozdziały
%------------
\cleardoublepage % Zaczynamy od nieparzystej strony
\pagestyle{headings}
\newpage
\section{Wstęp}
\subsection{Przedmowa}

Wraz z rozwojem telekomunikacji stopień jej skomplikowania oraz mnogość podłączonych urządzeń stale rośnie. Sieci 5G zwiastują obsługę miliardów urządzeń, co sprawia, że tradycyjne podejście do zarządzania sieciami staje się niewystarczające. W pewnym momencie manualne operowanie sieciami (ang. \textit{human-driven networks}) stanie się wręcz niemożliwe. Dlatego obserwujemy obecnie zwrot w stronę wirtualizacji oraz automatyzacji sieci. Równocześnie dynamiczny rozwój sztucznej inteligencji otwiera nowe możliwości. Te dwa czynniki stanowią wspólnie solidny fundament do tego, aby branża sieci telekomunikacyjnych postawiła sobie za cel budowę "inteligentnych" sieci - takich, które są w pełni autonomiczne, samowystarczalne oraz nie wymagają nadzoru ludzkiego.

W tym celu ETSI (European Telecommunications Standards Institute) powołało dwa komitety: ENI (Experiential Networked Intelligence) oraz ZSM (Zero touch network \& Service Management). Oba komitety mają na celu wypracowanie specyfikacji Kognitywnych Systemów Zarządzania Siecią (Cognitive Network Management system). Kognitywny system oznacza taki, który jest w stanie uczyć się i podejmować decyzje, bazując na zebranej wiedzy, w sposób przypominający ludzki umysł. Obie architektury opierają swoje działanie na sztucznej inteligencji oraz zamkniętych pętlach sterowania. 

Pętlą sterowania ETSI nazywa mechanizm, który monitoruje wydajność systemu lub procesu poddawanego kontroli w celu osiągnięcia pożądanego zachowania. Innymi słowy, pętla sterowania reguluje działanie zarządzanego obiektu. Pętle sterowania można podzielić na zamknięte lub otwarte, w zależności od tego, czy działanie sterujące zależy od sprzężenia zwrotnego z kontrolowanego obiektu. Jeśli tak, pętle nazywamy zamkniętą, jeśli nie - otwartą. 

W przypadku architektur ETSI, zamknięta pętla sterowania służy jako model organizacji przepływu pracy (ang. \hyperlink{def:workflow}{\textit{workflow}}) elementów odpowiedzialnych za sztuczną inteligencję. Obie architektury referencyjne opisują modelowanie, uruchamianie oraz zarządzanie zamkniętymi pętlami sterowania w celu osiągania celów zarządzania. Jednak wciąż brakuje ogólnodostępnej platformy, która pozwalałaby operatorom w praktyce projektować, wdrażać i zarządzać takimi pętlami sterowania zgodnie z założeniami ETSI ZSM i ENI, co stanowi istotną lukę technologiczną. Naturalnym następnym krokiem jest zaproponowanie architektury platformy, która umożliwia takie działania oraz wpisuje się w aktualne trendy systemów teleinformatycznych, takich jak podejście cloud-native, wykorzystanie mikrousług oraz integracja z ekosystemem Kubernetes.

\input{tex/1-wstep/2-cele-pracy.tex}
\subsection{Struktura pracy}

Praca została podzielona na 6 rozdziałów. Drugi rozdział przedstawia bardziej szczegółowo niż we wstępie badania podjęte przez ENI. Stanowi on wprowadzenie teoretyczne oraz pojęciowe, aby ułatwić przekaz w dalszej części pracy. Rozdział trzeci zawiera opis proponowanej architektury. Rozdział czwarty opisuje implementację PoC platformy oraz napotkane wyzwania podczas formułowania architektury, które celowo zostały zebrane w jedno miejsce i umieszczone oddzielnie w celu łatwiejszej lektury. Rozdział numer pięć przedstawia użycie platformy w praktyce przy okazji stanowiąc jej test. Na koniec, w rozdziale szóstym przeprowadzono analizę zaproponowanej architektury, jej możliwości oraz ograniczenia oraz wskazano kierunki potencjalnego rozwoju. 

%---------------
% Literatura
%---------------
\cleardoublepage % Zaczynamy od nieparzystej strony
\printbibliography
\clearpage



%--------------------------------------
% Spisy
%--------------------------------------
\newpage
\pagestyle{plain}
\vspace{0.8cm}
\acronymlist
\acronym{5GC}{5G Core Network}
\acronym{5GS}{5G System}

\listoffigurestoc    % Spis rysunków.
\vspace{1cm}         % vertical space
\listoftablestoc     % Spis tabel.
\vspace{1cm}         
\listofappendicestoc % Spis załączników

%-------------
% Załączniki
%-------------
% Obrazki i tabele w załącznikach nie trafiają do spisów
\captionsetup[figure]{list=no}
\captionsetup[table]{list=no}

\appendix{Definicje Lupus}\label{appendix:1}

Specyfikacja w formie elektronicznej znajduje się pod linkiem: \url{https://github.com/0x41gawor/lupus/blob/master/docs/defs.md}.
\appendix{Instalacja Lupus}\label{appendix:2}

Specyfikacja w formie elektronicznej znajduje się pod linkiem: \url{https://github.com/0x41gawor/lupus/blob/master/docs/installation.md}.

\subsection{Przedsłowie}

Instalacja \hyperlink{def:lupus}{\textbf{Lupus}} wymaga umiejętności technicznych oraz podstawowej znajomości operacyjnej Kubernetes.

\hyperlink{def:lupus}{\textbf{Lupus}} jest zaimplementowany jako projekt Kubebuilder\footnote{\url{https://book.kubebuilder.io}}. Zalecanym sposobem instalacji \hyperlink{def:lupus}{\textbf{Lupus}} jest sklonowanie tego repozytorium i przyjęcie roli dewelopera tego projektu.

Nie istnieje coś takiego jak instalacja \hyperlink{def:lupus}{\textbf{Lupus}} (np. w systemie operacyjnym). Można zainstalować \hyperlink{def:zasoby-wlasne}{\textbf{Zasoby Własne}} dla \hyperlink{def:element-lupus}{\textbf{Elementów Lupus}} w klastrze Kubernetes i uruchomić dla nich \hyperlink{def:operator-zasobu-element}{\textbf{Kontrolery}}. Niniejszy załącznik opisuje właśnie taki proces.

\subsection{Wymagania wstępne}

\hyperlink{def:uzytkownik}{\textbf{Użytkownik}} musi posiadać działający klaster Kubernetes. Może to być Minikube\footnote{\url{https://minikube.sigs.k8s.io/docs/}}, zainstalowany silnik kontenerów (ang. \textit{container engine}) (np. Docker\footnote{\url{https://docs.docker.com}}) oraz język Go\footnote{\url{https://go.dev}}.

\subsubsection{Instalacja Kubebuilder}

Instrukcja dostępna pod adresem: \url{https://book.kubebuilder.io/quick-start}.

\subsection{Klonowanie repozytorium}

\begin{lstlisting}[language=bash, caption={Klonowanie repozytorium}]
git clone https://github.com/0x41gawor/lupus
cd lupus
\end{lstlisting}

\subsection{Instalacja CRD w klastrze}

To polecenie zastosuje \hyperlink{def:crd}{\textbf{CRD}} (pl. Definicje \hyperlink{def:zasoby-wlasne}{\textbf{Zasobów Własnych}}) dla \hyperlink{def:master}{\textbf{Master}} i \hyperlink{def:element}{\textbf{Element}}, umożliwiając ich użycie.

\begin{lstlisting}[language=bash, caption={Instalacja CRD}]
make install
\end{lstlisting}

\subsection{Uruchomienie kontrolerów }

To polecenie uruchomi \textit{kontrolery} dla \textit{zasobów własnych} \hyperlink{def:master}{\textbf{master}} i \hyperlink{def:element}{\textbf{element}}.

\begin{lstlisting}[language=bash, caption={Uruchomienie kontrolerów}]
make run
\end{lstlisting}

Istnieje możliwość uruchomienia kontrolerów jako pody w klastrze Kubernetes. W tym celu użytkownik jest zaproszony do bliższego zapoznania się z platformą Kubebuilder. Dopóki \hyperlink{def:uzytkownik}{\textbf{Użytkownik}} jest pewien, że nie będzie dopisywał \hyperlink{def:funkcje-uzytkownika}{\textbf{Funkcji Użytkownika}} nie jest to zalecane podejście. 



\appendix{Specyfikacja notacji LupN}\label{appendix:3}\hypertarget{appendix:3}{}

Specyfikacja w formie elektronicznej znajduje się pod linkiem: \url{https://github.com/0x41gawor/lupus/blob/master/docs/spec/lupn.md}.


\hyperlink{def:lupn}{\textbf{LupN}} (od ang. \textit{loop} oraz \textit{Notation}) to język/notacja służąca do wyrażania \hyperlink{def:workflow-petli}{\textbf{Workflow Pętli}}. Nie zawiera opisu \hyperlink{def:czesc-obliczeniowa}{\textbf{Części Obliczeniowej}} \hyperlink{def:logika-petli}{\textbf{Logiki Pętli}}. \hyperlink{def:czesc-obliczeniowa}{\textbf{Część Obliczeniowa}} jest określona poza \hyperlink{def:lupus}{\textbf{Lupus}}, w \hyperlink{def:element-zewnetrzny}{\textbf{Elementach Zewnętrznych}}.

\texttt{LupN} specyfikuje:
\begin{itemize}
    \item \hyperlink{def:workflow-petli}{\textbf{Workflow Pętli}}, czyli workflow \hyperlink{def:element-lupus}{\textbf{Elementów Lupus}},
    \item odniesienia do \hyperlink{def:element-zewnetrzny}{\textbf{Elementów Zewnętrznych}}, wyrażone jako \hyperlink{def:destynacja}{\textbf{Destynacje}},
    \item \hyperlink{def:workflow-petli}{\textbf{Workflow Akcji}} w ramach \hyperlink{def:element-lupus}{\textbf{Elementu Lupus}},
    \item odniesienie (lub odniesienia) do \hyperlink{def:agent-egress}{\textbf{Agenta Egress}} jako \hyperlink{def:destynacja}{\textbf{Destynacja}}.
\end{itemize}

Można zauważyć iż, \texttt{LupN} wyraża \textbf{workflow} na dwóch poziomach: globalnym (czyli \hyperlink{def:workflow-petli}{\textbf{Workflow Elementów Lupus}}) oraz wewnątrz \hyperlink{def:element-lupus}{\textbf{Elementu Lupus}} (czyli \hyperlink{def:workflow-petli}{\textbf{Workflow Akcji}}). Możliwości obu poziomów są do siebie zbliżone, ale ostatecznie różne. Ten załącznik omówi również tę kwestię.


Z punktu widzenia implementacji \hyperlink{def:plik-lupn}{\textbf{Plik LupN}} to w rzeczywistości \textit{YAML manifest file} dla \hyperlink{def:zasoby-wlasne}{\textit{Zasobu Własnego}} \hyperlink{def:master}{\textbf{Master}}. Po zaaplikowaniu (ang. \textit{apply}), \hyperlink{def:operator-zasobu-master}{\textbf{Operator Zasobu Master}} uruchamia \hyperlink{def:element-lupus}{\textbf{Elementy Lupus}}, które realizują wyrażone \hyperlink{def:workflow-petli}{\textbf{Workflow Pętli}}.

\texttt{LupN} wyraża \hyperlink{def:workflow-petli}{\textbf{Workflow Pętli}} poprzez specyfikację różnych obiektów w notacji YAML. Obiekty te nazwano \hyperlink{def:obiekt-lupn}{\textbf{Obiektami LupN}}. \hyperref[appendix:3]{Załącznik 3} specyfikuje te obiekty oraz relacje między nimi. Wskazuje również, co oznacza użycie każdego z nich w kontekście \hyperlink{def:workflow-petli}{\textbf{Workflow Pętli}} i jak \hyperlink{def:operator-zasobu-element}{\textbf{Operator Zasobu Element}} interpretuje je w czasie działania.

Obiekty YAML w \hyperlink{def:plik-lupn}{\textbf{Pliku LupN}} są pochodnymi struktur Go, dlatego \hyperlink{def:obiekt-lupn}{\textbf{Obiekty LupN}} możemy opisać na ich podstawie. 

Wymagane jest wcześniejsze zapoznanie się z formatem YAML. Załącznik nie obejmuje translacji dokonywanej przez Kubernetes między strukturami Go a reprezentacjami obiektów YAML. Serializacja jest wykonywana przez \texttt{controller-gen} i opisana w \textit{Kubebuilder Book}. Tłumaczenie to można łatwo zaobserwować i nauczyć się, analizując przykłady zamieszczone w repozytorium projektu: \url{https://github.com/0x41gawor/lupus/tree/master/examples}. Dodatkowo, w \hyperlink{appendix:11}{Załączniku 11} zamieszczono krótkie komentarze pliku LupN wykorzystywanego w teście platformy opisanym w rozdziale \ref{sec:5}.

\subsection{Możliwości LupN}

Ze względu na różnice implementacyjne węzłów omówione w podrozdziale \ref{sec:dwa-rodzaje-workflow}, możliwości \hyperlink{def:workflow-petli}{Workflow Pętli} oraz \hyperlink{def:workflow-akcji}{Workflow Akcji} różnią się od siebie. Sekwencja wykonawcza elementów jest definiowana poprzez obiekty \texttt{next}. Każdy element definiuje listę następnych elementów. Zazwyczaj jest to jeden element. Możliwe są rozgałęzienia (ang. \textit{forks}) czyli lista z większą ilością elementów, ale wtedy zostaje wywołane wiele elementów niezależnie. Nie ma możliwości na zsynchronizowane powrotne złączenie przepływu danych. W przypadku akcji stosowany jest ten sam mechanizm opierający się na definiowaniu następnej akcji, z tym, że tutaj może być ona tylko jedna. Flow akcji interpretowane jest bowiem przez kontroler elementu, który procesuje je po jednej na raz. Z tego powodu możliwe jest sterowanie przepływem (ang. \textit{flow control}), które zostało zaimplementowane jako specjalny typ akcji - \texttt{switch}. 

\subsection{Specyfikacja}

\textbf{Plik LupN} posiada 4 główne pola (ang. \textit{top-level fields}).

\begin{lstlisting}[language=bash, caption={Główne pola pliku LupN}\label{lst:a31}]
apiVersion: lupus.gawor.io/v1
kind: Master
metadata:
  labels:
    app.kubernetes.io/name: lupus
    app.kubernetes.io/managed-by: kustomize
  name: lola
spec:
	<lupn-objects>
\end{lstlisting}

Każde z nich musi być ustawione jak w \ref{lst:a31} oprócz \texttt{metadata.name}, to pole odróżnia instancje pętli między sobą w obrębie klastra Kubernetes.

Notacja LupN rozpoczyna się od pola \texttt{spec}. Każdy obiekt Lupn zostanie opisany poprzez swoją definicję w Go.

\subsubsection{Drzewo obiektów LupN}

Podczas przeglądania specyfikacji \hyperlink{def:obiekt-lupn}{\textbf{obiektów LupN}} pomocne będzie śledzenie aktualnej pozycji w drzewie zależności obiektów. 

\begin{figure}[!h]
    \centering \includegraphics[width=1\linewidth]{a3-tree.png}
    \caption{Drzewko zależności obiektów LupN. Źródło: Opracowanie własne.}\label{fig:a3-tree}
\end{figure}

\subsubsection{MasterSpec}
\begin{lstlisting}[language=go, caption={MasterSpec}\label{lst:a32}]
// MasterSpec defines the desired state of Master
type MasterSpec struct {
	// Name of the Master CR (indicating the name of the loop)
	Name string `json:"name"`
	// Elements is a list of Lupus-Elements
	Elements []*ElementSpec `json:"elements"`
}
\end{lstlisting}
Każdy element na liście \texttt{Elements} spowoduje, że \textbf{Operator Zasobu Master} stworzy obiekt API typu \textbf{Lupus Element}. 

\subsubsection{ElementSpec}
\begin{lstlisting}[language=go, caption={MasterSpec}\label{lst:a32}, basicstyle=\ttfamily\tiny]
// ElementSpec defines the desired state of Element
type ElementSpec struct {
	// Name is the name of the element, its distinct from Kubernetes API Object name, 
    // but rather serves ease of managemenet aspect for loop-designer
	Name string `json:"name"`
	// Descr is the description of the lupus-element, same as Name it serves as 
    // the ease of management aspect for loop-designer
	Descr string `json:"descr"`
	// Actions is a list of Actions that lupus-element has to perform
	Actions []Action `json:"actions,omitempty"`
	// Next is a list of next objects (can be lupus-element or external-element) 
    // to which send the final-data
	Next []Next `json:"next,omitempty"`
	// Name of master element (used as prefix for lupus-element name)
	Master string `json:"master,omitempty"`
}
\end{lstlisting}

\subsubsection{Next}
\begin{lstlisting}[language=go, caption={Next}\label{lst:next}, basicstyle=\ttfamily\tiny]
// Next specifies the next loop-element in a loop workflow, 
// it may be either lupus-element or reference to an external-element
// It allows to forward the whole final-data, but also parts of it
type Next struct {
	// Type specifies the type of next loop-element, lupus-element (element) 
    // or external-element (destination)
	Type string `json:"type" kubebuilder:"validation:Enum=element,destination"`
	// List of input keys (Data fields) that have to be forwarded
	// Pass array with single element '*' to forward the whole input
	Keys []string `json:"keys"`
	// One of the fields below is not null
	Element     *NextElement `json:"element,omitempty" kubebuilder:"validation:Optional"`
	Destination *Destination `json:"destination,omitempty" kubebuilder:"validation:Optional"`
}
\end{lstlisting}

\subsubsection{NextElement}
\begin{lstlisting}[language=go, caption={NextElement}\label{lst:nextelement}]
// NextElement indicates the next loop-element 
// in loop-workflow of type lupus-element
type NextElement struct {
	// Name is the lupus-name of lupus-element 
	// (the one specified in Element struct)
	Name string `json:"name"`
}
\end{lstlisting}

\subsubsection{Destination}
\begin{lstlisting}[language=go, caption={Destination}\label{lst:destination}, basicstyle=\ttfamily\tiny]
// Destination represents an external-element
// It holds all the info needed to make a call to an external-element
// It supports calls to HTTP server, Open Policy Agent or user-functions
type Destination struct {
	// Type specifies if the external element is: a HTTP server in general, 
    // a special kind of HTTP server like Open Policy Agent or internal, a user-function
	Type string `json:"type" kubebuilder:"validation:Enum=http;opa;gofunc"`
	// One of these fields is not null depending on a Type
	HTTP   *HTTPDestination   `json:"http,omitempty" kubebuilder:"validation:Optional"`
	Opa    *OpaDestination    `json:"opa,omitempty" kubebuilder:"validation:Optional"`
	GoFunc *GoFuncDestination `json:"gofunc,omitempty" kubebuilder:"validation:Optional"`
}
\end{lstlisting}

\subsubsection{HTTPDestination}
\begin{lstlisting}[language=go, caption={HTTPDestination}\label{lst:httpdestination}]
// HTTPDestination defines fields specific to a HTTP type
// This is information needed to make a HTTP request
type HTTPDestination struct {
	// Path specifies HTTP URI
	Path string `json:"path"`
	// Method specifies HTTP method
	Method string `json:"method"`
}
\end{lstlisting}

\subsubsection{OpaDestination}
\begin{lstlisting}[language=go, caption={OpaDestination}\label{lst:opadestination}]
// OpaDestination defines fields specific to Open Policy Agent type
// This is information needed to make an Open Policy Agent request
// Call to Opa is actually a special type of HTTP call
type OpaDestination struct {
	// Path specifies HTTP URI, since method is known
	Path string `json:"path"`
}
\end{lstlisting}

\subsubsection{GoFuncDestination}
\begin{lstlisting}[language=go, caption={GoFuncDestination}\label{lst:gofuncdestination}]
// GoFuncDestination defines fields specific to GoFunc type
// This is information needed to call an user-function
type GoFuncDestination struct {
	// Name specifies the name of the function
	Name string `json:"name"`
}
\end{lstlisting}

\subsubsection{Action}
\begin{lstlisting}[language=go, caption={Action}\label{lst:action}, basicstyle=\ttfamily\tiny]
// Action represents operation that is performed on Data
// Action is used in Element spec. Element has a list of Actions 
// and executes them in a workflow manner
// In general, each action has an input and output keys that define 
// which Data fields it has to work on
// Each action indicates the name of the next Action in Action Chain
// There is special type - Switch. Actually, it does not perform any operation on Data, 
// but rather controls the flow of Actions chain
type Action struct {
	// Name of the Action, it is for designer to ease the management of the Loop
	Name string `json:"name"`
	// Type of Action
	Type string `json:"type" kubebuilder:"validation:Enum=send,nest,remove,rename,duplicate,print,insert,switch"`
	// One of these fields is not null depending on a Type.
	Send      *SendAction      `json:"send,omitempty" kubebuilder:"validation:Optional"`
	Nest      *NestAction      `json:"nest,omitempty" kubebuilder:"validation:Optional"`
	Remove    *RemoveAction    `json:"remove,omitempty" kubebuilder:"validation:Optional"`
	Rename    *RenameAction    `json:"rename,omitempty" kubebuilder:"validation:Optional"`
	Duplicate *DuplicateAction `json:"duplicate,omitempty" kubebuilder:"validation:Optional"`
	Print     *PrintAction     `json:"print,omitempty" kubebuilder:"validation:Optional"`
	Insert    *InsertAction    `json:"insert,omitempty" kubebuilder:"validation:Optional"`
	Switch    *Switch          `json:"switch,omitempty" kubebuilder:"validation:Optional"`
	// Next is the name of the next action to execute, in the case of Switch-type action it stands as a default branch
	Next string `json:"next"`
}
\end{lstlisting}

Pole \texttt{Next}, oprócz nazw akcji, może przyjąć jedną z dwóch zdefiniowanych wartości. Wartość \texttt{final} oznacza, że postać \hyperlink{def:dane}{\textbf{Danych}} po tej akcji jest już w swojej \hyperlink{def:finalne-dane}{\textbf{Finalnej Postaci}} i musi zostać przekazana do następnego \hyperlink{def:element-lupus}{\textbf{Elementu Lupus}}. Wartość \texttt{exit} oznacza nagłe zaniechanie aktualnej iteracji \hyperlink{def:zamknieta-petla-sterowania}{\textbf{Pętli Sterowania}} (zazwyczaj wskutek błędu).

\subsubsection{SendAction}
\begin{lstlisting}[language=go, caption={SendAction}\label{lst:sendaction}]
// SendAction is used to make call to external-element
// Element's controller obtains a data field using InputKey,
// and attaches it as a json body when performing a call to destination.
// Response is saved in data under an OutputKey
type SendAction struct {
	InputKey    string      `json:"inputKey"`
	Destination Destination `json:"destination"`
	OutputKey   string      `json:"outputKey"`
}
\end{lstlisting}

\subsubsection{InsertAction}
\begin{lstlisting}[language=go, caption={InsertAction}\label{lst:insertaction}]
// InsertAction is used to make a new field and insert value to it
// Normally new fields are created as an outcome of other types of actions
// It is useful in debugging or logging, 
// e.g. can indicate the path taken by the actions workflow
type InsertAction struct {
	OutputKey string               `json:"outputKey"`
	Value     runtime.RawExtension `json:"value"`
}
\end{lstlisting}

\subsubsection{NestAction}
\begin{lstlisting}[language=go, caption={NestAction}\label{lst:nestaction}]
// NestAction is used to group a number of data-fields together.
// Element's controllers gather fields indicated by InputKeys list
// and nest them in a new field under an OutputKey.
type NestAction struct {
	InputKeys []string `json:"inputKeys"`
	OutputKey string   `json:"outputKey"`
}
\end{lstlisting}

\subsubsection{RemoveAction}
\begin{lstlisting}[language=go, caption={RemoveAction}\label{lst:removeaction}]
// RemoveAction is used to delete a data-field.
// Elements' controllers remove fields indicated by the list InputKeys
type RemoveAction struct {
	InputKeys []string `json:"inputKeys"`
}
\end{lstlisting}

\subsubsection{RenameAction}
\begin{lstlisting}[language=go, caption={RenameAction}\label{lst:renameaction}]
// RenameAction is used to change the name of a data-field.
// InputKey indicates a field to be renamed
// OutputKey is the new field name.
type RenameAction struct {
	InputKey  string `json:"inputKey"`
	OutputKey string `json:"outputKey"`
}
\end{lstlisting}

\subsubsection{DuplicateAction}
\begin{lstlisting}[language=go, caption={DuplicateAction}\label{lst:duplicateaction}]
// DuplicateAction is used to make a copy of a data-field.
// InputKey indicates the field of which value has to be copied.
// OutputKey indicates the field to which values have to be pasted in.
type DuplicateAction struct {
	InputKey  string `json:"inputKey"`
	OutputKey string `json:"outputKey"`
}
\end{lstlisting}

\subsubsection{PrintAction}
\begin{lstlisting}[language=go, caption={PrintAction}\label{lst:printaction}]
// PrintAction is used to print the value of each field 
// indicated by InputKeys in a controller's console.
// It is useful in debugging or logging.
type PrintAction struct {
	InputKeys []string `json:"inputKeys"`
}
\end{lstlisting}

\subsubsection{Switch}
\begin{lstlisting}[language=go, caption={Switch}\label{lst:switch}]
// Switch is a special type of action used for flow-control
// When Element's controller encounters switch action on the chain
// it emulates the work of a switch known in other programming languages
type Switch struct {
	Conditions []Condition `json:"conditions"`
}
\end{lstlisting}

\subsubsection{Condition}
\begin{lstlisting}[language=go, caption={Condition}\label{lst:condition}, basicstyle=\ttfamily\tiny]
// Condition represents a single condition present in Switch action
// It defines on which Data field it has to be performed, 
// the actual condition to be evaluated,
// and the next Action if evaluation returns true.
type Condition struct {
	// Key indicates the Data field that has to be retrieved
	Key string `json:"key"`
	// Operator defines the comparison operation, e.g. eq, ne, gt, lt
	Operator string `json:"operator" kubebuilder:"validation:Enum=eq,ne,gt,lt"`
	// Type specifies the type of the value: string, int, float, bool
	Type string `json:"type" kubebuilder:"validation:Enum=string,int,float,bool"`
	// One of these fields is not null depending on a Type.
	BoolCondition   *BoolCondition   `json:"bool,omitempty" kubebuilder:"validation:Optional"`
	IntCondition    *IntCondition    `json:"int,omitempty" kubebuilder:"validation:Optional"`
	StringCondition *StringCondition `json:"string,omitempty" kubebuilder:"validation:Optional"`
	// Next specifies the name of the next action to execute if evaluation returns true
	Next string `json:"next"`
}
\end{lstlisting}

\subsubsection{BoolCondition}
\begin{lstlisting}[language=go, caption={BoolCondition}\label{lst:boolcondition}]
// BoolCondition defines a boolean-specific condition
type BoolCondition struct {
	Value bool `json:"value"`
}
\end{lstlisting}

\subsubsection{IntCondition}
\begin{lstlisting}[language=go, caption={IntCondition}\label{lst:intcondition}]
// IntCondition defines an integer-specific condition
type IntCondition struct {
	Value int `json:"value"`
}
\end{lstlisting}

\subsubsection{StringCondition}
\begin{lstlisting}[language=go, caption={StringCondition}\label{lst:stringcondition}]
// StringCondition defines a string-specific condition
type StringCondition struct {
	Value string `json:"value"`
}
\end{lstlisting}

\appendix{Specyfikacja interfejsów Lupus}\label{appendix:4}

Specyfikacja w formie elektronicznej znajduje się pod linkiem: \url{https://github.com/0x41gawor/lupus/blob/master/docs/spec/lupin-lupout.md}.

\subsection{Architektura}

\begin{figure}[!h]
    \centering \includegraphics[width=1\linewidth]{a4-arch.png}
    \caption{Architektura Lupus}\label{fig:a4-arch}
\end{figure}

\subsection{Interfejs Lupin}

Projektant może zdefiniować wiele \textbf{Elementów Lupus}, połączonych na różne, skomplikowane sposoby. Musi jednak zdecydować, który z nich zostanie wywołany przez \textbf{IAgenta Ingress}. Taki element można nazwać \textbf{Element Ingress}. Lupus zaleca posiadanie tylko jednego Elementu Ingress.
Jeśli \textbf{Agent Ingress} chce zasygnalizować, że można zaobserwować nowy stan systemu zarządzanego (co oznacza, że musi zostać uruchomiona nowa iteracja pęli sterowania), musi zmodyfikować pole \texttt{Status.Input} w \textit{obiekcie API} \textbf{Elementu Ingress}. Wartość umieszczona w tym polu będzie reprezentować nowy \textbf{Aktualny Stan}.

Pole \texttt{Status.Input} w \textbf{Ingress Element CR} jest typu \texttt{RawExtension}, co oznacza, że podlega pod specyfikacje danych \\TODO załącznik daty

JSON przesłany w tym miejscu będzie stanowił \textbf{Dane} dla tego elementu.

Oprogramowanie implementuje interfejs \texttt{Lupin}, jeśli w pewnym miejscu swojego kodu wysyła żądanie HTTP do \textbf{kube-api-server}, które aktualizuje status \textbf{Elementu Ingress}, a dokładniej pole \texttt{input}. Wartość musi być obiektem JSON, który reprezentuje \textbf{Aktualny Stan} \textbf{Zarządzanego Systemu}. 

\subsection{Interfejs Lupout}

Punktem wyjścia z \textbf{Systemu Sterowania Lupus} jest ostatni \textbf{Element Lupus} czyli (\textbf{Element Egress}). Wysyła on swoje \textbf{finalne dane} (lub ich część) do \textbf{Agenta Egress}. \textbf{Egress Agent} musi przekształcić to wejście w \textbf{Akcje Sterowania}, wykonywaną bezpośrednio na \textbf{systemie zarządzanym}.

Oprogramowanie implementuje interfejs \texttt{Lupout}, jeśli implementuje serwer HTTP, który akceptuje wejściowe dane JSON i tłumaczy je na \textbf{Akcje Sterowania} wykonywaną na \textbf{Systemie Zarządzanym}. 


\appendix{Specfyfikacju obiektu danych}\label{appendix:5}

Specyfikacja w formie elektronicznej znajduje się pod linkiem: \url{https://github.com/0x41gawor/lupus/blob/master/docs/spec/data.md}.

Data jest kluczowym elementem spełnienia \hyperref[req:5]{Wymagania 5}. \textbf{Dane} jest to sposób w jaki \textbf{użytkownik}, podczas każdej iteracji, może:
\begin{itemize}
    \item uzyskać informacje o \textbf{Aktualnym Stanie}
    \item przechowywać pomocnicze informacje (takie jak odpowiedzi od \textbf{Elementów Zewnętrznych})
    \item przechowywać informacje debuggingowe
    \item zapisywać informacje potrzebne do sformułowania \textbf{Akcji Sterowania}
\end{itemize}

\textbf{Dane} są reprezentowane jako JSON dający się zapisać w strukturze Go \texttt{map[string]interface{}}. Nie może, więc być jedną z następujących form obiektu JSON:
\begin{itemize}
    \item typem prymitywnym,
    \item tablicą,
    \item obiektem JSON z kluczami innymi niż string.
\end{itemize}

To w jaki sposób można operować na danych prezentuje specyfikacja akcji //TODO link.
\appendix{Specyfikacja akcji}\label{appendix:6}

Specyfikacja w formie elektronicznej znajduje się pod linkiem: \url{https://github.com/0x41gawor/lupus/blob/master/docs/spec/actions.md}.

Specyfikacja \hyperlink{def:akcja}{\textbf{Akcji}} w \hyperlink{def:lupn}{\textbf{LupN}} znajduje się w \hyperref[appendix:3]{Załączniku 3}. Niniejszy załącznik prezentuje przykładowe działanie \hyperlink{def:akcja}{\textbf{Akcji}} na \hyperlink{def:dane}{\textbf{Danych}}.

\hyperlink{def:akcja}{\textbf{Akcje}} zostały opracowane jako najbardziej atomowe operacje, które, gdy zostaną odpowiednio połączone, stanowią narzędzie umożliwiające \hyperlink{def:projektant}{\textbf{Projektantowi Pętli}} pełne operowanie na \hyperlink{def:dane}{\textbf{Danych}}.

Czasami operacja, która na pierwszy rzut oka wydaje się atomowa, wymaga użycia dwóch połączonych \hyperlink{def:akcja}{\textbf{Akcji}}. Z drugiej strony, zdarza się, że operacja początkowo uznana za atomową okazuje się jedynie szczególnym przypadkiem bardziej ogólnej operacji. Dobrym przykładem jest nieistniejąca już \hyperlink{def:akcja}{\textbf{Akcja}} \texttt{concat}. Została ona zaprojektowana do łączenia dwóch pól w jedno, jednak okazało się, że jest to specyficzny przypadek \hyperlink{def:akcja}{\textbf{Akcji}} \texttt{nest}, w której lista \texttt{InputKey} zawiera tylko dwa elementy.

\subsection{Podział ogólny}

Mamy 8 typów akcji:

\begin{itemize}
    \item \textbf{Send}
    \item \textbf{Nest}
    \item \textbf{Remove}
    \item \textbf{Rename}
    \item \textbf{Duplicate}
    \item \textbf{Insert}
    \item \textbf{Print}
    \item \textbf{Switch}
\end{itemize}

Możemy wyróżnić następujące kategorie:

\begin{itemize}
    \item 6 akcji, które mogą być używane do modyfikacji danych: \\ 
          \texttt{\{Send, Nest, Remove, Rename, Duplicate, Insert\}}
    \item 1 akcja do komunikacji z \textbf{Elementami Zewnętrznymi}: \\ 
          \texttt{\{Send\}}
    \item 2 akcje do debugowania: \\ 
          \texttt{\{Insert, Print\}}
    \item 1 akcja do logowania: \\ 
          \texttt{\{Print\}}
    \item 1 akcja do sterowania przepływem \textbf{workflow akcji}: \\ 
          \texttt{\{Switch\}}
\end{itemize}

\subsection{Przykłady}

Załącznik przedstawi przykładowe użycie 6 akcji, które mogą modyfikować dane.

Każdy przykład zawiera:

\begin{itemize}
    \item reprezentację JSON stanu \textbf{danych} przed modyfikacją akcji,
    \item notację \textbf{LupN} zastosowanej akcji,
    \item reprezentację JSON stanu \textbf{danych} po modyfikacji akcji
\end{itemize}

\subsubsection{Send}
\begin{figure}[!h]
    \centering \includegraphics[width=1\linewidth]{a6-send.png}
    \caption{Przykład modyfikacji danych przez akcje Send}\label{fig:a6-send}
\end{figure}
\subsubsection{Nest}
\begin{figure}[!h]
    \centering \includegraphics[width=1\linewidth]{a6-nest.png}
    \caption{Przykład modyfikacji danych przez akcje Nest}\label{fig:a6-nest}
\end{figure}
\subsubsection{Remove}
\begin{figure}[!h]
    \centering \includegraphics[width=1\linewidth]{a6-remove.png}
    \caption{Przykład modyfikacji danych przez akcje Remove}\label{fig:a6-remove}
\end{figure}
\subsubsection{Rename}
\begin{figure}[!h]
    \centering \includegraphics[width=1\linewidth]{a6-rename.png}
    \caption{Przykład modyfikacji danych przez akcje Rename}\label{fig:a6-rename}
\end{figure}
\subsubsection{Duplicate}
\begin{figure}[!h]
    \centering \includegraphics[width=1\linewidth]{a6-duplicate.png}
    \caption{Przykład modyfikacji danych przez akcje Duplicate}\label{fig:a6-duplicate}
\end{figure}
\subsubsection{Insert}

\input{tex/appendix/7-test-agent-ingress.tex}
\input{tex/appendix/8-test-agent-egress.tex}
\input{tex/appendix/9-test-opa.tex}
\appendix{Kod LupN (model danych pętli sterowania)}\hypertarget{appendix:10}{}

Kod w wersji elektronicznej znajduje się pod linkiem: \url{https://github.com/0x41gawor/lupus/blob/master/examples/open5gs/sample-loop/master.yaml}

\begin{lstlisting}[language=sh, caption={\emph{Kod LupN}}, label={lst:a101}, numbers=left, stepnumber=1]
apiVersion: lupus.gawor.io/v1
kind: Master
metadata:
  labels:
    app.kubernetes.io/name: lupus
    app.kubernetes.io/managed-by: kustomize
  name: lola
spec:
  name: "lola"
  elements:
    - name: "demux"
      descr: "Demuxes Data input into separate elements for each UPF"
      actions: 
        - name: "insert1"
          type: insert
          insert:
            outputKey: "open5gs-upf1"
            value: {name: "open5gs-upf1"}
          next: "insert2"
        - name: "insert2"
          type: insert
          insert:
            outputKey: "open5gs-upf2"
            value: {name: "open5gs-upf2"}
          next: "print"
        - name: "print"
          type: print
          print:
            inputKeys: ["*"]
          next: final
      next:
        - type: element
          element:
            name: "upf1"
          keys: ["open5gs-upf1"]
        - type: element
          element:
            name: "upf2"
          keys: ["open5gs-upf2"]
    - name: "upf1"
      descr: "Reconcilation of UPF1 deployment"
      actions:
        - name: "print1"
          type: print
          print:
            inputKeys: ["*"]
          next: "opa-point"
        - name: "opa-point"
          type: send
          send: 
            inputKey: "*"
            destination: 
              type: opa
              opa: 
                path: http://192.168.56.112:9500/v1/data/policy/point
            outputKey: "point"
          next: "print2"
        - name: "print2"
          type: print
          print:
            inputKeys: ["*"]
          next: "switch1"
        - name: "switch1"
          type: switch
          switch:
            conditions:
              - key: "point"
                operator: eq
                type: string
                string: 
                  value: "NORMAL"
                next: final
          next: "opa-spec"
        - name: "opa-spec"
          type: send
          send: 
            inputKey: "actual"
            destination: 
              type: opa
              opa: 
                path: http://192.168.56.112:9500/v1/data/policy/spec
            outputKey: "spec"
          next: "print3"
        - name: "print3"
          type: print
          print:
            inputKeys: ["*"]
          next: "switch2"
        - name: "switch2"
          type: switch
          switch:
            conditions:
              - key: "point"
                operator: eq
                type: string
                string: 
                  value: "CRITICAL"
                next: final
          next: "print4"
        - name: "print4"
          type: print
          print:
            inputKeys: ["*"]
          next: "opa-interval"
        - name: "opa-interval"
          type: send
          send: 
            inputKey: "point"
            destination: 
              type: opa
              opa: 
                path: http://192.168.56.112:9500/v1/data/policy/interval
            outputKey: "interval"
          next: "print5"
        - name: "print5"
          type: print
          print:
            inputKeys: ["*"]
          next: final
      next: 
        - type: destination
          destination: 
            type: http
            http: 
              path: http://192.168.56.112:9001/api/data
              method: POST
          keys: ["*"]
    - name: "upf2"
      descr: "Reconcilation of UPF2 deployment"
      actions:
        - name: "print"
          type: print
          print:
            inputKeys: ["*"]
          next: "opa-point"
        - name: "opa-point"
          type: send
          send: 
            inputKey: "*"
            destination: 
              type: opa
              opa: 
                path: http://192.168.56.112:9500/v1/data/policy/point
            outputKey: "point"
          next: "print2"
        - name: "print2"
          type: print
          print:
            inputKeys: ["*"]
          next: "switch1"
        - name: "switch1"
          type: switch
          switch:
            conditions:
              - key: "point"
                operator: eq
                type: string
                string: 
                  value: "NORMAL"
                next: final
          next: "opa-spec"
        - name: "opa-spec"
          type: send
          send: 
            inputKey: "actual"
            destination: 
              type: opa
              opa: 
                path: http://192.168.56.112:9500/v1/data/policy/spec
            outputKey: "spec"
          next: "print3"
        - name: "print3"
          type: print
          print:
            inputKeys: ["*"]
          next: "switch2"
        - name: "switch2"
          type: switch
          switch:
            conditions:
              - key: "point"
                operator: eq
                type: string
                string: 
                  value: "CRITICAL"
                next: final
          next: "print4"
        - name: "print4"
          type: print
          print:
            inputKeys: ["*"]
          next: "opa-interval"
        - name: "opa-interval"
          type: send
          send: 
            inputKey: "point"
            destination: 
              type: opa
              opa: 
                path: http://192.168.56.112:9500/v1/data/policy/interval
            outputKey: "interval"
          next: "print5"
        - name: "print5"
          type: print
          print:
            inputKeys: ["*"]
          next: final
      next: 
        - type: destination
          destination: 
            type: http
            http: 
              path: http://192.168.56.112:9001/api/data
              method: POST
          keys: ["*"]
\end{lstlisting}


\textbf{Krótki opis powyższego pliku}

Kod LupN w rzeczywistości jest plikiem manifestowym YAML zasobu \textbf{Master}. Specyfikacje obiektu tego typu zasobu w całości opisuje modelowaną pętle sterowania, dlatego analizę należy rozpocząć od linii 8. W lini 9 widzimy nazwę pętli. W lini 10 zdefiniowany jest obiekt elementów jakie należą do pętli. Omawiana w tym przykładzie pętla ma trzy elementy a ich obiekty znajdziemy odpowiednio w liniach: 11, 40 oraz 128. Skupmy się na elemencie pierwszym o nazwie "demux", którego celem jest rozdzielenie danych na dwie części i przekazanie odpowiednich części do następnych elementów pętli. Jego opis znajduje się między liniami 11 a 39. Linia 13 rozpoczyna opis workflow akcji. Mamy zdefiniowane tu trzy akcje (linie: 14, 20 i 26). Pierwsza akcja dodaje do danych pole \texttt{\{name: "open5gs-upf1"\}} w odpowiedniej sekcji struktury danych (przedstawionej w listingu \ref{lst:521}). Analogiczna akcja jest wykonywana dla drugiej części danych. Celem tej operacji jest nazwanie obu sekcji danych, aby później Agent Egress mógł rozpoznać na rzecz, którego wdrożenia UPF wykonuje akcje sterującą. Ostatnią akcją elementu "demux" jest wyświetlenie zawartości danych w logach operatorach. 

Kiedy workflow akcji dobiegnie końca, dane wysyłane są do destynacji zdefiniowanych przez obiekt specyfikowany w linii 31. Posiada on dwa obiekty next. Oba typu element. Każdemu elementowi przekazywana jest inna sekcja danych, specyfikowana poprzez klucz wskazujący na odpowiednie pole danych.

Kolejny element - "upf1" - odpowiedzialny jest za rekoncyliację wdrożenia UPF 1. Jego workflow akcji jest następujące:
\begin{itemize}
  \item Linia 43: Wyświetlenie zawartości danych
  \item Linia 48: Wysłanie wszystkich danych (operator \texttt{*}) do elementu zewnętrznego jakim jest "POINT"
  \item Linia 58: Ponowne wyświetlenie zawartości danych
  \item Linia 63: Decyzja na podstawie otrzymanej odpowiedzi od elementu zewnętrznego "POINT". Jeśli stan jest normalny, następuje natychmiastowe przejście do końca workflow akcji.
  \item Linia 74: Wysłanie rzeczywistego zużycia zasobów do elementu zewnętrznego "SPEC".
  \item Linia 84: Ponowne wyświetlenie zawartości danych
  \item Linia 89: Decyzja na podstawie pola "point", jeśli jest ono równe "CRITIAL" następuje natychmiastowe przejście do końca workflow akcji
  \item Linia 100: Ponowne wyświetlenie zawartości danych
  \item Linia 105: Wysłanie żądania HTTP z body w postaci pola JSON "point" na adres \texttt{http://192.168.56.112:9500/v1/data/policy/interval}
\end{itemize}

Pomijając akcje logujące stan danych, workflow elementu składa się z:
\begin{enumerate}
  \item trzech akcji typu \texttt{send}, które mimo różnych opisów powyżej różnią się jedynie parametrami wywołania.
  \item dwóch akcji typu \texttt{switch}, które decydują czy następne akcje są istotne w danej iteracji pętli
\end{enumerate} 

Kiedy workflow akcji dobiegnie końca, dane wysyłane są do destynacji zdefiniowanych przez obiekt specyfikowany w linii 120. Tym razem jest ono jednoelementowy i reprezentuje Agenta Egress.

Opis elementu "upf2" zawarty w liniach 128 do 215 jest analogiczny jak w przypadku elementu "upf1".

\end{document} % Dobranoc.
