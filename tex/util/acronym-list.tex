\vspace{0.8cm}
\acronymlist

\hypertarget{def:architektura}{}
\acronym{Architektura}{Zbiór reguł i metod opisujących funkcjonalność, organizację oraz implementację systemu.}

\hypertarget{def:dane}{}
\acronym{Dane}{Fakty oraz statystyk zebrane razem w celu analizy. Stanowią podstawę do wydobycia z ich \hyperlink{def:informacja}{Informacji}}

\hypertarget{def:custom-resource-definitions}{}
\acronym{Definicje Zasobów Własnych (ang. \textit{Custom Resource Definitions})}{Pliki manifestacyjne YAML służące do zarejestrowania w klastrze nowe zasoby własne}

\hypertarget{def:doswiadczenie}{}
\acronym{Doświadczenie (ang. \textit{Experience})}{Proces zdobywania \hyperlink{def:wiedza}{wiedzy} przez system podczas funkcjonowania w środowisku docelowym}.

\hypertarget{def:edukacja}{}
\acronym{Edukacja (ang. \textit{Education})}{Proces zdobywania \hyperlink{def:wiedza}{wiedzy} przez system poza środowiskiem docelowym, np. podczas trenowania modelu}.

\hypertarget{def:wiedza}{}
\acronym{Wiedza (ang. \textit{Knowledge})}{Zestaw wzorców, które są wykorzystywane do wyjaśniania, a także przewidywania tego, co się wydarzyło, dzieje lub może się wydarzyć w przyszłości. Bazuje na danych, informacjach oraz umiejętnościach zdobytych poprzez \hyperlink{def:doswiadczenie}{doświadczenie} oraz \hyperlink{def:edukacja}{edukację}}

\hypertarget{def:eni}{}
\acronym{ENI}{Experiential Networked Intelligence}

\hypertarget{def:etsi}{}
\acronym{ETSI}{Europen Telecommunications Standards Institute}

\hypertarget{Informacja}{Reprezentacja pojęć będących przedmiotem zainteresowania środowiska w formie niezależnej od formy \hyperlink{def:dane}{danych}}

\hypertarget{def:kognitywnosc}{}
\acronym{Kognitywność (ang. \textit{cognition})}{Proces rozumienia danych oraz informacji w celu produkcji nowych danych, informacji oraz wiedzy}

\hypertarget{def:kognitywny-system-zarzadzania-siecia}{}
\acronym{Kognitywny System Zarządzania Siecią}{System zarządzania siecią, który jest jednocześnie \hyperlink{def:system-kognitywny}{kognitywny}. Docelowy system specyfikowany przez ENI}

\hypertarget{def:kontroler}{}
\acronym{Kontroler (ang. \textit{controller})}{Proces monitorujący stan obiektów API w klastrze i doprowadzający je do stanu pożądanego}

\hypertarget{def:obiekt-api}{}
\acronym{Obiekt API (ang. \textit{API object})}{Zasób reprezentujący element klastra, np. Poda, Deployment, Service. Jest przechowywany w etcd i zarządzany przez API Server. Jest to konkretna instancja danego typu (ang. \textit{Kind}) zasobów}

\hypertarget{def:plik-manifestowy}{}
\acronym{Plik Manifestowy YAML (ang. \textit{YAML Manifest File})}{Plik konfiguracyjny zapisany w formacie YAML, definiujący obiekty API Kubernetes. Służy do deklaratywnego zarządzania zasobami klastra}

\hypertarget{def:polityka}{}
\acronym{Polityka (ang. \textit{Policy})}{Zestaw reguł, który jest używany do zarządzania i kontrolowania zmiany i/lub utrzymania stanu jednego lub więcej zarządzanych obiektów}

\hypertarget{def:regulator}{}
\acronym{Regulator (ang. \textit{Control System})}{Pojęcie z teorii sterowania. System, który reguluje pracę obiektu}

\hypertarget{def:data-driven}{}
\acronym{Sterowany danymi (ang. \textit{Data-Driven})}{Nie narzucający żadnej logiki pętli – logika jest interpretowana na podstawie danych.}

\hypertarget{def:system-eni}{}
\acronym{System ENI}{Centralny punkt architektury ENI. Odpowiedzialny za zamkniętą pętlę sterowania, za pomocą której ENI chce osiągnąć kognitywność. Lupus aspiruje do tego, aby zaproponować architekturę dla Systemu ENI na Kubernetes}

\hypertarget{def:system-kognitywny}{}
\acronym{System kognitywny}{System, który uczy się, wnioskuje oraz podejmuje decyzje w sposób przypominający ludzki umysł}

\hypertarget{def:uczenie-maszynowe}{}
\acronym{Uczenie maszynowe (ang. \textit{Machine Learning})}{proces, który zdobywa nową wiedzę i/lub aktualizuje istniejącą wiedzę w celu optymalizacji funkcjonowanie systemu przy użyciu przykładowych obserwacji}

\hypertarget{def:swiadomosc-kontekstu}{}
\acronym{Świadomość kontekstu (ang. \textit{Context Aware})}{Posiadanie przez system informacji oraz wiedzy, które opisują środowisko w jakim znajduje się dana jednostka w celu lepszego doboru \hyperlink{def:wzorzec}{wzorca} do rozwiązania danego problemu}

\hypertarget{def:wzorzec}{}
\acronym{Wzorzecz (ang. \textit{pattern})}{Generyczne, reużywalne rozwiązanie danego problemu. Część składowa wiedzy}.

\hypertarget{def:warstwa-sterowania}{}
\acronym{Warstwa Sterowania (ang. \textit{Control Plane})}{Zestaw komponentów zarządzających stanem klastra Kubernetes. W szczególności \textsl{Controller Manager}, który nadzoruje kontrolery zasobów}

\hypertarget{def:workflow}{}
\acronym{Workflow}{Sekwencja połączonych węzłów, czasami zależnych warunkowo, która realizuje określony cel. Zazwyczaj workflow definiuje się w celu organizacji pracy.}

\hypertarget{def:wnioskowanie}{}
\acronym{Wnioskowanie (ang. \textit{reasoning})}{Proces, w którym system wyciąga logiczne wnioski z dostępnych danych i wiedzy}

\hypertarget{def:wzorzec-operator}{}
\acronym{Wzorzec Operator (ang. \textit{Operator Pattern})}{Rozszerzenie Kubernetes pozwalające rozwijać kontrolery dla zasobów własnych}

\hypertarget{def:zasoby-wlasne}{}
\acronym{Zasoby Własne (ang. \textit{Custom Resources})}{Mechanizm rozszerzenia klastra Kubernetes o nowe typy obiektów API}