\vspace{0.8cm}
\acronymlist

\acronym{ETSI}{Europen Telecommunications Standards Institute}

\acronym{ENI}{Experiential Networked Intelligence}

\acronym{Architektura}{Zbiór reguł i metod opisujących funkcjonalność, organizację oraz implementację systemu.}\label{def:architektura}

\acronym{Workflow}{Sekwencja połączonych węzłów, czasami zależnych warunkowo, która realizuje określony cel. Zazwyczaj workflow defniuje się w celu organizacji pracy.}\label{def:workflow}

\acronym{Wnioskowanie (ang. \textit{reasoning})}{Proces, w którym system wyciąga logiczne wnioski z dostępnych danych i wiedzy}\label{def:wnioskowanie}

\acronym{Kognitywność (ang. \textit{cognition})}{Proces rozumienia danych oraz informacji w celu produkcji nowych danych, informacji oraz wiedzy}\label{def:kognitywność}

\acronym{System kognitywny}{System, który uczy się, wnioskuje oraz podejmuje decyzje w sposób przypominający ludzki umysł}\label{def:system-kognitywny}

\acronym{System ENI}{}

\acronym{Regulator (ang. \textit{Control System})}{}

\acronym{Warstwa Sterowania (ang. \textit{Control Plane})}{}

\acronym{Wzorzec Operator (ang. \textit{Operator Pattern})}{}

\acronym{Definicje Zasobów Własnych (ang. \textit{Custom Resource Definitions})}{}

\acronym{Zasoby Własne (ang. \textit{Custom Resources})}{}

\acronym{Sterowany danymi (ang. \textit{Data-Driven})}{}

\acronym{kontrolera (ang. \textit{controller})}{}

\acronym{obiekt API (ang. \textit{API object})}{}

