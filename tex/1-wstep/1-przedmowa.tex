\subsection{Przedmowa}

Wraz z rozwojem telekomunikacji stopień jej skomplikowania jak i mnogość podłączonych urządzeń stale rośnie. Sieci 5G zwiastują obsługę miliardów urządzeń, co sprawia, że tradycjne podejście do zarządzania sieciami staje się niewystarczające. W pewnym momencie manualnie operowanie sieciami (ang. \textit{humna-driven networks}) stanie się wręcz niemożliwe. Dlatego obserwujemy obecnie zwrot w stronę wirtualizacji oraz automatyzacji sieci. Jednocześnie dynamiczny rozwój sztucznej inteligencji otwiera nowe możliwość. Te dwa czynniki stanowią wspólnie świetny fundament do tego, aby branża sieci telekomunikacyjnych postawiła sobie za cel budowę "inteligentnych" sieci - takich, które są w pełni autonomiczne, samowystarczalne oraz niewymagają nadzoru ludzkiego. 

W tym celu ETSI (European Telecommunications Standards Institute) powołało komitet o nazwie ENI - Experiential Networked Intelligence, który ma na celu wypracowanie specyfikacji dla Kognitywnych Systemów Zarządzania Siecią (Cognitive Network Management system). Kognitywny system oznacza taki, który jest w stanie uczyć się i podejmować decyzje bazujące na zebranej wiedzy w sposób przypominający ludzki umysł. ENI opiera swoją architekturę na zamkniętych pętlach sterowania.

Pętlą sterowania, ENI nazywa mechanizm, który monitoruje wydajność systemu lub procesu podddawanego kontroli w celu osiągnięcia pożądanego zachowania. Innymi słowy, pętla sterowania reguluje działanie zarządzanego obiektu. Pętle sterowania można podzielić na zamknięte lub otwarte w zależności od tego czy działanie sterujące zależy od sprzężenia zwrotnego z kontrolowanego systemu lub obiektu. Jeśli tak, pętle nazywamy zamkniętą, jeśli nie - otwartą. 

W przypadku architektury ENI, zamknięta pętla sterowania służy jako model organizacji pracy (ang. textit{workflow}) elementów odpowiedzialnych za sztuczną inteligencję. W dokumencie \cite*{enioverview} ENI dokonało przeglądu obiecujących architektur zamkniętych pętli sterowania znanych ludzkości. Naturalnym następnym krokiem jest zaproponowanie platformy, na której można takowe pętle zamodelować oraz uruchomić. 