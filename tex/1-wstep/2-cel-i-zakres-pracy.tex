\subsection{Cel i zakres pracy}

Celem pracy jest kontynuacja badań ETSI, polegająca na zaproponowaniu \hyperlink{def:architektura}{\textit{architektury}} platformy, na której możliwe będzie modelowanie, uruchamianie oraz zarządzanie zamkniętymi pętlami sterowania. Niniejsza praca nie koncentruje się na aspektach związanych ze sztuczną inteligencją. Platforma ma jedynie służyć do modelowania, uruchamiania i zarządzania przepływem pracy pętli (ang. \hyperlink{def:workflow}{\textit{workflow}}), ale sama nie stanowi środowiska wykonawczego dla jej komponentów.

W zakres pracy wchodzi: przegląd powiązanej literatury, sformułowanie wymagań dla platformy, opracowanie jej architektury, implementacja PoC (ang. \textit{Proof of Concept}), przeprowadzenie testów oraz analiza potencjału platformy w kontekście dalszego rozwoju. Po publikacji praca może stanowić podstawę do implementacji \hyperlink{def:kognitywny-system-zarzadzania-siecia}{Kognitywnych Systemów Zarządzania Siecią}, zgodnych ze specyfikacjami ETSI.

