\subsection{Cel i zakres pracy}

Celem pracy jest kontynuacja badań ENI, polegająca na zaproponowaniu \hyperlink{def:architektura}{\textit{architektury}} platformy, na której możliwe będzie modelowanie oraz uruchamianie zamkniętych pętli sterowania. Niniejsza praca nie skupia się na aspektach związanych ze sztuczną inteligencją. Platforma ma jedynie służyć do zamodelowania oraz uruchomienia \hyperlink{def:workflow}{\textit{workflow}} pętli, ale sama nie stanowi środowiska wykonawczego dla jej komponentów. 

W zakres pracy wchodzi: sformułowanie wymagań na platformę, opracowanie proponowanej architektury wsparte licznymi badaniami podczas jej powstawania, implementacja PoC (ang. \textit{Proof of Concept}) według proponowanej architektury, przeprowadzenie testów platformy oraz analiza jej potencjału w kontekście dalszego rozwoju. Po publikacji praca może stanowić podstawę do implementacji \hyperlink{def:kognitywny-system-zarzadzania-siecia}{Kognitywnych Systemów Zarządzania Siecią}, zgodnych ze specyfikacjami ENI. 