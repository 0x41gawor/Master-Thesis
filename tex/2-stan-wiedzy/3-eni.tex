\subsection{ENI}

ENI rozwija specyfikacje w kierunku Kognitywnego Systemu Zarządzania, który będzie regulował działanie sieci (ułatwiając przy tym pracę operatora) poprzez wykorzystanie technik sztucznej inteligencji, takich jak uczenie maszynowe (ang. \textit{machine learning}) oraz wnioskowanie (ang. \textit{reasoning}). 

Z poprzedniej sekcji wiemy, że aby \textit{wnioskować} potrzebna jest wiedza. Skąd taki system pozyskiwałby wiedzę? Otóż, dzieje się to poprzez proces \textit{machine learning}'u. Pierwsza faza pozyskiwania wiedzy dzieje się podczas tzw. \textit{treningu}, jeszcze przed wdrożeniem systemu. Ale system również może uczyć się "z doświadczenia" (ang. \textit{experience}) po wdrożeniu, kiedy już jest w pełni operacyjny. Stąd mówimy o empirycznej inteligencji sieciowej (ang. \textit{experiential networked intelligence}). Czyli z jednej strony wiedza potrzebna do wnioskowania, a w ostateczności podejmowania decyzji płynęłaby z samej sieci, co w przypadku telefonistki jest równoważne przychodzącemu połączeniu. Ale telefonistka ma również niejako wbudowaną w siebie wiedzę o tym, jak zorganizowane są łącza na przełącznicy komutacyjnej. Tak samo sieć musi mieć pewną części wiedzy nieco narzuconą z góry, która stanowi jako zestaw reguł używanych do zarządzania i kontrolowania stanu sieci. Takie reguły nazywamy \textit{politykami}. Mogą one płynąć od: aplikacji zarządzających siecią, użytkowników sieci, systemów OSS/BSS lub Orkiestratora. Ważne jest to, aby polityki były \textit{świadome kontekstu}. Pozwoli to na stworzenie systemu kognitywnego, czyli takiego, który uczy się, wnioskuje oraz podejmuje decyzje w sposób przypominający ludzki umysł. Taki system z kolei jest w stanie w dużym stopniu odciążyć operatora i zautomatyzować zadania takie jak: dynamiczne przydzielanie zasobów (ang. \textit{dynamic resource allocation}), równoważenie obciążenia (ang. \textit{load balancing}), zarządzanie wydajnością energetyczną (ang. \textit{energy efficiency management}, samo naprawiające się sieci (ang. \textit{self-healing networks}), optymalizacja jakości doświadczeń użytkowników (ang. \textit{QoE optimization}), egzekwowanie polityk (ang. \textit{policy enforcement}), zapewnienie zgodności z regulacjami (ang. \textit{regulatory compliance}) i wiele innych.

