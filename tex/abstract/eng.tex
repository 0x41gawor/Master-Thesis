The project described in this thesis focuses on proposing and implementing a reusable architecture (hereafter referred to as "the Platform") that enables the modeling and execution of closed control loops in Kubernetes. The genesis of the project lies in the work of one of the ETSI committees called "ENI - Experiential Networked Intelligence," which aims to simplify the work of telecommunications network operators by leveraging artificial intelligence mechanisms in closed control loops based on metadata-driven and context-aware policies. In one of its documents, ENI reviews closed control loops known to humanity from other fields. A natural next step is to propose a platform on which operators could design and execute such loops. 

To achieve this, a set of requirements and assumptions for such a system was defined. Kubernetes was chosen as the runtime environment due to its widespread adoption in the community and its inherent use of closed control loops. An extensive analysis was conducted on how Kubernetes extension mechanisms, such as "Custom Resources" and the "Operator" pattern, could be used to create a framework enabling the modeling of closed control loops. 

This thesis describes the developed platform, its architecture, the semantics of syntax in the defined objects, operational principles, integrations with external systems, and a user guide. It also discusses the platform's implementation, the technologies behind it, and the decisions made during its development. Finally, the thesis presents a practical test of the platform's functionality using the Open5GS 5G system emulator in combination with UERANSIM. The work concludes with a list of findings and potential extensions or improvements to the platform.