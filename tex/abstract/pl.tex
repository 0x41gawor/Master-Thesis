Niniejszy projekt jest poświęcony opracowaniu architektury i implementacji platformy, która pozwala na modelowanie, uruchamianie oraz zarządzanie zamkniętymi pętlami sterowania w Kubernetes. Tematyka pracy jest powiązana z wynikami prac standaryzacyjnych komitetów ETSI ZSM czy ETSI ENI, które skupiają się na specyfikacji autonomicznych oraz kognitywnych systemów zarządzania sieciami telekomunikacyjnymi. 

W tym też kontekście w pracy skupiono się na wdrożeniowym aspekcie takich systemów proponując rozwiązanie pozwalające na swobodne definiowanie architektury autonomicznej pętli, otwarte na integrację z zewnętrznymi silnikami polityk i aplikacjami analitycznymi w celu osadzenia w pętli wymaganej logiki decyzyjnej oraz bazującej na mechanizmach implementacyjnych dostępnych w środowisku Kubernetes.

W pracy opisano przegląd powiązanej literatury, powstałą platformę, jej architekturę oraz instrukcję użytkowania. Omówiona została przykładowa implementacja platformy, technologie za nią stojące oraz decyzje podjęte podczas jej powstawania. Finalnie przedstawiono również test działania platformy w praktyce. Pracę podsumowuje lista wniosków oraz potencjalnych dróg rozwoju platformy. 