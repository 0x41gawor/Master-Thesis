\subsection{Wstęp}

Niniejszy rozdział przedstawia zaproponowaną w pracy architekturę  platformy, na której możliwe jest modelowanie oraz uruchamianie workflow zamkniętych pętli sterowania. Platformie nadano nazwę w celu ułatwienia jej opisu. Nazwa brzmi "Lupus". Powstała od przekształcenia angielskiego słowa "loops" oznaczającego pętle, oraz od zakotwiczenia o wyraz mający znaczenie nadające się na "maskotkę" projektu. "Lupus" po łacinie oznacza wilka, stąd w logo projektu wilk. 

Architektura w rozumieniu ENI jest to "zbiór reguł i metod opisujących funkcjonalność, organizację oraz implementację systemu". Niniejszy rozdział pomija aspekt implementacji, która jest omówiona w następnym rozdziale.

\begin{figure}[h]
    \centering
    \includegraphics[width=1\linewidth]{3-logo.png}
    \caption{Logo platformy Lupus. Autor: Natalia Ruszkowska, na potrzeby niniejszej pracy.}
    \label{fig:logo}
\end{figure}