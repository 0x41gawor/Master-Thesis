\subsection{Instrukcja Użycia}

Niniejsza podsekcja prezentuje krótką instrukcję użycia platformy Lupus. \textbf{Użytkownikiem} Lupus może stać się dowolna organizacja bądź pojedyncza osoba. Ważne jest aby w zespole użytkownik posiadał kompetencje z zakresu tworzenia oprogramowania. W zespole \textbf{Użytkownika} wyróżniamy rolę \textbf{Designera}, który jest odpowiedzialny jedynie za projektowanie i wyrażanie \textbf{Workflow Pętli}, niekoniecznie też musi posiadać umiejętności techniczne. 

Kiedy dany \textbf{Użytkownik} planuje użyć Lupus jako \textbf{Systemu Kontroli} do rozwiązywania \textbf{Problemu Zarządzania} w swoim \textbf{Systemie Zarządzanym}, powinien:
\begin{enumerate}
    \item Zainstalować Lupus w swoim klastrze
    \item Zintegrować \textbf{Zarządzany System} z Lupusem poprzez implementację \textbf{Agentów Translacji}
    \item Zaplanować \textbf{Workflow Pętli} (w dowolnym narzędziu)
    \item Przygotować \textbf{Elementy Zewnętrzne} 
    \item Wyrazić \textbf{Workflow Pętli} w \textbf{LupN} 
    \item Zaaplikować pliki manifestacyjne zawierające kod \textbf{LupN} w klastrze
\end{enumerate}

