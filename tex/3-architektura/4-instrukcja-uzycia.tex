\subsection{Instrukcja Użycia}

Niniejsza podsekcja prezentuje krótką instrukcję użycia platformy \hyperlink{def:lupus}{\textbf{Lupus}}. \hyperlink{def:uzytkownik}{\textbf{Użytkownikiem}} \hyperlink{def:lupus}{\textbf{Lupus}} może stać się dowolna organizacja bądź pojedyncza osoba. Ważne jest, aby zespół \hyperlink{def:uzytkownik}{\textbf{Użytkownika}} posiadał kompetencje z zakresu tworzenia oprogramowania. W zespole \hyperlink{def:uzytkownik}{\textbf{Użytkownika}} wyróżniamy rolę \hyperlink{def:projektant}{\textbf{Projektanta}}, który jest odpowiedzialny jedynie za projektowanie i wyrażanie \hyperlink{def:workflow-petli}{\textbf{Workflow Pętli}} oraz niekoniecznie – zgodnie z \hyperref[req:8]{Wymaganiem 8} – musi posiadać umiejętności techniczne. 

Kiedy dany \hyperlink{def:uzytkownik}{\textbf{Użytkownik}} planuje użyć \hyperlink{def:lupus}{\textbf{Lupus}} jako \hyperlink{def:system-sterowania}{\textbf{Systemu Sterowania}} do rozwiązywania \hyperlink{def:problem-zarzadzania}{\textbf{Problemu Zarządzania}} w swoim \hyperlink{def:system-zarzadzany}{\textbf{Systemie Zarządzanym}}, powinien:
\begin{enumerate}
    \item Zainstalować \hyperlink{def:lupus}{\textbf{Lupus}} w swoim klastrze Kubernetes.
    \item Zintegrować \hyperlink{def:system-zarzadzany}{\textbf{System Zarządzany}} z \hyperlink{def:lupus}{\textbf{Lupusem}} poprzez implementację \hyperlink{def:agent-translacji}{\textbf{Agentów Translacji}}.
    \item Zaplanować \hyperlink{def:workflow-petli}{\textbf{Workflow Pętli}}.
    \item Przygotować \hyperlink{def:element-zewnetrzny}{\textbf{Elementy Zewnętrzne}}.
    \item Wyrazić \hyperlink{def:workflow-petli}{\textbf{Workflow Pętli}} w \hyperlink{def:lupn}{\textbf{LupN}}.
    \item Zaaplikować pliki manifestacyjne zawierające kod \hyperlink{def:lupn}{\textbf{LupN}} w klastrze.
\end{enumerate}

Podjęcie przez użytkownika takich akcji nazywamy pojedynczym \hyperlink{def:wdrozenie-lupus}{\textbf{Wdrożeniem Lupus}}.

\subsubsection{Instalacja Lupus}

\hyperlink{def:lupus}{\textbf{Lupus}} jest zaimplementowany jako \textit{Zasoby Własne} (ang. \textit{Custom Resources}) w Kubernetes. Instalacja polega na zainstalowaniu tych zasobów w swoim klastrze. 

Pełna dokumentacja tego procesu znajduje się w \hyperref[appendix:2]{Załączniku 2}.

\subsubsection{Implementacja Agentów Translacji}

To \hyperlink{def:uzytkownik}{\textbf{Użytkownik}} podczas \hyperlink{def:wdrozenie-lupus}{\textbf{Wdrożenia}} jest odpowiedzialny za implementację \hyperlink{def:agent-translacji}{\textbf{Agentów Translacji}}. Tylko \hyperlink{def:uzytkownik}{\textbf{Użytkownik}} zna specyfikę swojego \hyperlink{def:system-zarzadzany}{\textbf{Systemu Zarządzanego}}. Stąd w zespole \hyperlink{def:uzytkownik}{\textbf{Użytkownika}} potrzebne są umiejętności programistyczne. 

Podczas implementacji należy kierować się specyfikacją \hyperlink{def:interfejsy-lupus}{\textbf{Interfejsów Lupus}} zawartą w \hyperref[appendix:4]{Załączniku 4}.

\subsubsection{Planowanie Logiki Pętli}

\hyperlink{def:workflow-petli}{\textbf{Workflow Pętli}} powinno wyrazić \hyperlink{def:logika-petli}{\textbf{Logikę Pętli}}, czyli w każdej iteracji doprowadzić \hyperlink{def:system-zarzadzany}{\textbf{System Zarządzany}} do \hyperlink{def:stan-pozadany}{\textbf{Stanu Pożądanego}}. \hyperlink{def:uzytkownik}{\textbf{Użytkownik}} w tym kroku jedynie modeluje wysokopoziomowo \hyperlink{def:workflow-petli}{\textbf{Workflow Pętli}}, rysując np. jego diagram. Dopiero takie spojrzenie podpowie \hyperlink{def:uzytkownik}{\textbf{Użytkownikowi}}, jakich \hyperlink{def:element-zewnetrzny}{\textbf{Elementów Zewnętrznych}} potrzebuje.

\subsubsection{Przygotowanie Elementów Zewnętrznych}

\hyperlink{def:element-zewnetrzny}{\textbf{Elementem Zewnętrznym}} może być dowolne oprogramowanie, które \hyperlink{def:projektant}{\textbf{Projektant}} ma chęć włączyć w \hyperlink{def:workflow-petli}{\textbf{Workflow Pętli}}. Mogą to być już gotowe systemy (np. sztucznej inteligencji), ale równie dobrze \hyperlink{def:uzytkownik}{\textbf{Użytkownik}} może stworzyć oprogramowanie samodzielnie. Ważne jest to, aby wystawiały one jakiś sposób komunikacji. Na razie jedynym wspieranym przez \hyperlink{def:lupus}{\textbf{Lupus}} sposobem jest komunikacja HTTP. \hyperlink{def:uzytkownik}{\textbf{Użytkownik}} musi umożliwić więc komunikację tego typu.

Rekomendowanym przez \hyperlink{def:lupus}{\textbf{Lupus}} typem \hyperlink{def:element-zewnetrzny}{\textbf{Elementu Zewnętrznego}} jest serwer \hyperlink{def:opa}{\textbf{Open Policy Agent}}. Przygotowanie w tym wypadku polega na uruchomieniu takiego serwera oraz wgraniu mu odpowiednich polityk. \hyperlink{def:uzytkownik}{\textbf{Użytkownik}} jednakże może wydewelopować dowolny serwer HTTP.

Ostatnim możliwym \hyperlink{def:element-zewnetrzny}{\textbf{Elementem Zewnętrznym}} są \hyperlink{def:funkcje-uzytkownika}{\textbf{Funkcje Użytkownika}}. Są to funkcje w kodzie \textit{kontrolera} zasobu \hyperlink{def:element-lupus}{\textbf{Lupus Element}}, które są definiowane przez \hyperlink{def:uzytkownik}{\textbf{Użytkownika}}. Stanowią one alternatywę dla serwerów HTTP, gdy specjalne wdrażanie takowych może się okazać zbyt kosztowne. Przykładowo, jeśli logika wykonywana przez dany \hyperlink{def:element-zewnetrzny}{\textbf{Element Zewnętrzny}} miałaby mieć tylko kilka linijek kodu, dużo łatwiej użyć \hyperlink{def:funkcje-uzytkownika}{\textbf{Funkcji Użytkownika}}. Ich dokładniejszy opis znajduje się w \hyperref[sec:funkcje-uzytkownika]{Sekcji \ref{sec:funkcje-uzytkownika}}.


\subsubsection{Wyrażenie workflow pętli}

Gdy już całe \hyperlink{def:workflow-petli}{\textbf{Workflow Pętli}} oraz \hyperlink{def:element-zewnetrzny}{\textbf{Elementy Zewnętrzne}} są gotowe, czas wyrazić je w notacji \hyperlink{def:lupn}{\textbf{LupN}}. Za jej pomocą wyraża się \hyperlink{def:workflow-petli}{\textbf{Workflow Pętli}} jako zbiór \hyperlink{def:element-lupus}{\textbf{Elementów Lupus}}, połączenia między nimi oraz specyfikacja każdego z nich.

\subsubsection{Aplikacja plików manifestacyjnych}

Aby stworzyć pętlę opisaną przez \hyperlink{def:kod-lupn}{\textbf{Kod LupN}} w pliku manifestacyjnym zasobu \hyperlink{def:master}{\textbf{Master}}, należy wykonać komendę z listingu \ref{lst:2}, która tworzy instancję \textit{obiektu API} typu \hyperlink{def:master}{\textbf{Master}}. Kontroler tego zasobu stworzy odpowiednie obiekty API typu \hyperlink{def:element}{\textbf{Element}} według specyfikacji w pliku \hyperlink{def:plik-lupn}{\textbf{LupN}}. Podczas iteracji pętli, kontroler każdego \hyperlink{def:element}{\textbf{Elementu}} interpretuje ich specyfikację, wykonując odpowiednie \hyperlink{def:akcja}{\textbf{Akcje}} na \hyperlink{def:dane}{\textbf{Danych}}. \hyperlink{def:workflow-petli}{\textbf{Workflow Akcji}} również specyfikowane jest w notacji \hyperlink{def:lupn}{\textbf{LupN}} (\hyperref[appendix:3]{Załącznik 3}).

\begin{lstlisting}[language=sh, caption={\emph{Stworzenie zasobu Master}}\label{lst:2}]
    kubectl apply -f <nazwa_pliku>
\end{lstlisting}
