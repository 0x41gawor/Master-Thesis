\subsection{Wymagania i założenia}

Po pierwsze Lupus ma przyjąć rolę regulatora (ang. {control system}) znanego z teorii sterowania. Regulowanymi systemami mają być w tym przypadku systemy telekomunikacyjne. Aby odegrać rolę ENI System, platforma musi być w stanie zamodelować oraz uruchomić \underbar{dowolną} zamkniętą pętle sterowania, zwłaszcza te zawarte w \cite{enioverview}. 

Z racji ogólności systemu oraz zwiększenia szansy na dobre przyjęcie w społeczności wybrano Kubernetes jako infrastrukturę dla platformy. Kubernetes natywnie używa zamkniętych pętli sterowania w swojej warstwie sterowania (ang. \textit{control plane}), z których użytkownik jest w stanie skorzystać za pomocą mechanizmów rozszerzeń Kubernetes takich jak Custom Resource Definitions (CRD) oraz Operator Pattern. Są to mechanizmy dobrze znane w branży. 

Wybór Kubernetes narzuca jednakże pewne ograniczenie. Regulowane procesy nie mogą odbywać się w czasie rzeczywistym. Warstwa sterowania Kubernetes działa nieco wolniej. 

Lupus musi być "data-driven" co na polski można przetłumaczyć jako "napędzany danymi". Oznacza to, że platforma nie może narzucać żadnej postaci logiki pętli. Warstwa sterowania Kubernetes musi być w stanie interpretować zamiary użytkownika platformy, który może wyrazić dowolną pętle. Zamiary te wyrażone są właśnie w danych. 

Logikę pętli możemy podzielić na dwie części: workflow pętli oraz części obliczeniowa. Workflow jest to zdefiniowane elementów oraz relacji między nimi. Częścią obliczeniową za to nazywamy procesowanie wykonywane przez konkretne elementy. Z racji podejścia "data-driven" nie możemy zaszyć części obliczeniowej w warstwie sterowania Kubernetes. Dlatego elementy odpowiedzialne za części obliczeniową są "na zewnątrz" pętli Lupus, przykładowo są to serwery HTTP, do których Lupus wykonując pętle może się odwołać na rzecz jej logiki.

Z tego opisu powstaje garść wymagań oraz założeń, które z pewnością nie są zbiorami rozłącznymi i wielu miejscach się zacierają, jednakże warto wyszczególnić je w sposób wylistowany poniżej, aby móc łatwiej się do nich odwoływać w dalszej części pracy:

\begin{enumerate}
    \item \label{req:1} Lupus jest skierowany do branży telekomunikacyjnej.
    \item \label{req:2} Lupus umożliwia modelowanie i uruchamianie dowolnych architektur zamkniętych pętli sterowania, w szczególności tych zaproponowanych w \cite{enioverview}.
    \item \label{req:3} Lupus zarządza procesami, które nie wymagają regulacji w czasie rzeczywistym (są "non-realtime").
    \item \label{req:4} Lupus jest zaimplementowany na bazie Kubernetes, wykorzystując jego \textit{Controller Pattern}.
    \item \label{req:5} Lupus jest oparty na danych (\textit{data-driven}), co oznacza, że nie narzuca i nie ma wbudowanej żadnej domyślnej logiki pętli.
    \item \label{req:6} Faktyczne komponenty przetwarzające w pętli (część obliczeniowa) Lupus są zewnętrzne względem niego (np. serwery HTTP, szczególnie Open Policy Agent).
    \item \label{reg:7} Lupus powinien być w stanie regulować pracą dowolnego systemu teleinformatycznego bez żadnych jego modyfikacji
\end{enumerate}

Powyższa lista nazwana jest Wymaganiami i referowana w dalszej części dokumentu. 
