\subsection{Wymagania i założenia}\label{sec:32}

%\subsubsection{Wstęp}

Niniejszy podrozdział opisuje proces formułowania wymagań i założeń dotyczących platformy Lupus, ukazując przy tym aspekt badawczy pracy. Kolejne podrozdziały (od \ref{sec:33}) przedstawiają docelową architekturę, abstrahując od poniższej dyskusji. Taki podział ma na celu ułatwienie odbioru treści przez czytelnika. Opis w tym podrozdziale opiera się na analizach architektur systemów zarządzania i pętli sterowania, przeprowadzonych w poprzednim rozdziale.

\subsubsection{Wymagania ogólne}

Obie architektury referencyjne \hyperlink{sec:zsm}{ETSI ZSM} oraz \hyperlink{sec:cladra}{CLADRA} definiują je jako frameworki umożliwiające tworzenie wielu równolegle działających zamkniętych pętli sterowania złożonych z różnych komponentów systemu zarządzania. Niniejsza praca proponuje architekturę platformy umożliwiającą modelowanie i uruchamianie takich pętli. Pełniąc te dwie role, platforma powinna również zapewniać funkcję Menedżera Zamkniętych Pętli, zdefiniowanego przez architekturę CLADRA, który nimi zarządza. Na tej podstawie można wyróżnić trzy główne grupy wymagań funkcjonalnych:
\begin{itemize}
    \item wymagania dotyczące środowiska wykonawczego (ang. \textit{runtime environment}), 
    \item wymagania określające, jakie architektury pętli powinny być możliwe do modelowania,
    \item wymagania określające funkcje zarządzania pętlami, które należy zapewnić
\end{itemize}

Ponieważ pętle sterowania wykorzystywane są w wielu branżach, już na tym etapie można określić wymaganie, aby platforma była dedykowana branży teleinformatycznej.

\subsubsection{Wymagania na środowisko wykonawcze}

Platforma musi być zgodna z architekturą referencyjną ETSI ZSM oraz CLADRA, a co za tym idzie – kompatybilna z ich architekturą bazową Open Digital Architecture (ODA) \cite{tmforum2018}. Powinna również wspierać \hyperlink{sec:zsm}{dwanaście pryncypiów architektury ZSM}. Ponadto, aby wpisać się w aktualne trendy inżynierii oprogramowania, platforma powinna być gotowa do wdrożenia w środowisku chmur obliczeniowych (ang. cloud-native). Zamiast budować własne środowisko, platforma powinna bazować na już istniejących rozwiązaniach, co pozwoli uniknąć dodatkowych kosztów wdrożenia. Istotnym atutem byłoby wykorzystanie środowiska dobrze znanego w branży teleinformatycznej, co zwiększyłoby szanse na szeroką adopcję w społeczności. Na tej podstawie formułujemy następujące wymagania:

\begin{enumerate}
    \setcounter{enumi}{0} 
    \item \label{req:1} Środowisko wykonawcze powinno być dobrze znane w branży teleinformatycznej
    \item \label{req:2} Środowisko wykonawcze powinno wspierać \hyperlink{list:2}{12 pryncypiów architektury ZSM}
    \item \label{req:3} Środowisko wykonawcze powinno być \textit{cloud-native}
\end{enumerate}

\subsubsection{Wymagania na modelowanie pętli}

Punktem wyjścia jest możliwość zamodelowania dowolnej zamkniętej pętli sterowania, przeanalizowanej w \hyperlink{sec:25}{podrozdziale \ref{sec:25}}. Kluczowym wymaganiem jest zapewnienie mechanizmów feedback i feedforward, co oznacza konieczność obsługi komunikacji pomiędzy elementami pętli. Ponadto, platforma musi wspierać wszystkie opisane typy pętli, takie jak hierarchiczne, rozproszone oraz federacyjne.

Warto zaznaczyć, że cechy adaptacyjności i kognitywności pętli nie są bezpośrednią funkcjonalnością platformy, która ją uruchamia, lecz są implementowane w zewnętrznych względem niej modułach sztucznej inteligencji. Na przykład w architekturze ETSI ZSM te aspekty są odzwierciedlone w serwisach zarządzania. Zdecydowano więc, że platforma nie przetwarza \hyperlink{def:czesc-obliczeniowa}{część obliczeniowej logiki pętli} bezpośrednio, lecz deleguje ją do zewnętrznych komponentów.

Porównanie pętli OODA z MAPE-K pokazuje, że modelowane pętle mogą być sekwencyjne lub współbieżne. Sekwencyjność oznacza wykonanie operacji poszczególnych elementów w określonej kolejności, natomiast brak sekwencyjności wskazuje na współbieżne działanie elementów. W projekcie platformy przyjmujemy, że współbieżność jest bardziej ogólnym przypadkiem, na podstawie którego można implementować sekwencyjność. Sekwencyjność w środowisku współbieżnym może być osiągana poprzez pobudzanie elementów w ustalonej kolejności.

Z kolei pętle niesekwencyjne (jak OODA) działają w sposób ciągły, oczekując na jeden z trzech sygnałów: feedback, feedforward lub sygnał ze środowiska zewnętrznego. Dlatego definiujemy wymaganie, że element pętli jest aktywowany przez dane wejściowe (co wpisuje się w pryncypia architektury ETSI ZSM). Jak pokazuje model OODA, element pętli może pobudzać wiele innych elementów, a nie tylko jeden, a w niektórych przypadkach komunikacja może obejmować także środowisko zewnętrzne.

Struktura wewnętrzna elementu „Orient” w modelu OODA może być skomplikowana i wielowarstwowa, obejmując wewnętrzne operacje, a nawet dodatkowe pętle sterowania. Z kolei element „Decide” może zmieniać przebieg iteracji pętli, co oznacza, że każda iteracja może wyglądać inaczej, w zależności od podjętych decyzji.

Tak jak wskazuje \cite{fallon2019}, modelowanie pętli sterowania powinno być ustandaryzowane i ujednolicone, tak aby zapoznanie się z jedną notacją lub językiem umożliwiało jej wielokrotne użycie w różnych scenariuszach. Notacja powinna być intuicyjna i nie wymagać wyspecjalizowanej wiedzy technicznej, co zwiększyłoby potencjalne grono użytkowników platformy.

W idealnym przypadku platforma powinna oferować graficzny interfejs użytkownika umożliwiający łatwe modelowanie pętli, który w tle dokonywałby translacji na ustandaryzowaną notację. Element wykonawczy, odpowiedzialny za interpretację notacji opisującej workflow pętli, jak i sama notacja, powinny być neutralne technologicznie, nie narzucając żadnej konkretnej logiki ani schematu działania pętli.

Jedna pętla sterowania zazwyczaj obsługuje określoną funkcję zarządzania, konkretny system zarządzany lub domenę zarządzania. Konieczne jest jednak uogólnienie tego pojęcia, aby zapewnić większą elastyczność w definiowaniu zakresu działania poszczególnych pętli.

Ponieważ architektury ZSM oraz CLADRA są przystosowane do pracy w środowisku wielodostawczym (multi-vendor environment), platforma Lupus powinna wspierać konwersję danych wejściowych na jednolity model wewnętrzny, podobnie jak proponuje to pętla FOCALE v3 \cite{strassner2009}. Istotne jest jednak to, aby nie narzucać żadnego konkretnego modelu.

Dodatkowo, aby systemy zarządzane przez z Lupus nie musiały być modyfikowane, konieczne jest wprowadzenie komponentu pośredniczącego, który będzie pełnił dwie kluczowe funkcje:
\begin{itemize}
    \item Integracja systemu zarządzanego z Lupus bez konieczności jego modyfikacji.
    \item Translacja zbieranych danych na format wewnętrzny platformy.
\end{itemize}
Z powyższych rozważań zdefiniowano następujące wymagania:
\begin{enumerate}
    \setcounter{enumi}{3}
    \item \label{req:4} Platforma musi wspierać komunikację między elementami pętli poprzez ich „pobudzanie” danymi.
    \item \label{req:5} Platforma musi umożliwiać koordynację pomiędzy pętlami w sposób hierarchiczny, federacyjny oraz rekurencyjny.
    \item \label{req:6} Elementy pętli działają współbieżnie.
    \item \label{req:7} Część obliczeniowa logiki pętli musi być delegowana do komponentów zewnętrznych.
    \item \label{req:8} Element pętli może pobudzać wiele innych elementów lub komunikować się ze środowiskiem zewnętrznym.
    \item \label{req:9} Element pętli może zawierać wewnętrzne workflow, umożliwiające m.in. podejmowanie decyzji zmieniających przebieg pętli w danej iteracji.
    \item \label{req:10} Elementy pętli są sterowane danymi (data-driven) i nie powinny mieć narzuconej struktury działania.
    \item \label{req:11} Modelowanie pętli powinno być ustandaryzowane i ujednolicone poprzez określoną notację lub język.
    \item \label{req:12} Zastosowana notacja nie powinna wymagać specjalistycznych umiejętności technicznych.
    \item \label{req:13} Kod w notacji powinien móc być generowany za pomocą interfejsu graficznego.
    \item \label{req:14} Notacja powinna w jednolity sposób definiować zarządzany przez siebie byt.
    \item \label{req:15} Pętla komunikująca się ze środowiskiem zewnętrznym powinna wykorzystywać komponent pośredniczący, który umożliwia translację formatów danych oraz integrację.
\end{enumerate}

Należy doprecyzować pojęcia środowiska zewnętrznego oraz komponentów zewnętrznych.
\begin{itemize}
    \item Komponenty zewnętrzne to elementy nienależące bezpośrednio do platformy, lecz funkcjonujące w obrębie lub na rzecz systemu zarządzania, takiego jak ENI, ZSM czy CLADRA. Mogą one realizować określone funkcje obliczeniowe, analityczne lub decyzyjne, wspierając działanie platformy w kontekście zarządzania siecią.
    \item Środowisko zewnętrzne odnosi się do systemów spoza samego systemu zarządzania, specyfikowanego przez ETSI lub TM Forum, i obejmuje zazwyczaj infrastrukturę sieciową oraz inne elementy teleinformatyczne, którymi zarządza platforma. W odróżnieniu od komponentów zewnętrznych, środowisko zewnętrzne nie jest częścią systemu zarządzania, lecz stanowi jego obiekt operacyjny.
\end{itemize}

\subsubsection{Wymagania na zarządzanie pętlami}

Wymagania w tej kategorii są jednoznacznie określone i nie wymagają dodatkowych rozważań, ponieważ aspekt zarządzania pętlami został kompleksowo opisany w architekturze CLADRA jako zestaw funkcjonalności Menedżera Zamkniętych Pętli. Pełen zakres tych funkcjonalności został przedstawiony w formie tabeli w \hyperlink{appendix:11}{Załączniku 11}.

W związku z tym, niniejszy podrozdział definiuje jedno zbiorcze wymaganie, obejmujące zapewnienie wszystkich funkcjonalności Menedżera Zamkniętych Pętli zgodnie z architekturą CLADRA:

\begin{enumerate}
    \setcounter{enumi}{15} 
    \item \label{req:16} Platforma powinna zapewniać wszystkie funkcje zarządzania pętlami zdefiniowane w \hyperlink{appendix:11}{Załączniku 11}.
\end{enumerate}

\subsubsection{Dyskusja}

Pracę rozpoczęto od wyboru środowiska wykonawczego, które spełniałoby postawione wymagania (\ref{req:1}, \ref{req:2}, \ref{req:3}). Oprogramowaniem, które od razu przychodzi na myśli w zgodzie z hasłem cloud-native jest Kubernetes\footnote{\url{https://kubernetes.io}}. Jego architektura została zaprojektowana z myślą o modularności i skalowalności, co doskonale wpisuje się w pryncypia architektury ZSM. Ponadto, ze względu na szeroką adopcję w systemach wielu dostawców, Kubernetes jest platformą dobrze znaną w branży teleinformatycznej. 

Dodatkowo, Kubernetes swoim wbudowanym zasobom (ang. \textit{built-in resources}) takim jak Pody czy Wdrożenia (ang. \textit{deployments}) zapewnia funkcjonalności definiowane przez Wymaganie \ref{req:16}. To skłoniło do postawienia hipotezy, że jeśli pętle jako byt mogły występować tak samo jak zasoby wbudowane Kubernetes i podlegać dokładnie takiemu samemu cyklowi zarządzania, wybór Kuberenetes jako środowiska wykonawczego pozwoliłby spełnić zarazem wymagania \ref{req:1}-\ref{req:3} oraz \ref{req:16}. 

Krótka analiza wykazała, że jest to możliwe, ponieważ Kubernetes umożliwia definiowanie tzw. zasobów własnych (Custom Resources)\footnote{\url{https://kubernetes.io/docs/concepts/extend-kubernetes/api-extension/custom-resources/}}, które można rejestrować w klastrze i używać w taki sam sposób jak zasoby wbudowane. W ramach dalszych badań dokładnie przeanalizowano ten koncept.

Każdy zasób wbudowany w Kubernetes ma przypisany kontroler, który reguluje jego działanie\footnote{\url{https://kubernetes.io/docs/concepts/architecture/controller/}}. W przypadku zasobów własnych możliwe jest ich rozwijanie za pomocą Wzorca Operatora\footnote{\url{https://kubernetes.io/docs/concepts/extend-kubernetes/operator/}} (ang. \textit{Operator Pattern}). Kontrolery dla zasobów własnych nazywamy operatorami, a ich tworzenie umożliwia framework Kubebuilder\footnote{\url{https://book.kubebuilder.io}}.



Zmaterializowanie zamknietej pęli sterowania jako zasób Kubernetes spełnia wymaganie \ref{req:16} w następujący sposób:
\begin{enumerate}[label=Fx.\arabic*] % Dodajemy prefix "Fx." przed numerem
    \item Pętla może być zdefiniowana jako zasób Kubernetes poprzez plik manifestowy YAML, zawierającym m.in. nazwę oraz opis celu.
    \item Workflow pętli może być również zdefiniowany w pliku YAML.
    \item Proces wdrażania pętli polega na utworzeniu jej pliku YAML.
    \item Instancja pętli jest tworzona poprzez zaaplikowanie pliku YAML (kubectl apply -f <filename>), po czym jej cyklem życia zarządza Kubernetes.
    \item Monitorowanie możliwe jest poprzez komendę \texttt{kubectl describe} lub śledzenie logów operatora.
    \item Zakończenie działania pętli możliwe jest za pomocą komendy \texttt{kubectl delete}.
    \item Usunięcie pętli polega na usunięciu jej pliku YAML.
    \item Odkrywanie pętli polega na wyszukaniu określonego typu zasobów w kubernetes (np. \texttt{kubectl get loops}).
    \item Ochrona pętli realizowana poprzez zabezpieczenie klastra Kubernetes.
    \item Zmiana pętli polega na modyfikacji jej pliku YAML i ponownym zaaplikowaniu. 
    \item Walidacja polega na sprawdzeniu składni pliku YAML, co może być realizowane poprzez odpowiednie mechanizmy walidacyjne zasobów własnych w Kubebuilderze.
    \item Przechowywanie pętli polega na przechowywaniu plików YAML w systemie plików (ang. \textit{filesystem}).
    \item Kontrola pętli polega na działaniach utrzymaniowych analogicznych jak w przypadku zasobów wbudowanych Kubernetes.
    \item Rekonfiguracja polega na modyfikacja pliku YAML lub rekonfiguracji komponentów pętli, która odbywa się poza Kubernetes
    \item Ekspozycja jest możliwa poprzez API Kubernetes
    \item Orkiestracja jest możliwa poprzez narzędzia oferowane jako dodatki do Kubernetes
    \item Wstrzymanie pętli polega by na usunięciu instancji jej zasobu (\texttt{kubectl delete}). Jest to akceptowalne ze względu na bezstanowość pętli lub możliwość zapisu stanu w komponentach zewnętrznych.
    \item Przywrócenie pętli osiągane jest poprzez ponowne zaaplikowanie pliku YAML.
\end{enumerate}
 
W ten sposób pokryto dwie kluczowe grupy wymagań: na środowisko wykonawcze oraz na zarządzanie pętlami. 

Dalsza część pracy koncentruje się na eksploracji możliwości spełnienia wymagań dotyczących modelowania pętli. W związku z wyborem Kubernetes oraz Wzorca Operatora, modelowanie pętli oraz urzeczywistnienie tego modelowania, czyli wykonanie logiki (workflow) pętli opiera się na propozycji odpowiedniej notacji w plikach YAML oraz zaprojektowaniu operatora odpowiedzialnego za jej interpretację.

Kontrolery w Kubernetes działają na zasadzie zamkniętej pętli sterowania. Gdy w zarządzanym przez nie obiekcie zachodzi zmiana, kontrolery zostają o tym poinformowane. Ich zadaniem jest porównanie aktualnego stanu obiektu ze stanem pożądanym, a jeśli występują rozbieżności, wykonanie operacji mających na celu doprowadzenie systemu do stanu zgodnego z oczekiwanym. Proces ten nazywany jest rekoncyliacją (reconciliation) i oznacza on pogodzenie ze sobą dwóch chwilowo rozbieżnych stanów.

Stan aktualny wyrażony jest jako pole status danego zasobu i przechowywany w bazie etcd\footnote{\url{https://bulldogjob.pl/readme/etcd-czyli-mozg-klastra-kubernetes}}. Natomiast stan pożądany definiowany jest w pliku YAML w polu spec i również zapisywany w etcd. Gdy którykolwiek z tych stanów ulegnie zmianie, warstwa sterowania Kubernetes pobudza kontroler, przekazując mu żądanie rekoncyliacji.

Mechanizm ten sugeruje możliwość realizacji wymagania \ref{req:4} dotyczącego komunikacji elementów pętli. Komunikacja mogłaby polegać na wzajemnym pobudzaniu się elementów poprzez modyfikację pola status w ich reprezentacjach Kubernetes. W związku z tym podjęto decyzję, że każdy element pętli będzie reprezentowany jako obiekt Kubernetes\footnote{\url{https://kubernetes.io/docs/concepts/overview/working-with-objects/}} typu własnego o nazwie \texttt{Element}.

Dane wykorzystywane do pobudzania elementów pętli byłyby wpisywane do pola \texttt{status}, które operator może odczytać podczas rekoncyliacji obiektu. W celu odróżnienia od pojedynczych elementów pętli, zasób własny reprezentujący całą pętlę nazwano \texttt{Master}.

Przekształcenie elementu pętli w obiekt Kubernetes pozwala spełnić wymaganie \ref{req:6}, ponieważ obiekty Kubernetes istnieją stale w klastrze, a ich kontrolery mogą zostać wywołane w dowolnym momencie poprzez żądanie rekoncyliacji.

Zasób \texttt{Element} reprezentuje pojedynczy element pętli.
Zasób \texttt{Master} odpowiada za instancjonowanie elementów oraz zarządzanie ich cyklem życia.
Operator zasobu \texttt{Master} jest uruchamiany raz podczas instancjonowania obiektu. W tym momencie odczytuje on plik YAML, który zawiera definicję pętli i określa, jakie elementy należy utworzyć.

Ponieważ elementy pętli specyfikują jej model, notacja używana do ich definiowania powinna być zestandaryzowana i ujednolicona, zgodnie z wymaganiem \ref{req:11}. Notację tę nazwano LupN.

Notacja LupN opisuje elementy pętli sterowania oraz komunikację między nimi. Opiera się na obiektach, które są interpretowane przez operator podczas pobudzenia danego elementu.

Każdy element pętli składa się z dwóch kluczowych obiektów:
\begin{itemize}
    \item Obiekt akcji – definiuje wewnętrzne workflow elementu, co pozwala spełnić wymaganie \ref{req:9}.
    \item Obiekt \texttt{next} – reprezentuje komponenty, do których element przekazuje swoje dane wyjściowe.
\end{itemize}

Aby spełnić wymaganie \ref{req:8}, obiekt \texttt{next} może wskazywać na:
\begin{itemize}
    \item inne elementy pętli,
    \item komponenty systemu zarządzania, które odpowiadają za część obliczeniową logiki pętli,
    \item środowisko zewnętrzne (w przypadku elementów końcowych pętli).
\end{itemize}

Komponenty odpowiedzialne za przetwarzanie logiki pętli oraz środowisko zewnętrzne są reprezentowane przez obiekt Destynacji, który specyfikuje komunikację za pośrednictwem protokołu HTTP. Wybór wsparcia dla HTTP wynika z jego szerokiej adopcji w systemach rozproszonych opartych na architekturze mikroserwisowej. Zakładamy więc, że powłoka integracyjna architektur ZSM znajdzie swoje implementacje w oparciu o komunikacje HTTP.

Wewnętrzne workflow elementu specyfikuje komunikację z komponentami zewnętrznymi oraz podejmowanie decyzji dotyczących rozgałęziania (fork) przebiegu pętli sterowania. Specyfikacja ta wyrażana jest za pomocą łańcucha obiektów \textbf{akcji} (\texttt{actions}), które ściśle współpracują z \textbf{danymi}. Zgodnie z wymaganiem \ref{req:10} operator elementu musi być ogólny i posiadać interpreter akcji, które mają pozwolić na elastyczną i opartą o dane orkiestrację wykonywanych operacji zarządzania. 

Dane pojawiają się w elemencie jako wartość wpisana przez poprzedni element w polu \texttt{status} i stanowią nośnk informacji podczas pojedynczej iteracji pętli. Jako format umożliwiający reprezentację dowolnych danych uznano JSON, który zapewnia elastyczność i powszechną kompatybilność w systemach rozproszonych.

Elementy współpracują z danymi poprzez obiekty akcji. Akcje występują w różnych typach, które razem tworzą zestaw operacji manipulujących danymi. Podczas projektowania typów akcji postawiono na atomowość i elastyczność. Kluczwym typem akcji jest \texttt{send}, który umożliwia komunikację z komponentami zewnętrznymi poprzez protokół HTTP, spełniając wymaganie \ref{req:7}. Akcja \texttt{send} specyfikuje:
\begin{itemize}
    \item endpoint, na który należy wysłać żądanie HTTP,
    \item które pole JSON należy wysłać w żądaniu HTTP,
    \item w którym polu należy zapisać odpowiedź.
\end{itemize}


Inne typy akcji służą do odpowiedniej organizacji danych. Pola danych można łączyć, usuwać, zmieniać ich nazwy oraz duplikować. Dane są wykorzystywane głównie do zapisywania parametrów wysyłanych do komponentów zewnętrznych oraz przechowywania otrzymanych odpowiedzi. Specjalnym typem akcji jest \texttt{switch}, który pozwala wykonać wyrażenie warunkowe na danych i na jego podstawie uzależnić wybór następnej akcji do wykonania (wymaganie \ref{req:10}).

Należy podkreślić, że dla osób pracujących z Kubernetes naturalnym sposobem sterowania są operatory Kubernetes. Bardziej zaawansowany użytkownik może implementować logikę pętli poprzez modyfikacje kodu operatorów. Niezależnie od poziomu zaawansowania, każdy użytkownik ma możliwość delegowania logiki do komponentów zewnętrznych.

Pozostaje jeszcze kwestia komunikacji elementów pętli ze środowiskiem zewnętrznym. W celu spełnienia wymagania \ref{req:15} zdefiniowano komponenty, które nazwano agentami translacji Zdecydowano, że to użytkownik platformy najlepiej zna specyfikę swojego środowiska zewnętrznego, dlatego odpowiedzialność za implementację agentów translacji pozostaje po jego stronie. Zostały jednakże zdefiniowane interfejsy, które musi spełniać agent translacji w celu komunikacji z Lupus.

Wprowadzone w niniejszym podrozdziale pojęcia takie jak akcje czy dane są szerzej omówione w dalszej części pracy. 

Poniżej prezentowane jest spełnienie wymagań dotyczących modelowania pętli:
\begin{enumerate}
    \setcounter{enumi}{15}
    \item Elementy pętli są pobudzane poprzez zmianę statusu obiektu Kubernetes.
    \item Dowolna koordynacja pracy pętli jest możliwa poprzez odpowiednią implementację wejściowego agenta translacji. Jeśli jego serwer HTTP umożliwia aktywowanie pętli z zewnątrz, nic nie stoi na przeszkodzie, aby jedna pętla mogła uruchomić inną.
    \item Wymaganie spełnione poprzez implementacje elementów jako obiekty Kubernetes.
    \item Operator elementu może używać akcji typu \texttt{send} do komunikacji z komponentem zewnętrznym za pomocą protokołu HTTP. Ponadto posiada mechanizm przechowywania danych, umożliwiający zapis wyników operacji.
    \item Obiekt notacji LupN wskazujący następny komponent pętli pozwala na definiowanie wielu odbiorców. Kolejny komponent może być innym elementem pętli lub dowolnym serwerem HTTP, zwłaszcza należącym do środowiska zewnętrznego.
    \item Obiekt notacji LupN opisujący wnętrze elementu zawiera łańcuch akcji, które stanowią jego wewnętrzne workflow. Pozwala ono sterować przebiegiem pętli za pomocą akcji typu \texttt{switch}. 
    \item Operator elementu posiada interpreter akcji, co pozwala na elastyczność.
    \item Modelowanie pętli odbywa się w notacji LupN, której specyfikacja znajduje się w dalszej części pracy.
    \item Kod notacji LupN jest oparty na obiektach YAML i charakteryzuje się prostą składnią, łatwą do przyswojenia.
    \item Możliwa jest implementacja narzędzia umożliwiającego konwersję modeli graficznych na kod LupN.
    \item Architektura definiuje komponenty zwane Agentami Translacji, które zapewniają zgodność ze środowiskiem zewnętrznym.
\end{enumerate}

Z uwagi na fakt, że warstwa sterowania Kubernetes nie działa w czasie rzeczywistym, zamknięte pętle sterowania implementowane na platformie również nie będą działały w ten sposób. Pętle sterowania wymagające operacji w czasie rzeczywistym nie należą do domeny zarządzania, lecz są częścią mechanizmów automatyzacji implementowanych bezpośrednio w elementach sieciowych. W takich przypadkach decyzje są podejmowane lokalnie, bez udziału systemów nadrzędnych.