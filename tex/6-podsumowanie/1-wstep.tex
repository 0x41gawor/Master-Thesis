\subsection{Wstęp}
Niniejszy rozdział przedstawia wyzwania, z którymi się mierzono podczas pracy, analizuje wyniki w odniesieniu do założeń i celów projektu, omawia możliwe ograniczenia uzyskanych rezultatów, porównuje je z istniejącymi rozwiązaniami oraz przedstawia perspektywy dalszego rozwoju.

\subsection{Wyzwania}

W trakcie realizacji projektu napotkano liczne wyzwania, w tym:
\begin{itemize}
    \item Dogłębne zrozumienie platformy Kubernetes – konieczne było opanowanie nie tylko jej podstaw operacyjnych, ale także mechanizmów działania warstwy sterowania. Tworzenie własnych operatorów jest uważane za zaawansowane zadanie, wymagające szczegółowej znajomości architektury Kubernetes.
    \item Projektowanie operatora bez wbudowanej logiki pętli – znalezienie odpowiedniego podejścia wymagało kilku iteracji implementacyjnych. Początkowo każdy element pętli był osobnym zasobem z własnym operatorem. Stopniowe identyfikowanie części wspólnych doprowadziło do stworzenia jednego uniwersalnego typu elementu oraz interpretera akcji w operatorze.
    \item Implementacja obiektu \textbf{danych} – ze względu na dynamicznie określaną w czasie działania strukturę, trudne było jego odwzorowanie w języku Go, który jest silnie typowany. Dodatkowo, nawet po skutecznej reprezentacji w pamięci, manipulowanie danymi okazało się wyzwaniem. Szczególnie czasochłonne były kwestie związane z obsługą zagnieżdżonych pól.
    \item Wdrożenie platformy Open5GS-k8s do testowania – wymagało znacznego nakładu pracy, zwłaszcza w zakresie podłączania zewnętrznych urządzeń UE. Proces ten wiązał się z dogłębną analizą oraz nauką rozszerzeń sieciowych wykorzystywanych przez autora repozytorium\footnote{\url{https://github.com/niloysh/open5gs-k8s/issues/7}}.
    \item Złożoność dokumentacji – ze względu na skalę projektu oraz jego charakter, w którym efektem końcowym nie jest konkretna funkcjonalność, lecz propozycja frameworka dla użytkowników, opracowanie kompletnej i przejrzystej dokumentacji okazało się czasochłonne.
    \item 
\end{itemize}

\subsection{Analiza w odniesieniu do założeń} 
Uzyskano platformę, za pomocą której można modelować, uruchamiać oraz zarządzać zamkniętymi pętlami sterowania, co potwierdzają wyniki przedstawione w rozdziale \ref{sec:5}. 
Wszystkie wymagania zdefiniowane w podrozdziale \ref{sec:32} zostały spełnione. Jedynym brakiem można uznać niezaimplementowany w PoC interfejs graficzny określony w wymaganiu \ref{req:13}. 
\subsection{Ograniczenia}

Jednym z głównych ograniczeń jest długość plików kodu LupN. Ich rozbudowana struktura może wpływać na czytelność oraz utrudniać analizę i edycję. Kolejnym ograniczeniem może być nadmierna ogólność – duży ciężar implementacji spoczywa na użytkowniku platformy. 

\subsection{Porównanie z istniejącymi rozwiązaniami}

Jedynym bezpośrednio porównywalnym rozwiązaniem jest ONAP/CLAMP. Proponowana platforma oferuje dużo większą elastyczność modelowanych pętli oraz daje dowolność wyboru silnika polityk.

\subsection{Możliwości rozwoju}

W przypadku komercyjnej implementacji proponowanej architektury możliwe jest uzyskanie platformy, która może być integralną częścią systemów zarządzania sieciami telekomunikacyjnymi zgodnymi ze specyfikacjami ETSI ENI, ETSI ZSM oraz TM Forum.

