\subsection{Wstęp}
Niniejszy rozdział przedstawia wyzwania, z którymi się mierzono podczas pracy, analizuje wyniki w odniesieniu do założeń i celów projektu, omawia możliwe ograniczenia uzyskanych rezultatów, porównuje je z istniejącymi rozwiązaniami oraz przedstawia perspektywy dalszego rozwoju.

\subsection{Wyzwania}

W trakcie realizacji projektu napotkano liczne wyzwania, w tym:
\begin{itemize}
    \item Konieczne było dogłębne zrozumienie platformy Kubernetes – nie tylko jej podstaw na poziomie operacyjnym, ale także mechanizmów działania warstwy sterowania. Tworzenie własnych operatorów jest uważane za bardzo zaawansowane zadanie w kontekście Kubernetes.
    \item Wymyślenie w jaki sposób może działać operator, który nie ma zaszytej żadnej logiki pętli zajęło kilka iteracji implementacyjnych projektu. W każdej przybliżano się do celu od zaimplementowania konkretnej struktury pętli, gdzie każdy element był zasobem innego typu i posiadał oddzielny operator, poprzez stopniowe wynajdywanie części wspólnych operatorów, aż do stworzenia jednego typu elementu i interpretera akcji w operatorze
    \item Jednym z największych wyzwań była implementacja obiektu \textbf{danych}. Ze względu na dynamicznie określaną w czasie działania strukturę, trudno jest reprezentować taki obiekt w Go, które jest językiem silnie typowanym. Po drugie, nawet jak już zareprezentuje się taki obiekt w pamięci, ciężko nim operować. Kwestie takie jak implementacja zagnieżdżania pól czy obsługa dzikiej karty (\texttt{*}) były bardzo czasochłonne.
    \item Dużo pracy poświęcono wdrożeniu platformy Open5GS-k8s do testowania. Aspekt podłączenia zewnętrznego UE wymagał czasochłonnej analizy i edukacji rozszerzeń sieciowych używanych przez autora repozytorium\footnote{\url{https://github.com/niloysh/open5gs-k8s/issues/7}}.
    \item Z racji skali projektu oraz jego naturze, gdzie efektem końcowym nie jest uzyskanie danej funkcjonalności, a zaproponowanie framework'u użytkownikom, czasochłonny był proces dokumentacji platformy
\end{itemize}

\subsection{Analiza w odniesieniu do założeń} 

Wszystkie wymagania zdefiniowane w podrozdziale \ref{sec:32} zostały spełnione. Jedynym niezrealizowanym wymaganiem pozostaje interfejs graficzny, określony w wymaganiu \ref{req:13}. Uzyskano platformę, za pomocą której można modelować zamknięte pętle sterowania, co potwierdzają wyniki przedstawione w rozdziale \ref{sec:5}. 

\subsection{Ograniczenia}

Jednym z głównych ograniczeń jest długość plików kodu LupN. Są one bardzo długie, przez co ich czytelność jest ograniczona. Kolejnym ograniczeniem może być nadmierna ogólność – duży ciężar implementacji spoczywa na użytkowniku platformy.

\subsection{Porównanie z istniejącymi rozwiązaniami}

Jedynym bezpośrednio porównywalnym rozwiązaniem jest ONAP/CLAMP. Proponowana platforma oferuje dużo większą elastyczność modelowanych pętli oraz daje dowolność wyboru silnika polityk.

\subsection{Możliwości rozwoju}

Proponowana architektura, po wdrożeniu komercyjnym, może stać się integralną częścią systemów zarządzania zgodnych ze specyfikacjami ETSI ENI, ETSI ZSM lub TM Forum.

