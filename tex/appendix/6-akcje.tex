\appendix{Specyfikacja akcji}\label{appendix:6}

Specyfikacja w formie elektronicznej znajduje się pod linkiem: \url{https://github.com/0x41gawor/lupus/blob/master/docs/spec/actions.md}.

Specyfikacja akcji w LupN znajduje się w załączniku //TODO. Ten dokument prezentuje przykładowe działanie \textbf{akcji} na \textbf{danych}.

Akcje zostały opracowane jako najbardziej atomowe operacje, które, gdy zostaną odpowiednio połączone, stanowią narzędzie umożliwiające \textbf{designer'owi pętli} pełne operowanie na \textbf{danych}.

Czasami operacja, która na pierwszy rzut oka wydaje się atomowa, wymaga użycia dwóch połączonych akcji. Z drugiej strony, zdarza się, że operacja początkowo uznana za atomową okazuje się jedynie szczególnym przypadkiem bardziej ogólnej operacji. Dobrym przykładem jest nieistniejąca już akcja \texttt{concat}. Została ona zaprojektowana do łączenia dwóch pól w jedno, jednak okazało się, że jest to specyficzny przypadek akcji \texttt{nest}, w której lista \texttt{InputKey} zawiera tylko dwa elementy.

\subsection{Podział ogólny}

Mamy 8 typów akcji:

\begin{itemize}
    \item \textbf{Send}
    \item \textbf{Nest}
    \item \textbf{Remove}
    \item \textbf{Rename}
    \item \textbf{Duplicate}
    \item \textbf{Insert}
    \item \textbf{Print}
    \item \textbf{Switch}
\end{itemize}

Możemy wyróżnić następujące kategorie:

\begin{itemize}
    \item 6 akcji, które mogą być używane do modyfikacji danych: \\ 
          \texttt{\{Send, Nest, Remove, Rename, Duplicate, Insert\}}
    \item 1 akcja do komunikacji z \textbf{Elementami Zewnętrznymi}: \\ 
          \texttt{\{Send\}}
    \item 2 akcje do debugowania: \\ 
          \texttt{\{Insert, Print\}}
    \item 1 akcja do logowania: \\ 
          \texttt{\{Print\}}
    \item 1 akcja do sterowania przepływem \textbf{workflow akcji}: \\ 
          \texttt{\{Switch\}}
\end{itemize}
