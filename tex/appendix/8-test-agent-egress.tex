\appendix{Kod Agenta Egress}\hypertarget{appendix:8}{}

Kod w wersji elektronicznej znajduje się pod linkiem: \url{https://github.com/0x41gawor/lupus/blob/master/examples/open5gs/sample-loop/egress-agent.py}

\begin{lstlisting}[language=python, basicstyle=\ttfamily\tiny, caption={\emph{Kod Agenta Egress}}\label{lst:a81}]
from flask import Flask, request, jsonify
from kubernetes import client, config
from kubernetes.client.rest import ApiException

app = Flask(__name__)

# Load Kubernetes config (use kubeconfig if running locally, in-cluster config if running in a cluster)
try:
    config.load_incluster_config()
except:
    config.load_kube_config()

# Function to patch deployment resources
def patch_deployment_resources(namespace, deployment_name, resource_type, cpu, memory):
    try:
        api_instance = client.AppsV1Api()

        # Define the patch
        patch = {
            "spec": {
                "template": {
                    "spec": {
                        "containers": [
                            {
                                "name": "upf",
                                "resources": {
                                    resource_type.split(".")[-1]: {  # Extract "requests" or "limits"
                                        "cpu": cpu,
                                        "memory": memory
                                    }
                                }
                            }
                        ]
                    }
                }
            }
        }

        # Patch the deployment
        api_response = api_instance.patch_namespaced_deployment(name=deployment_name, namespace=namespace, body=patch)
        return api_response

    except ApiException as e:
        return str(e)

import requests

def send_interval_request(url: str, value: int):
    headers = {
        "Content-Type": "application/json"
    }
    data = {
        "value": value
    }
    try:
        response = requests.post(url, json=data, headers=headers)
        response.raise_for_status()  # Raise an exception for HTTP errors
        return response.json()
    except requests.RequestException as e:
        print(f"Request failed: {e}")
        return None



@app.route('/api/data', methods=['POST'])
def get_data():
    data = request.get_json()
    print(data)
    if "spec" in data:
        deployment_name = data.get('name')
        namespace = 'open5gs'
        lim_cpu = data['spec']['limits'].get('cpu')
        lim_ram = data['spec']['limits'].get('memory')
        req_cpu = data['spec']['requests'].get('cpu')
        req_ram = data['spec']['requests'].get('memory')
        res1 = patch_deployment_resources(namespace, deployment_name, 'resources.limits', lim_cpu, lim_ram)
        res2 = patch_deployment_resources(namespace, deployment_name, 'resources.requests', req_cpu, req_ram)

    if "interval" in data:
        interval = data['interval']
        response = send_interval_request("http://192.168.56.112:9000/api/interval", interval)
    
    return jsonify({"res": "ok"})

if __name__ == '__main__':
    app.run(host='0.0.0.0', port=9001)
\end{lstlisting}