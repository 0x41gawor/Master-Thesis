\appendix{Kod Agenta Ingress}\hypertarget{appendix:7}{}

Kod w wersji elektronicznej znajduje się pod linkiem: \url{https://github.com/0x41gawor/lupus/blob/master/examples/open5gs/sample-loop/ingress-agent.py}

\begin{lstlisting}[language=python, basicstyle=\ttfamily\tiny, caption={\emph{Kod Agenta Ingress}}\label{lst:a71}]
from kubernetes import client, config
import argparse
import time
import json
from datetime import datetime
import subprocess  
from kubernetes.client import CustomObjectsApi  
from flask import Flask, request, jsonify
from threading import Thread

# Load Kubernetes configuration
config.load_kube_config()

# Kubernetes API clients
core_v1_api = client.CoreV1Api()
apps_v1_api = client.AppsV1Api()

import logging
log = logging.getLogger('werkzeug')
log.setLevel(logging.ERROR)

global interval
round = 0

def get_pods_by_deployment_prefix(prefix):
    try:
        # List all deployments in the current namespace
        deployments = apps_v1_api.list_deployment_for_all_namespaces(watch=False)
        matching_deployments = [d for d in deployments.items if d.metadata.name.startswith(prefix)]
        # Find all pods associated with matching deployments
        pods = core_v1_api.list_pod_for_all_namespaces(watch=False)
        deployment_pod_map = {}
        for deployment in matching_deployments:
            filtered_pods = []
            for pod in pods.items:
                if pod.metadata.owner_references:
                    for owner in pod.metadata.owner_references:
                        if owner.kind == "ReplicaSet" and owner.name.startswith(deployment.metadata.name):
                            filtered_pods.append(pod)
            deployment_pod_map[deployment.metadata.name] = filtered_pods
        
        return deployment_pod_map
    except client.rest.ApiException as e:
        print(f"Exception when calling Kubernetes API: {e}")
        return {}

def get_pod_resources(pod):
    containers = pod.spec.containers
    resource_info = []
    for container in containers:
        resources = container.resources
        resource_info.append({
            "container_name": container.name,
            "requests": resources.requests or {},
            "limits": resources.limits or {},
        })
    return resource_info

def get_pod_actual_usage(pod_name, namespace):
    try:
        # Use kubectl top to fetch actual usage (requires metrics-server installed)
        result = subprocess.run(
            ["kubectl", "top", "pod", pod_name, "-n", namespace, "--no-headers"],
            stdout=subprocess.PIPE,
            stderr=subprocess.PIPE,
            text=True
        )
        if result.returncode == 0:
            usage_data = result.stdout.strip().split()
            if len(usage_data) >= 3:
                return {
                    "cpu": usage_data[1],
                    "memory": usage_data[2],
                }
        else:
            print(f"Error fetching actual usage for pod {pod_name}: {result.stderr}")
            return {}
    except Exception as e:
        print(f"Error running kubectl top for pod {pod_name}: {e}")
        return {}

def get_json_data():
    deployment_prefix = "open5gs-upf"
    deployment_pod_map = get_pods_by_deployment_prefix(deployment_prefix)
    metrics = {}

    for deployment_name, pods in deployment_pod_map.items():
        # Initialize deployment entry in metrics
        metrics[deployment_name] = {
            "requests": {},
            "limits": {},
            "actual": {}
        }


        for pod in pods:
            pod_name = pod.metadata.name
            namespace = pod.metadata.namespace

            # Resource requests and limits
            resource_info = get_pod_resources(pod)
            for container_info in resource_info:
                metrics[deployment_name]["requests"] = container_info["requests"]
                metrics[deployment_name]["limits"] = container_info["limits"]

            # Actual resource usage
            actual_usage = get_pod_actual_usage(pod_name, namespace)
            metrics[deployment_name]["actual"] = actual_usage
    
    return json.dumps(metrics)

def send_to_kube(state):
    custom_objects_api = CustomObjectsApi()  # Use CustomObjectsApi to interact with CRDs
    try:
        # Retrieve the custom resource
        observe = custom_objects_api.get_namespaced_custom_object(
            group="lupus.gawor.io",
            version="v1",
            namespace='default',
            plural="elements",
            name='lola-demux'
        )
        
        # Update the `status.input` field with the state
        observe_status = observe.get('status', {})
        observe_status['input'] = json.loads(state)  # Convert JSON string to an object
        observe_status['lastUpdated'] = datetime.utcnow().isoformat() + "Z"  # Proper ISO 8601 format

        # Update the custom resource's status
        custom_objects_api.patch_namespaced_custom_object_status(
            group="lupus.gawor.io",
            version="v1",
            namespace='default',
            plural="elements",
            name='lola-demux',
            body={"status": observe_status}  # Send only the `status` field
        )
        print("Updated Kubernetes custom resource status successfully.")
    except Exception as e:
        print(f"Error updating custom resource: {e}")

def periodic_task():
    json_data = get_json_data()
    timestamp = datetime.utcnow().strftime('%Y/%m/%d %H:%M:%S')
    global round 
    round = round + 1
    print(timestamp + " Round: " + str(round) + "\n" + json_data)
    send_to_kube(json_data)

app = Flask(__name__)

@app.route('/api/interval', methods=['POST'])
def update_interval():
    global interval
    try:
        data = request.get_json()
        if "value" in data and isinstance(data["value"], int) and data["value"] > 0:
            interval = data["value"]
            return jsonify({"message": "Interval updated successfully", "new_interval": interval}), 200
        else:
            return jsonify({"error": "Invalid input. 'value' must be a positive integer."}), 400
    except Exception as e:
        return jsonify({"error": str(e)}), 500

if __name__ == "__main__":
    parser = argparse.ArgumentParser(description='Periodic K8s metrics fetcher')
    parser.add_argument('--interval', type=int, default=60, help='Interval in seconds for periodic task')
    args = parser.parse_args()
    
    # Declare that you are modifying the global variable
    interval = args.interval

    # Start Flask server in a separate thread
    flask_thread = Thread(target=lambda: app.run(host='0.0.0.0', port=9000, debug=False, use_reloader=False))
    flask_thread.daemon = True
    flask_thread.start()
    
    while True:
        periodic_task()
        time.sleep(interval)
\end{lstlisting}