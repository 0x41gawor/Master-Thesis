\appendix{Instalacja Lupus}\label{appendix:2}

Specyfikacja w formie elektronicznej znajduje się pod linkiem: \url{https://github.com/0x41gawor/lupus/blob/master/docs/installation.md}.

\subsection{Przedsłowie}

Instalacja \hyperlink{def:lupus}{\textbf{Lupus}} wymaga umiejętności technicznych oraz podstawowej znajomości operacyjnej Kubernetes.

\hyperlink{def:lupus}{\textbf{Lupus}} jest zaimplementowany jako projekt Kubebuilder\footnote{\url{https://book.kubebuilder.io}}. Zalecanym sposobem instalacji \hyperlink{def:lupus}{\textbf{Lupus}} jest sklonowanie tego repozytorium i przyjęcie roli dewelopera tego projektu.

Nie istnieje coś takiego jak instalacja \hyperlink{def:lupus}{\textbf{Lupus}} (np. w systemie operacyjnym). Można zainstalować \hyperlink{def:zasoby-wlasne}{\textbf{Zasoby Własne}} dla \hyperlink{def:element-lupus}{\textbf{Elementów Lupus}} w klastrze Kubernetes i uruchomić dla nich \hyperlink{def:operator-zasobu-element}{\textbf{Kontrolery}}. Niniejszy załącznik opisuje właśnie taki proces.

\subsection{Wymagania wstępne}

\hyperlink{def:uzytkownik}{\textbf{Użytkownik}} musi posiadać działający klaster Kubernetes. Może to być Minikube\footnote{\url{https://minikube.sigs.k8s.io/docs/}}, zainstalowany silnik kontenerów (ang. \textit{container engine}) (np. Docker\footnote{\url{https://docs.docker.com}}) oraz język Go\footnote{\url{https://go.dev}}.

\subsubsection{Instalacja Kubebuilder}

Instrukcja dostępna pod adresem: \url{https://book.kubebuilder.io/quick-start}.

\subsection{Klonowanie repozytorium}

\begin{lstlisting}[language=bash, caption={Klonowanie repozytorium}]
git clone https://github.com/0x41gawor/lupus
cd lupus
\end{lstlisting}

\subsection{Instalacja CRD w klastrze}

To polecenie zastosuje \hyperlink{def:crd}{\textbf{CRD}} (pl. Definicje \hyperlink{def:zasoby-wlasne}{\textbf{Zasobów Własnych}}) dla \hyperlink{def:master}{\textbf{Master}} i \hyperlink{def:element}{\textbf{Element}}, umożliwiając ich użycie.

\begin{lstlisting}[language=bash, caption={Instalacja CRD}]
make install
\end{lstlisting}

\subsection{Uruchomienie kontrolerów }

To polecenie uruchomi \textit{kontrolery} dla \textit{zasobów własnych} \hyperlink{def:master}{\textbf{master}} i \hyperlink{def:element}{\textbf{element}}.

\begin{lstlisting}[language=bash, caption={Uruchomienie kontrolerów}]
make run
\end{lstlisting}

Istnieje możliwość uruchomienia kontrolerów jako pody w klastrze Kubernetes. W tym celu użytkownik jest zaproszony do bliższego zapoznania się z platformą Kubebuilder. Dopóki \hyperlink{def:uzytkownik}{\textbf{Użytkownik}} jest pewien, że nie będzie dopisywał \hyperlink{def:funkcje-uzytkownika}{\textbf{Funkcji Użytkownika}} nie jest to zalecane podejście. 


