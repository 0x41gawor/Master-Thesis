\appendix{Definicje Lupus}\label{appendix:1}

Specyfikacja w formie elektronicznej znajduje się pod linkiem: \url{https://github.com/0x41gawor/lupus/blob/master/docs/defs.md}.

\acronym{Problem Zarządzania (ang. \textit{Management Problem})}{}

\acronym{System Zarządzany (ang. \textit{Managed System})}{}

\acronym{System Sterowania (ang. \textit{Control System})}{}
\acronym{Pętla Sterowania (ang. \textit{Control Loop})}{}
\acronym{Zamknięta Pętla Sterowania (ang. \textit{Closed Control Loop})}{}
\acronym{Akcja Sterująca (ang. \textit{Control Action})}{}
\acronym{Sprzeżenie Zwrotne (ang. \textit{Control Feedback})}{}
\acronym{Stan Aktualny (ang. \textit{Current State})}{}
\acronym{Stan Pożądany (ang. \textit{Desired State})}{}
\acronym{Pętla Sterowania (ang. \textit{Control Loop})}{}
\acronym{Wdrożenie Lupus (ang. \textit{Lupus deployment, Lupus usage or Lupus application})}{}
\acronym{Użytkownik (ang. \textit{User})}{}
\acronym{Agent Translacji (ang. \textit{Translation Agent})}{}
\acronym{Interfejs Lupin (ang. \textit{Lupin interface})}{}
\acronym{Interfejs Lupout (ang. \textit{Lupout interface})}{}
\acronym{Interfejsy Lupus} (ang. \textit{Lupus interfaces}){}
\acronym{Finalne dane (ang. \textit{Final Data})}{}
\acronym{Workflow Pętli (ang. \textit{Loop Workflow})}{}
\acronym{Logika Pętli (ang. \textit{Loop logic})}{}
\acronym{Element Pętli (ang. \textit{Loop Element})}{}
\acronym{Element Lupus (ang. \textit{Lupus Element})}{}
\acronym{Element Zewnętrzny (ang. \textit{External Element})}{}
\acronym{Część Obliczeniowa (ang. \textit{Computing Part})}{}
\acronym{LupN (ang. \textit{LupN})}{}
\acronym{Element Ingres (ang. \textit{Ingress Element})}{}
\acronym{Element Egress (ang. \textit{Egress Element})}{}
\acronym{Open Policy Agent (ang. \textit{Open Policy Agent})}{}
\acronym{Dane (ang. \textit{Data})}{}
\acronym{kod LupN (ang. \textit{LupN code})}{}
\acronym{plik LupN (ang. \textit{LupN file})}{}
\acronym{Element (ang. \textit{Element})}{}
\acronym{Master (ang. \textit{Master})}{}
\acronym{Pole Danych (ang. \textit{Data Field})}{}
\acronym{Użytkownik (ang. \textit{User})}{}
\acronym{Użytkownik (ang. \textit{User})}{}
\acronym{Użytkownik (ang. \textit{User})}{}

