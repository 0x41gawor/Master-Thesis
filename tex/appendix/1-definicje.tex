\appendix{Definicje Lupus}\label{appendix:1}

Specyfikacja w formie elektronicznej znajduje się pod linkiem: \url{https://github.com/0x41gawor/lupus/blob/master/docs/defs.md}.

\hypertarget{def:akcja}{}
\acronym{Akcja (ang. \textit{Action})}{Obiekt LupN. Za jej pomocą możliwe są operacje na danych}

\hypertarget{def:akcja-sterujaca}{}
\acronym{Akcja Sterująca (ang. \textit{Control Action})}{Akcja wywnioskowana przez System Sterowania i wykonywana na Systemie Zarządzanym przez Agenta Egress. Akcja Sterowania ma za zadanie przybliżyć Stan Aktualny do Stanu Pożądanego.}

\hypertarget{def:agent-translacji}{}
\acronym{Agent Translacji (ang. \textit{Translation Agent})}{Agent oprogramowania na styku Lupus oraz Systemu Zarządzanego pełniący rolę integratora. Termin zbiorczy na Agent Ingress i Agent Egress.}

\hypertarget{def:agent-egress}{}
\acronym{Agent Egress (ang. \textit{Egress Agent})}{Jeden z agentów translacji umieszczony na wyjściu z Lupus. Przekazuje finalne dane reprezentujące akcję sterowania do systemu zarządzanego. Implementuje interfejs Lupout.}

\hypertarget{def:agent-ingress}{}
\acronym{Agent Ingress (ang. \textit{Ingress Agent})}{Jeden z agentów translacji umieszczony na wejściu do Lupus. Przekazuje stan aktualny zarządzanego systemu do Lupus. Implementuje interfejs Lupin.}

\hypertarget{def:czesc-obliczeniowa}{}
\acronym{Część Obliczeniowa (ang. \textit{Computing Part})}{Część logiki pętli odpowiedzialna za obliczenia. Wykonywana jest przez elementy zewnętrzne, nie elementy Lupus.}

\hypertarget{def:dane}{}
\acronym{Dane (ang. \textit{Data})}{Nośnik informacji w jednej iteracji pętli. Informacje zapisane są w formacie JSON.}

\hypertarget{def:destynacja}{}
\acronym{Destynacja (ang. \textit{Destination})}{Referencja do Elementu Zewnętrznego. Jednoznacznie opisuje żądanie HTTP, jakie ma wykonać Operator Zasobu Element.}

\hypertarget{def:element}{}
\acronym{Element (ang. \textit{Element})}{Typ (ang. \textit{kind}) obiektu API stanowiącego Element Lupus.}

\hypertarget{def:element-egress}{}
\acronym{Element Egress (ang. \textit{Egress Element})}{Ostatni element w danej ścieżce workflow pętli. Implementuje interfejs Lupout.}

\hypertarget{def:element-ingres}{}
\acronym{Element Ingres (ang. \textit{Ingress Element})}{Pierwszy element w workflow pętli. Implementuje interfejs Lupin.}

\hypertarget{def:element-lupus}{}
\acronym{Element Lupus (ang. \textit{Lupus Element})}{Jeden z elementów pętli. Działa w warstwie sterowania Kubernetes. Jego celem jest wyrażenie workflow pętli oraz delegacja części obliczeniowej do elementów zewnętrznych.}

\hypertarget{def:element-petli}{}
\acronym{Element Pętli (ang. \textit{Loop Element})}{Element realizujący jednostkę logiki pętli. Termin zbiorczy na Element Lupus oraz Element Zewnętrzny.}

\hypertarget{def:element-zewnetrzny}{}
\acronym{Element Zewnętrzny (ang. \textit{External Element})}{Jeden z elementów pętli. Działa poza klastrem Kubernetes. Jego celem jest część obliczeniowa logiki pętli.}

\hypertarget{def:finalne-dane}{}
\acronym{Finalne dane (ang. \textit{Final Data})}{Postać danych po wykonaniu ostatniej akcji z workflow akcji Elementu. W przypadku elementu Egress ona definiuje Akcję Sterującą.}

\hypertarget{def:funkcje-uzytkownika}{}
\acronym{Funkcje użytkownika (ang. \textit{User Functions})}{Alternatywa dla wdrażania serwerów HTTP do prostych obliczeń. Funkcje użytkownika działają jak elementy zewnętrzne, ale są wykonywane w klastrze Kubernetes. Jest to jedna z dostępnych \hyperlink{def:destynacja}{Destynacji} w \hyperlink{def:lupn}{Lupn}.}

\hypertarget{def:interfejs-lupin}{}
\acronym{Interfejs Lupin (ang. \textit{Lupin interface})}{Interfejs wejściowy do Lupus. Implementować go musi Agent Ingress oraz Element Ingress.}

\hypertarget{def:interfejs-lupout}{}
\acronym{Interfejs Lupout (ang. \textit{Lupout interface})}{Interfejs wyjściowy z Lupus. Implementować go musi Agent Egress oraz Element Egress.}

\hypertarget{def:interfejsy-lupus}{}
\acronym{Interfejsy Lupus (ang. \textit{Lupus interfaces})}{Termin zbiorczy na Interfejs Lupin oraz Interfejs Lupout.}

\hypertarget{def:kod-lupn}{}
\acronym{Kod LupN (ang. \textit{LupN code})}{Kod wyrażający notację LupN. Zbiór obiektów LupN zapisanych w YAML. Zapisany jest w pliku LupN.}

\hypertarget{def:logika-petli}{}
\acronym{Logika Pętli (ang. \textit{Loop logic})}{Kroki, które muszą zostać wykonane, aby w każdej iteracji pętli przybliżyć Stan Aktualny do Stanu Pożądanego. Instrukcje określające, jak na podstawie reprezentacji Stanu Aktualnego ma zostać \textit{wywnioskowana} Akcja Sterująca}. Na logikę pętli składa się orkiestracja pracy jej elementów oraz wykonywane przez nie przetwarzanie (tzw. \hyperlink{def:czesc-obliczeniowa}{część obliczeniowa}.

\hypertarget{def:lupn}{}
\acronym{LupN (ang. \textit{LupN})}{Notacja do wyrażania workflow pętli za pomocą obiektów LupN. Zapisywana w pliku LupN.}

\hypertarget{def:lupus}{}
\acronym{Lupus (ang. \textit{Lupus})}{System o proponowanej w pracy architekturze, który pełni rolę \hyperlink{def:system-sterowania}{Systemu Sterowania}.}

\hypertarget{def:lupus-master}{}
\acronym{Lupus Master (ang. \textit{Lupus Master})}{Nadzorca pojedynczej pętli Lupus. Odpowiedzialny za kreację obiektów Zasobu Lupus Element.}

\hypertarget{def:master}{}
\acronym{Master (ang. \textit{Master})}{Typ (ang. \textit{kind}) obiektu API dla Lupus Master.}

\hypertarget{def:opa}{}
\acronym{Open Policy Agent (ang. \textit{Open Policy Agent})}{Serwer HTTP dedykowany do definiowania polityk i stosowania ich w systemach informatycznych.}

\hypertarget{def:operator-zasobu-master}{}
\acronym{Operator Zasobu Master (ang. \textit{Master Operator})}{\textit{Kontroler} zasobu Master. Odpowiedzialny za interpretację workflow pętli notacji LupN i tworzenie obiektów Element.}

\hypertarget{def:obiekt-lupn}{}
\acronym{Obiekt LupN (ang. \textit{LupN Object})}{Obiekty składające się na składnię LupN za pomocą, których wyrażane jest workflow pętli}

\hypertarget{def:operator-zasobu-element}{}
\acronym{Operator Zasobu Element (ang. \textit{Element Operator})}{\textit{Kontroler} zasobu Element. Odpowiedzialny za interpretację workflow akcji notacji LupN i jego wykonanie.}

\hypertarget{def:plik-lupn}{}
\acronym{Plik LupN (ang. \textit{LupN file})}{Plik manifestowy YAML zasobu Master. Zapisana jest w nim notacja LupN.}

\hypertarget{def:pole-danych}{}
\acronym{Pole Danych (ang. \textit{Data Field})}{Jedno pole JSON w danych. Dostępne pod kluczem. Przekazywane jako wejście/wyjście akcji.}

\hypertarget{def:problem-zarzadzania}{}
\acronym{Problem Zarządzania (ang. \textit{Management Problem})}{Problem występujący w Systemie Zarządzanym, który System Sterowania ma za cel rozwiązać.}

\hypertarget{def:projektant}{}
\acronym{Projektant (ang. \textit{Designer})}{Jednostka użytkownika odpowiedzialna jedynie za projekt pętli. Nie musi posiadać umiejętności technicznych, a jedynie posługiwać się notacją LupN.}

\hypertarget{def:sprzezenie-zwrotne}{}
\acronym{Sprzężenie Zwrotne (ang. \textit{Control Feedback})}{Reprezentacja Stanu Aktualnego wysyłana przez (lub pobierana z) Systemu Zarządzanego.}

\hypertarget{def:stan-aktualny}{}
\acronym{Stan Aktualny (ang. \textit{Current State})}{Obserwowany w danej chwili stan Systemu Zarządzanego. Kontrastuje ze Stanem Pożądanym.}

\hypertarget{def:stan-pozadany}{}
\acronym{Stan Pożądany (ang. \textit{Desired State})}{Pożądany stan obserwowany w Systemie Zarządzanym, który System Sterowania próbuje osiągnąć. Zazwyczaj można go wywieść z Problemu Zarządzania.}

\hypertarget{def:system-sterowania}{}
\acronym{System Sterowania (ang. \textit{Control System})}{System odpowiedzialny za rozwiązanie Problemu Zarządzania w Systemie Zarządzanym za pomocą Zamkniętej Pętli Sterowania. Lupus aspiruje do tej roli.}

\hypertarget{def:system-zarzadzany}{}
\acronym{System Zarządzany (ang. \textit{Managed System})}{Dowolny system w posiadaniu Użytkownika, w którym występuje Problem Zarządzania.}

\hypertarget{def:uzytkownik}{}
\acronym{Użytkownik (ang. \textit{User})}{Organizacja lub indywidualność, która ma za cel użyć Lupus w celu rozwiązania Problemu Zarządzania w swoim Systemie Zarządzanym. W zespole użytkownika potrzebne są kompetencje programistyczne (do wdrożenia agentów translacji oraz elementów zewnętrznych), a także wyróżnia się rolę Projektanta.}

\hypertarget{def:wdrozenie-lupus}{}
\acronym{Wdrożenie Lupus (ang. \textit{Lupus deployment, Lupus usage or Lupus application})}{Pojedyncze rozwiązanie Problemu Zarządzania przez Użytkownika w danym Systemie Zarządzanym.}

\hypertarget{def:workflow-akcji}{}
\acronym{Workflow Akcji (ang. \textit{Actions Workflow})}{\textit{Workflow} złożone z Akcji w pojedynczym Elemencie Lupus. Interpretowane przez Operator Zasobu Element}.

\hypertarget{def:workflow-petli}{}
\acronym{Workflow Pętli (ang. \textit{Loop Workflow})}{\textit{Workflow} złożone z Elementów Pętli. Musi realizować Logikę Pętli.}

\hypertarget{def:zamknieta-petla-sterowania}{}
\acronym{Zamknięta Pętla Sterowania (ang. \textit{Closed Control Loop})}{Sposób, w jaki Lupus jako System Sterowania rozwiązuje Problem Zarządzania.}
