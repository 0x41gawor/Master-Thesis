\appendix{Definicje}\label{appendix:1}

\acronym{Architektura}{Zbiór reguł i metod opisujących funkcjonalność, organizację oraz implementację systemu.}\label{def:architektura}

\acronym{Workflow}{Sekwencja połączonych węzłów, czasami zależnych warunkowo, która realizuje określony cel. Zazwyczaj workflow defniuje się w celu organizacji pracy}\label{def:workflow}

\acronym{Wnioskowanie (ang. \textit{reasoning})}{Proces, w którym system wyciąga logiczne wnioski z dostępnych danych i wiedzy}\label{def:wnioskowanie}

\acronym{Kognitywność (ang. \textit{cognition})}{Proces rozumienia danych oraz informacji w celu produkcji nowych danych, informacji oraz wiedzy}\label{def:kognitywność}

\acronym{System kognitywny}{System, który uczy się, wnioskuje oraz podejmuje decyzje w sposób przypominający ludzki umysł}\label{def:system-kognitywny}