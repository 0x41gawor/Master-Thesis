\appendix{Specfyfikacju obiektu danych}\label{appendix:5}

Specyfikacja w formie elektronicznej znajduje się pod linkiem: \url{https://github.com/0x41gawor/lupus/blob/master/docs/spec/data.md}.

Data jest kluczowym elementem spełnienia \hyperref[req:5]{Wymagania 5}. \textbf{Dane} jest to sposób w jaki \textbf{użytkownik}, podczas każdej iteracji, może:
\begin{itemize}
    \item uzyskać informacje o \textbf{Aktualnym Stanie}
    \item przechowywać pomocnicze informacje (takie jak odpowiedzi od \textbf{Elementów Zewnętrznych})
    \item przechowywać informacje debuggingowe
    \item zapisywać informacje potrzebne do sformułowania \textbf{Akcji Sterowania}
\end{itemize}

\textbf{Dane} są reprezentowane jako JSON dający się zapisać w strukturze Go \texttt{map[string]interface{}}. Nie może, więc być jedną z następujących form obiektu JSON:
\begin{itemize}
    \item typem prymitywnym,
    \item tablicą,
    \item obiektem JSON z kluczami innymi niż string.
\end{itemize}

To w jaki sposób można operować na danych prezentuje specyfikacja akcji //TODO link.