\appendix{Specyfikacja obiektu danych}\label{appendix:5}

Specyfikacja w formie elektronicznej znajduje się pod linkiem: \url{https://github.com/0x41gawor/lupus/blob/master/docs/spec/data.md}.

\hyperlink{def:dane}{\textbf{Dane}} są kluczowym elementem spełnienia \hyperref[req:5]{Wymagania 5}. \hyperlink{def:dane}{\textbf{Dane}} to sposób, w jaki \hyperlink{def:uzytkownik}{\textbf{Użytkownik}}, podczas każdej iteracji, może:
\begin{itemize}
    \item uzyskać informacje o \hyperlink{def:stan-aktualny}{\textbf{Aktualnym Stanie}},
    \item przechowywać pomocnicze informacje (takie jak odpowiedzi od \hyperlink{def:element-zewnetrzny}{\textbf{Elementów Zewnętrznych}}),
    \item przechowywać informacje debuggingowe,
    \item zapisywać informacje potrzebne do sformułowania \hyperlink{def:akcja-sterujaca}{\textbf{Akcji Sterowania}}.
\end{itemize}

\hyperlink{def:dane}{\textbf{Dane}} są reprezentowane jako JSON dający się zapisać w strukturze Go \texttt{map[string]interface{}}. Nie mogą, więc być jedną z następujących form obiektu JSON:
\begin{itemize}
    \item typem prymitywnym,
    \item tablicą,
    \item obiektem JSON z kluczami innymi niż \texttt{string}.
\end{itemize}

To, w jaki sposób można operować na \hyperlink{def:dane}{\textbf{Danych}}, prezentuje specyfikacja akcji (\hyperref[appendix:6]{Załącznik 6}).

// TODO: Opisać koncepcję wildcard oraz pojęcie pola danych.
